\section{Философия Канта (продолжение)}
Лекция 11, семестр 9, 12.11.2005


\subsection{Иоганн Готлиб Фихте}

Родился в деревушке Радинау, в семье сельского ремесленника. В семье было восемь детей. С детства проявлял одарённость, но денег на обучение в школе не было, но ему повезло. В этой деревне поселился протестантский пастор, который быстро приобрёл славу за свои проповеди. Однажды богатый дворянин барон фон Мильтис приехал послушать эти проповеди, но опоздал на проповедь и маленький Фихте рассказал ему её наизусть. Это поразило барона, и он увидел насколько одарён мальчик. Он взял заботу о его образовании. Фихте устроили сначала в школу, а по окончании в закрытое дворянское заведение. Затем он поступил в университет на теологический факультет в Йене. Потом переехал в Лейпциг в поисках работы, но не нашёл. Стал домашним учителем, этим и жил.

В 1790 году познакомился с работами Канта. Пришёл в восторг от его работ и поехал повидать Канта, но он его немного разочаровал. Фихте сблизился с Кантом и в 1792 написал свою первую философскую работу – <<Опыт критики откровения>>. Вышла без автора, и везде носился слух, что автором был Кант. Работу очень хвалили (так как думали, что это Кант её написал), но Кант узнав об этом написал, что автор Фихте. И поэтому пришлось хвалить и Фихте. После этого Фихте женился, и его пригласили заведовать кафедрой философии в Йенском университете.

    Выступал в защиту якобинской диктатуры. Требовал от государей Европы свободы мысли, которую они до сих пор подавляли. Доказывал, что революция это вполне законная вещь. Говорил, что государства создаются согласно требованиям разума.

Во время преподавания в Йене познакомился с Гёте и Шиллером. Не боялся критиковать дураков и поэтому нажил много врагов. Фихте был редактором университетского журнала и однажды разрешил публикацию статьи атеистического характера, хотя и не был с ней согласен (разрешил ради свободы слова). К этому придрались, потребовали признания вины и потребовали склонить голову. Фихте отказался категорически, и его выгнали с кафедры. После этого он перебрался в Берлин, занимался преподаванием, написанием научных работ. Там вышла одна из замечательных его работ – <<Ясное как солнце сообщение широкой публике о подлинной сущности (...)ей философии>>

Выступал резко против французской оккупации, против Наполеона, считая его преступником, который задушил великую революцию. В 1808 году выпустил книгу <<Речи к немецкой нации>>, где он призывал немцев объединится, восстать и изгнать французских оккупантов. В 1810 году после создания Берлинского университета Фихте становится сначала деканом философского факультета, а затем и ректором университета, но через год уходит с этой должности. В 1814 году заразился тифом от жены и умер! Его жизнь была во многом сходна с его философией.
\subsection{Философия}

Природа интересовала Фихте меньше всего, то есть натурфилософии у него не было. В центре его внимания был Человек и Человеческий род, и их познание. У него считалось, что главная цель познания это выяснить сущность человека, познать каковы потребности человека. Какие потребности фундаментальны. Такая философия должна была помочь раскрыть эти потребности и помочь их удовлетворить. Чтобы познать человека нужно познать культуру и общества, в которой он живёт. Для этого нужно знать место общества в общей истории человечества. У Фихте была своеобразная концепция философии истории, не слишком разработанная, в основе которой лежала идея французских просветителей, идея прогресса, идея поступательного развития, но чётко стадии он не выделяет.

Общество должно быть таким, чтобы обеспечить полное раскрытие всех способностей человека. Государство нужно оценивать по степени близости к идеалу к высшим порядкам. Вначале человечество бессознательно тянулось к этим порядкам, но сейчас происходит переворот, наступает период, когда люди сознательно изменяют общество в нужном направлении.

После оккупации французами германских государств, у Фихте на первый план выходит идея национального освобождения и объединения. Не обошлось здесь и без элементов национализма. Было очевидно, что существовала потребность в национальном объединении, национальной борьбе и поэтому есть и элементы национализма и превосходства немецкой нации. Но главное всё-таки у него это демократическое национальное движение. Однако мало объединится, нужно покончить со всеми старыми институтами, то есть с феодализмом, деспотизмом, создать государство с конституционными порядками, единое демократическое немецкое государство. А дальше государство должно обеспечить развитие всяческих способностей, счастье и свободу, а для этого нужен досуг. Чтобы был досуг нужно развивать промышленность и сельское хозяйство, дабы обеспечить людей всем необходимым. Причём добиться большой производительности труда, чтобы обеспечить свободное время.

Фихте начал с кантианства, особенно нравилась ему идея активности мышления. Но это была активность теоретическая. Этого ему было мало. Ему нужна была ещё и практическая активность, то есть не просто, чтобы познавать мир, но чтобы изменять его, преобразовывать.

    У него впервые возникает идея, что философия нужна, чтобы понять и изменить Мир. По Канту человек создаёт мир для нас, но из материала, который пришёл из мира в себе, и в этом смысле человек ничего сделать не может. Поэтому Фихте отменяет вещи в себе, нет никакого мира в себе, есть только мир для нас, который целиком строится Духом.

    \textbf{Центральное понятие философии Фихте это понятие <<Я>> - Разум. Причём Разум вообще.}

Разум абстрактный, абсолютный. <<Я>> Фихте это не моё <<Я>>, это абсолютное <<Я>>, которое создаёт Мир, каким он должен быть. В результате происходит раздвоение: есть <<Я>> и есть не <<Я>>, то есть <<Я>> создаёт не <<Я>>. В это не <<Я>> входит и конкретное <<Я>> и другие предметы, но каково соотношение не ясно. То ли вещи создаёт конкретное (эмпирическое) <<Я>>, то ли абсолютное <<Я>>.

    Философия Фихте довольна противоречива. Но самое главное он пытается показать развитие этого абсолютного <<Я>> - это развитие предельно общих понятий. Как у Канта есть категории рассудка, у Фихте есть категории абсолютного <<Я>>.

Отличие состоит в том, что у Фихте категории образуют систему. Категории у него вытекают одна из другой.

    Мир категорий развивается, движется. У Фихте впервые появляется диалектика понятий. В этом огромная заслуга Фихте. Возникают идеи противоречия, в результате которых происходит движение, развитие

\subsection{Фридрих Вильгельм Йозеф Шеллинг}

Для Шеллинга характерен необычайно быстрый творческий взлёт, развитие его таланта. Поэтому Гегеля изучают после Шеллинга, хотя Гегель и старше на 5 лет.

Вюрсбург (???) Бавария. Где ему предложили должность профессора и зав кафа философии. Там тоже не ужился и переехал через 3 года в Мюнхен. Там наметился перелом в философском развитии Шеллинга. Он повернул к религии и мистике. Стал заниматься религией и открыл для себя личного Бога. В 1809 году он написал книгу <<Философия откровения>>. После этого до самой смерти он не публиковал своих работ. Во время его творческого кризиса росла слава Гегеля, и Шеллинг проникся страшной завистью, называл его плагиатором. В 1841 году, после смерти Гегеля, Шеллинг занимает его место зав кафа в Берлинском университете. Умер Шеллинг в 1854 году.
\subsubsection{Философия Шеллинга}

Шеллинг начал с фихтеанства, но потом разработал свою систему объективного идеализма. Главным у него становится природа. Но природа есть воплощение Духа который никогда не находится в покое, он развивается. И самое ценное в этом это картина развития природы. Начиная с простейших форм, когда Дух действует неосознанно, заканчивая Человеком, когда Дух наконец осознает себя. Идея поступательного развития природы есть непреходящее достижение Шеллинга. У него есть так же работы, в которых рассматривается и развитие, прогресс самого Духа. Дух начинает с простейших форм. Это ощущение, раздражение, восприятие и только потом появляется Разум. Он выделяет две формы мышления: рассудок и разум. Рассудок целиком скован законами формальной логики и поэтому не может дать полного знания. Рассудок и Разум у него выступают как две стадии мышления. Разум не скован формальной логикой. Он говорит об особой форме познания – интеллектуальное созерцание. Дальше он говорит о появлении воли, свободы. Но свободу он воспринимает как осознание необходимости.

\subsection{Георг Вильгельм Фридрих Гегель}

Родился в герцогстве Вюртенберг, в семье гос чиновника, семья была довольно обеспечена. В 1788 году поступает в Тюбингенский университет и оканчивает его в 1793 году. Большое влияние на Гегеля оказала Французская революция, он вспоминал её как восход солнца. Работал домашним учителем. Несколько лет жил в Берне. Там была более свободная атмосфера. В 1797 году возвращается в германию, работает гувернером во Франкфурте на Майне. В 1801 году он оказывается в Йенском университете. Вплоть до этого года он пишет только богословские труды. После этого он написал работу <<Различие между системой философии Фихте и Шеллинга>>. Дальше он создаёт свою собственную систему.

В 1825 году вышла его книга <<Философия Права>>. После этого работ у него выходило. Умер он в 1831 году от холеры. После его смерти вышли масса его работ, например его знаменитая <<Философия Истории>>.
