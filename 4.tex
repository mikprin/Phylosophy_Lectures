\section{Поздняя схоластика и эпоха Возрождения.}
Лекции по философии науки проф. Семенова Юрия Ивановича
Лекция 4, семестр 9, 24. 9.2005

\subsection{Поздняя схоластика}
\subsubsection{Иоанн Дунс Скотт (1270-1308)}
Данные из биографии

Окончил Оксфорд. Жил в Париже, преподавал в Кельне. Францисканец. Теология – не наука, но прямо он об этом не говорил. Но этот вывод сам собой напрашивается. Если в философии важна теория, то теология – это наука сугубо практическая. Какая цель у теологии? Научить человека найти путь к небесному блаженству. Оторвал теологию от философии, тем самым одновременно освобождал философию от влияния теологии, а с другой стороны и теологию из-под влияния философии. Для него было главным освободить философскую мысль от плена духовенства. И что самое интересное, к такому выводу пришел богослов, схоласт, но и одновременно философ.

Надо сказать, что когда он пытался обосновать вот такой подход к теологии (что она не наука, что там размышлять не о чем), он исходил из своеобразной позиции, из теологического индетерминизма. > Он заявлял, что в области божественного разума нет никакой закономерности, не действуют никакие причинные связи. Бог – это воля и воля Бога абсолютно свободна. Ничем себя Бог не ограничивает и в любое время может поменять все. Никаких закономерностей здесь быть не может, ибо божественная воля выше всего.
Философское учение

Учение о материи. Материя – это всеобщая субстанция, из которой все возникает (кроме Бога). Вещи образуются тогда, когда происходит воссоединение материи (которая является содержанием) и формы. Это как раз то, что пропагандирует Фома Аквинский. Надо сказать, что он не придерживался взгляда Аристотеля, что материя – это только возможность и становится действительностью, когда проникает форма. > С его точки зрения материя вполне самостоятельна. Материя была до формы, без формы, она сама по себе активна. В этом смысле она первична по отношению к форме и форма дает ей не просто определенность, а вносит конкретность (материя просто становится конкретными вещами). Это уже большой сдвиг от Аристотеля в направлении скорее материалистическом.

Конечно же, перед ним стояла проблема общего частного, которая разделяла всех на номиналистов и реалистов. Он был реалистом. И в этом отношении его взгляд похож на взгляд Авиценны, на взгляд Фомы Аквинского и многих других о том, что в начале идеи существуют в божественном разуме, дальше идеи существуют в вещах и, наконец, идеи существуют в нашей голове. Но надо сказать, что в отличие от многих реалистов он считал, что главное бытие – это бытие единичного (индивидуальное бытие). Самое главное, самое совершенное проявление материи – это, конечно, единичные, конкретные вещи. Главное в том смысле, что вещи являются первоначалом и их бытие – это самое совершенное бытие, в то время как для многих реалистов вещи все-таки производные от идей.

Отсюда вытекал определенный взгляд на теорию познания. С чего начинается познание мира? С познания единичных вещей. А познание единичных вещей – это чувственное познание (главная, основная форма познания). Но также он считал, что одного знания единичного не достаточно. Ощущения еще не дают нам настоящие знания. > Для того чтобы знание стало знанием, необходим интеллект, ум, необходимы понятия. И он говорил о том, что интеллект играет огромную роль. Он не просто воспринимает, не просто помнит обо всех прибывших к нам представлениях, он их обрабатывает, дополняет. То есть он высказал идею активности интеллекта.

Этот интеллект обрабатывает данные чувственного познания и превращает чувственное в полное, настоящее, подлинное человеческое знание.

Критический релятивизм. Надо сказать, что его точка зрения, согласно которой в мире не действуют причинно следственные связи, мир индетерминирован легла в основу того, что можно назвать критическим релятивизмом. То есть он считал, что мораль может быть для всех разной, и доказывал он это на священописании: он говорил, что у нас в Европе единственной законной формой брака является моногамия, но если мы почитаем библию, то узнаем, что все патриархи были многоженцами. Нет абсолютно однозначного набора правил, что входило в противоречия с основными положениями, пропагандирующимися в то время (святые нормы которые нужно абсолютно соблюдать).

Надо сказать, что эти идеи в дальнейшем получили развитие в работах другого крупнейшего последнего великого схоласта западной Европы Уильяма Оккама.
\subsubsection{Уильям Оккам (1285 - 1349)}
Биография

Родился в английской деревушке неподалеку от Лондона. Учился в Оксфорде. Вступил в орден Францисканцев. Надо упомянуть, что в то время когда жил Оккам, разразился конфликт между орденом и папой Римским. Наиболее радикальные францисканцы осуждали накопление богатств священнослужителями. В тоже время папы стремились к светской власти, пытались распространить свое влияние на всех монархов Европы. И то время было временем ожесточенной борьбы между притязаниями пап на то чтобы быть верховными светскими правителями Европы.

И одним из монархов, который отвергал притязания пап, был император Германии Людвиг Баварский. И в этой борьбе Уильям Оккам был на стороне Людвига Баварского. Он вел борьбу с папством по двум линиям: с одной стороны как францисканец против богатств, а с другой стороны как сторонник Людвига Баварского против притязаний пап на светскую власть. Поэтому был вынужден бежать, скрылся в Германии. И как гласит легенда, явился к Людвигу и сказал: <<защищай меня мечем, а я тебя буду защищать пером>>, то есть идейно обосновывать борьбу против папского владычества.
Философское учение

У него была своеобразная социальная концепция, согласно которой папа не должен обладать полнотой власти над католиками. Верховная власть – это сама община верующих, которая избирает собор. Так что община и собор, как представители этой общины, и есть истинная верховная власть, что же касается власти пап, то это временная власть, и она должна быть ограниченной. А самое главное, необходимо четкое разделение духовной и светской власти. Вполне понятно, что это вызвало недовольство и преследование.

В его работах так же возникает проблема теологии и философии. В отличие от Скота у Оккама была довольно категоричная точка зрения: теология и философия – это две области духа, которые не должны соприкасаться. Он считал, что никакими доводами разума воспринять мир нельзя и не нужно: вера настолько хороша, что она в этом абсолютно не нуждается. Более того, когда мы пытаемся познавать истины откровения средствами разумными, мы тем самым оскорбляем самого Бога. Собственно говоря, зачем на нужно доказывать что Бог прав, когда он и так абсолютно бесспорно всегда прав.

    То есть он пропагандировал теорию двойственной истины (истина откровения и истина науки). Теология не должна вмешиваться в дела разума, в дела философии, а с другой стороны наука не должна вмешиваться в дела откровения, потому что там совсем другой источник знания.

Когда он обращается к проблеме общего отдельного, то в отличие от Скота он выступал как номиналист. С его точки зрения в мире нет ничего общего, в мире нет ни каких универсалий. Универсалии существуют только в душе человека. Считал, что существует не только слово, но и понятие, а это понятие (например, кошка) фиксирует сходства общие черты конкретного отдельного предмета (между всеми кошками). Такая точка зрения называется концептуализм или умеренный номинализм. Напомним, что одним из первых умеренных номиналистов был Пьер Абеляр. Так вот, эту линию Пьера Абеляра отстаивал и Уильям Оккам.

Отсюда, так как весь мир есть скопление конкретных отдельных вещей, и кроме этих отдельных ничего нет (не считая Бога, конечно), то вполне понятно с чего нужно начинать, с изучения вещей, с изучения отдельного. Он выделял три формы знания:

    Интуитивное знание – знание, полученное в опыте при взаимодействии человека с внешним объективным миром (с вещами). Это главное знание.
    Абстрактное знание – размышления.
    Знание откровения.

Он был эмпириком и считал опытное знание главным (все должны проверять на опыте). Что же касается откровения – то это другое знание, и оно получается не из опыта общения с миром, а из опыта общения с божеством.

Чем знаменит Оккам? Обычно говорят о <<бритве Оккама>>. Это его мысль о том, что когда мы объясняем какое-то явление одним достаточно полным способом, не стоит придумывать другие способы. Другими словами, <<Простейшее объяснение — наилучшее>>.

Резкое размежевание между философией и теологии говорит о том, что пришла нужда избавиться от теологии и развивать философию как вполне светскую науку. Это признак глубочайшего внутреннего кризиса самой схоластической философии, которая исходила из того, что теология и философия всегда тесно связаны (союзники).
\subsection{Философская и общественно-политическая мысль эпохи возрождения (средина XIV – XVI в.в.)}

Вообще-то говорят не об одном возрождении. Вспомните <<Каролинские возрождение>> (где пробудился интерес к древней мудрости), возрождение XII века (когда пошел поток знаний с арабского востока, но в основном практических знаний) и, наконец, великое возрождение.

Что значит слово <<возрождение>> и почему эпоха называется возрождением? В честь попытки возродить славную погибшую когда-то античную культуру. Людей, которые творили в то время, называли гуманистами. Так вот, с точки зрения гуманистов была величайшая античная культура, потом пришли варвары, и наступила эпоха полного бескультурья, и наступила, как они говорили, <<черная ночь средневековья, когда погасли все звезды, и мир погрузился в трясину варварства>>. Они не замечали, что в эпоху средневековья была своя культура.

С чем была связано появление новой культуры? В чем ее особенность? В центре внимания средневековых людей был Бог и человек должен был думать не о том, как добиться счастья в этом мире, а как обеспечить себе блаженство на том свете. Была пропаганда аскетизма, которая привела к появлению монастырей, умерщвления плоти, заточение плотской любви (считалось, что любовь – это отвратительная вещь, величайший грех перед богом, а Бог разрешает этим заниматься, то только ради того чтобы не погибло человечество). Это сказалось на всем средневековом искусстве.

    А вот что касается новой культуры, то в центре внимания стоял живой человек, который имел не только душу, но и плоть и который хочет быть счастливым и здесь на земле. Поэтому эта новая культура и называлась гуманистической.

C чем был связан это перелом? С определенными сдвигами в средневековом обществе. Как мы уже говорили, что где-то в XI веке стали возникать города, которые в последующем получили развитие. А города были своеобразными центрами, где процветали ремесло и торговля (а это уже не феодальные отношения). Конечно, это были города в эпоху феодализма, но в самих городах феодализма-то не было. И вот эти города были своеобразными островками внутри феодального общества.

Но, конечно, подобная ситуация наблюдалась не везде. В Италии не было централизованного государства, и поэтому там города окрепли, подчинили себе округу и стали государствами. Таким способом были ликвидированы феодальные отношения не только в городе, но и в деревне. Самые известные города государства: Флоренция (родина великого Данте и великого Макиавелли), Генуя, Венеция, Пиза, Сиена – всего было около 20 городов-государств. Они процветали в основном за счет внешней торговли: через них шли торговые пути. Вот тут-то и зародилась новая культура – в начале возрождение было явление чисто итальянским.

Надо сказать, что эпоха Возрождения сыграла огромную роль в развитии западной Европы во всех областях, в том числе и в философии, правда значительных сдвигов в этой области не произошло. Философы в то время были далеко не самостоятельными, и главным было не столько создание новых систем, сколько упор на разрушение старых порядков. То есть эпоха Возрождения была своеобразной эпохой разрушения средневековых авторитетов. Началась методическая работа по разрушению всех оков, которые сковывали человеческую мысль.

Проведем небольшой обзор людей, которые не были философами, но которые били по старым представлениям, создавая тем самым основу для последующего блестящего взлета философской мысли.
\subsubsection{Франческа Петрарка (1304–1374)}

Он был мыслителем, собирателем античной литературы (которую он поручал переписывать с тем, чтобы она стал общедоступной). Один из создателей итальянского литературного языка. До этого времени все трактаты писались на латыни. Продолжая традиции, Петрарка свои трактаты писал на латыни, а вот стихи на итальянском. Кстати, что итальянский язык возник на базе упрощенной (вульгарной) латыни. Так что сходство было, конечно, но, тем не менее, одно дело латинский язык, а другое разговорный итальянский. Был итальянским патриотом, призывал к объединению Италии и созданию единого государства. Воспевал он не только любовь, причем земную любовь, но и красоты природы, что было неприлично в средние века.
\subsubsection{Джованни Боккаччо (1313-1375)}

Крупнейший литератор, прозаик. Прославился своим знаменитым Декамероном. Сборник, содержащий сто новелл, написанных на живом, итальянском языке. Книга была написана где-то в 50-53 годах. Эти годы ознаменованы ужасающим бедствием – чумой, которая свирепствовала, по меньшей мере, три года. Чума охватила всю Европу, причем в некоторых местах население сократилось на половину! Действие Декамерона разворачивается во время эпидемии, когда 3 юноши и 7 девушек укрылись на загородной даче. Выходить за пределы дачи было страшно, потому что легко можно было заразиться. А так как им жить было скучно, то они договорились, что каждый вечер каждый их десяти человек будет рассказывать какую-нить историю. Десять дней по десять новелл. Декамерон – буквально, десятидневник. Содержание Декамерона было таково, что его не решились публиковать. В центре внимания, конечно, жизнь людей, огромное внимание уделяется любви плотской, причем по тем временам это считалось страшно неприличным (конечно, по сравнению с тем, что сейчас печатают, это вполне невинно), резкой критике верхов и духовенства. Главным объектом сатиры был монах, ведущий разгульный образ жизни и мошенничающий.
Лоренцо де Балла (1407-1457)

Когда началась эпоха возрождения, когда были найдены тексты античных авторов и в частности античных историков, стал вопрос об атрибуции, то есть о выяснении не подложности. Как раз этим и занимался де Балла (критикой источников, восстановительной критикой). Более того, он был одним из основателей критики источников. Он считал, что такой критике должны подвергаться не только светская литература, но и священописания. И в частности он выяснил следующее. Дело в том, что Папы римские, когда они претендовали на господство над Римом и его округом обычно ссылались на некую грамоту. Согласно преданию император Константин, при котором христианство стало государственной религией, он же перенес столицу из Рима в Константинополь, якобы написал грамоту, где передал всю власть в Италии Папе. Лоренцо Балла подверг строжайшему анализу текст этой грамоты и выяснил, что она датирована не в четвертом веке нашей эры, а где-то в 12-13 веке. Вполне понятно, что это вызвало возмущения.

Он был и философом, хотя и не самостоятельным. Его заслуга в том, что он попытался возродить эпикуреизм. Вспомнил материализм Эпикура и стал апологетом этого учения. Одним из первых поставил под сомнение авторитет Аристотеля. Надо сказать, что Аристотель был язычником, а уже в последствии был доработан и превращен в верного католика. И его формулы стали догмами, в которых нельзя было сомневаться, поэтому люди, которые пытались поспорить, начинали заниматься схоластикой, что, несомненно, тормозило развитие философской мысли. Аристотель в такой форме стал препятствием на пути развития философского и научного знания.

1455г – изобретено книгопечатанье. Это стало прорывом. За следующие сорок лет типографии возникли в 260-ти городах Европы! Стали возникать личные библиотеки.
\subsubsection{Николай Крэбс (1434-1464)}

Немец по происхождению. Получил высокое звание кардинала. Неоплатоник. Занимался теорией математических игр. Пытался создать свою картину вселенной и объявил, что вселенная бесконечна. А что касается Земли, то она одна из множества небесных тел. В бесконечной вселенной не может быть центра. И поэтому Земля – не центр мироздания. Конечно, Земля, с его точки зрения, шарообразна. А вот говорить о том, что Земля вращается вокруг Солнца, говорить не решился (или просто не знал).
\subsubsection{Никола Макиавелли (1469-1527)}

Он был крупнейшим историком. Стала возрождаться историческая наука и, прежде всего историки эпохи возрождения выступили против провиденционализма (учения, согласно которому ход истории определен богом). Конечно, они не были атеистами, но считали, что историю создают только естественные силы. Перечислим некоторых крупнейших представителей историков эпохи Возрождения: Леонардо Бруни (1370-1444), Флавио Бьондо (1392-1463). Надо сказать, что в трудах Бруни и Бьондо зародилась как раз та самая периодизация истории, которой мы с вами сейчас пользуемся (Античность Средневековье и Возрождение). В трудах Николо Макиавелли эта периодизация тоже отстаивается.

Он был не только историком, но и социологом (хотя назвать его так нельзя, так как науки социологии то еще не было). Он был, по сути, социальным мыслителем. Тоже поролся против приведенционализма. Огромное значение он предавал действиям людей. Считал, что история управляется волей и разумом людей, особенно выдающихся. Но он не мог не видеть, что есть в истории то, что от людей не зависит, и вводил понятие судьбы, фортуны. Действия людей на половину определяются судьбой, а на половину сознанием. Конечно, подобной точки зрения придерживаются и другие историки, но Макиавелли пошел глубже: он написал книгу, которая называется <<История Флоренции>>. В Италии было много обществ, много государств, которые враждовали между собой. По сути дела это были независимые государства. Работу же Макиавелли из других подобных выделяло, то, что он рассмотрел внутреннюю борьбу, которая шла во Флоренции. И он показал, что эта борьба между разными группировками внутри Флоренции имела в своей основе разные интересы, корни которых, в конечном счете, уходили к имущественным отношениям. В конце этой борьбы была ликвидирована республиканская форма правления, и установился во Флоренции режим тирании.

Макиавелли был патриотом Италии. На Итальянские республики то и дело нападали соседи, то немцы, то французы и поэтому он мечтал о едином и могучем Итальянском государстве. Он понимал, что для объединения Италии нужен деспот, правитель который пойдет на все, и только таким путем можно добиться объединения. Он был сторонником республики, но, видя, что республиканская форма правления не подходит для Италии, он стал выдвигать идею могучего правителя, который вооруженной рукой объединит Италию. И для великой цели он советовал ничем не брезговать. Ради великой цели правителям можно все (освобождение политики от морали).

Кстати одно время он был крупным государственным деятелем Флоренции (государственным секретарем совета Флоренции), а потом был изгнан и умер в изгнании.

Ну а теперь, когда мы немного затронули итальянцев, мы поговорим о том, как волны возрождения стали охватывать одно государство западной Европы за другим. В самой Италии, кстати, к концу XV века оно стало угасать (она стала на путь рефеодализации).
Эразм Ротердамский (1467-1536)

Писал он на латыни, был космополитом (жил в Италии, в Швеции, в Голландии, в Германии, в Англии, во Франции), ученик Лоренцо Балла. Самая знаменитая работа: <<Похвальное слово глупости>> (1509г.) – при жизни Эразма выдержала 40 изданий. Там выступает величайшая сила истории – глупость и эта глупость поет себе похвалу. Дальше он сатирически изображает разнообразные мнения современной жизни, высмеивается глупых правителей. Он говорит, что если бы не было глупости то, конечно же, все священники умер ли бы с голоду. Именно глупость обеспечивает их высоким жизненным уровнем. Высмеивает он и теологию, называя ее <<смрадным болотом>>. Кроме дураков никто теологией не занимается. Обрушивается не только на теологию, но и на схоластику с ее философией.
Томас Мор (1478-1535)

Выходец и зажиточной буржуазной семьи. Не имея дворянского знания, сумел достигнуть высоких постов. Последней его должностью была лорд-канцлер (премьер-министр) Англии. Прославился своей книгой <<Утопия>>. (Полное название: <<Золотая книга столь же полезная, как и забавная о наилучшем устройстве государства и о новом острове Утопия>> - 1516г.) В этой книге автор подвергает жестоко критике общественный строй Англии и других государств Европы. Делит современное общество на две группы людей, из которых одна пользуется роскошью, а другая живет в нищете. Причиной же, главным злом является частная собственность, которая порождает такое вопиющее неравенство. Он считает такое общество несправедливым, не отвечающим природе человека. Допускает наличие другого общественного стоя, но не призывает к уничтожению предыдущего. Картина Англии этого времени очень хорошо описана в книге <<Принц и Нищий>>.

То время можно назвать эпохой великих географических открытий. Открывались все новые земли, и вот Мор рисует, что один из капитанов английского флота наткнулся на остров Утопия, где были совсем иные порядки, где не было частной собственности, где люди все были равны, где трудились все до единого и получали по заслугам. То есть рисует социалистическое государство, управляемое народом, где господствуют демократические порядки. Конечно, это общество было основано на ремесленном труде, отсюда и довольно суровые условия жизни. Что же касается преступников, то их превращали в рабов!

Как погиб Томас Мор? Генрих Восьмой решил развестись с одно женой и завести другую, причем он женился целых восемь раз. Так как Папа Римский отказал ему в разводе, то это было толчком к тому, что в Англии началась …. реформация. Давно уже короли поглядывали на богатство монастырей с тем, чтобы его присвоить. Он объявил себя главой английской церкви, отказался подчиняться Папе Римскому и конфисковал имущество всех монастырей. И в таких условиях Томас Мор выступил против развода, в конечном итоге его приговорили к смертной казни. И поэтому его в дальнейшем стали считать мучеником, потому что он выступил против реформации, за католическую церковь.
