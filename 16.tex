

\section{Иррационализм и позитивизм. <<Аналитическая философия>>}
Лекции по философии науки проф. Семенова Юрия Ивановича
Лекция 16, семестр 9, 17.12.2005
\subsection{Иррационализм.}

Мистика – учение о том, что для того, чтобы познать мир не нужно заниматься исследованиями, нужно ждать пока тебя озарит. Возможно, каким-либо сверхестественным способом. Мистика в какой-то степени предшественница иррационализма. В него входит мистика, но они различны. Мистика – часть религиозной философии. Однако, иррационализм – светская философия. Современная версия рационализма – экзистенциализм. Представители - Жан Поль Сартр, Альберт Камю. Иррационализм зарождается в начале 19го века. У позднего Шеллинга имеются его элементы.

\subsection{Шопенгауэр}

Иррационалисты пытаются давить на эмоции и на пророчество. Типичный пример Ницше.

\paragraph{Артур Шопенгауэр.} 1788-1860 1817 - Мир как воля и представления. Работа прошла незамеченной, но в 60 годы она стала заметна, и из-за нее разгорелся большой интерес.
\paragraph{Основные положения.}

Духовное начало. Если Гегель клал разумное начало, то Шопенгауэр клал волю, волю которая носит иррациональный характер.

Он пытается показать волю. Вначале воля развивается бессознательно, и действуют механические законы. Возникает сознание, как нечто вспомогательное. Возникает представление. И это представление раздваивается на объект и субъект. Те Я и мир, который существует для меня. И мир от меня не зависит. Здесь очень похоже уже на Канта.

    В этом мире действует закон достаточного основания. Он представлен в четырёх ипостасях: 1. Бытие (для времени и пространства) 2. Причинность 3. Логическое обоснование 4. Закон мотивации.

Есть 2 мира - мир для меня, мир Я и мир для воли. Отличие от Канта мир в себе - мир воли и он познаваем.
Учение о познании

    И тот другой миры познаваемы: 1. Познание интуитивное. Когда не надо думать и рассуждать 2. Познание рефлексивное

Рефлексивное познание носит практический характер и раскрывает отношения между вещами. Носит прагматический и корыстный характер.

Интуитивное познание ради наслаждения познанием и результатом. Именно при помощи интуитивного сознания можно поникнуть в мир идей. (Почти платонизм). Высшей формой является художественное познание, но к этому познанию расположены не все. Только гении могут проникнуть туда. Обо всех философах до Шопенгауэра говорит с пренебрежением.
Этика

Много сходного с Кантом. Никакой свободы нет. А свобода в том мире и воля свободна. Мир - это мир скуки и страдания. Откуда проистекли страдания. Людям всё время чего-то хочется. И жрать и роскошь и баб. Нужно ограничить себя.

Почти все заимствовал из буддизма. Пропаганда аскетизма. Чем меньше стремления - тем меньше страданий. Как умрёшь - так будет хорошо. Веры в спасителя нет. Тело разрушится и разрушится мир и страдания.

Конечно все это квазифилосовия. Как раз последняя философия - это отклик определенных классов. Это восстановление веры после века рационализма.

\subsection{Ницше. 1844-1899}

Науки там никакой нет. Пророчил и призывал. В Европе разразился духовный кризис, и все ценности отброшены и исчезли. Все рушится. И он прав. Задача - показать пути выхода. Нисколько не интересует природа.

    Главное понятие - понятие жизни. Жизнь - а непонятно. 3 разных смысла: 1. То, что жрёт. 2. Общественная жизнь 3. Переживания.

Главная задача - переживать. Жизнь это не дух и не материя. Это то-то 3е. Основа жизни - воля. Воля к власти. Все кипит, и нет никакого порядка. Сила возникает из ничего и уходит в никуда. Познание как способ эволюции. Нужно познание и воля власти. А наука стремиться найти устойчивое - интегралы движения. Но так как устойчивого ничего нет то все это фикция. Истина есть, но это то что полезно. Чистая мифология. Наука - сплошная ложь, но это полезно! И помогают нам выжить. Религия утешает и тем самым дает надежду, лживую, но надежду. Это согревает душу. Если вся наука мифы - то как быть с философией Ницше - это такой же миф!

Концепция вечного движения - упадок, величие и так вечно. И как раз мир на нисходящей.

Надо предложить какой-то путь. А где же причина. Причина в христианстве! Оно учит, что все люди равны и братья. И эта пропаганда равенства заразили низы. Так как низы вдруг захотели равенства, а равенства не может быть. Делятся на 2 расы: Господа и быдло. Христианство это мораль рабов. Надо отбросить всякую мораль. Для господ мораль не нужна. Они могут делать всё что угодно. Для белокурой бестии нет никаких преград. Рабов в конюшни! Эти люди имеют право на все. Война отчищает тело и нации. А потом будут мировые войны, где погибнут миллионы. Ницше первый философ империализма. Его идеи были использованы Гитлером потом. Но у него нет различия по национальностям.
\subsection{Позитивизм}

Нужна была компромиссная философия. Из потребностей капитализма возник позитивизм.

Основоположником позитивизма был Огюст Конт Был секретарём у Сен-Симона. А потом он вступил на путь самостоятельный: Курс позитивной философии. В 60ых годах появилось целое направление.

    Концепция 3ех фаз человеческого познания. 1. Мышление теологическое Человек и тогда пытался объяснить, что происходит в мире. Люди объясняли действием сверхъестественных сил. 2.Метафизическая фаза. Люди стали умнее. Начали отказываться религии. Стали объяснять, что лежит какая-то субстанция. 3.Наконец наступает позитивное мышление. Мир - чувственный мир. А мышление это осмысление чувственных данных.

Мы знаем мир, каким он является в нашем сознании. Это агностицизм. Обыкновенный Юмизм. Не отрывайтесь от реальности. А что за пределами сознания - нелепый вопрос.

    Трезвый подход к миру. Наука должна эти явления сортировать, по признаку подобия. Мышление - систематизация. И причинность. И это все результат наблюдений и мы никогда не докажем что так будет всегда. Наука должна навести порядок. Вот эти регулярности фиксирует наука. Всё должно быть проверяемо.

Философия должна заниматься наведением порядка во всей науки. Всё должно свести к чувствасвам. 3 эпохи и также и эпохи развития общества. Все науки делит по сложности, от простого к сложному: 1. Математика 2. Астрономия 3. Физика 4. Химия 5. Биология 6. Социология

Нужна опытная социология. Попытался обосновать, как существует каждое общество. Социологический организм. Мы должны изучать динамику социологических организмов. Тут люди как клетки в биологическом организме.

Конт это эклектика. Рассуждает как материалист. Конт это первый позитивизм 30-80 годы 19 века.

Джон стюард Милль. 1806-1873. Написал грандиозную книгу. <<Система логики. Спиллагистическая и индуктивной логики>>. Развивает индуктивную логику. Все приёмы Бекона у него отмечены четко. Главное индукция. Все индуктивная логика - вероятностный характер.

Герберт Спенсер 1820-1903 С одной стороны, что мы не выйдем за пределы чувств. Кто прав наука или религия сказать не возможно.

Где было господство религии, позитивизм сыграл позитивную роль. А в Британии, Франции и других развитых странах она только мешала.

\subsection{Второй позитивизм конец XIX- начало XX}

\paragraph{Рихард Авенариус. 1843-1896}

\paragraph{Эрнст Мах 1838-1916}

Оба начинали типа Беркли. Чистый субъективный идеализм. Потом начисли сдвигаться в сторону феноменализма. Объективный мир - мир ощущений.

Вводит понятия мир. Все вещи это элементы мира и ощущения. Это не духовное и не материальное - это третье. Оно поднялось выше материализма и идеализма.

    Философия новейшего естествознания - эмпириокритицизм. (Махизм)

Пришлось Ленину написать креатив, что большевики не махисты. В 1909 году. Там он обличает махизм и то, что он спекулирует на естествознании. Электрические заряды - материальны.

\subsection{Аналитическая философия}

1. Логический позитивизм. Неопозитивизм или Третий позитивизм. 2.Лингвистпическая философия

Изучение человеческого познания. Весь позитивизм основывается на чувственных данных. Восприятия у разных субъектов не возможно даже сопоставить. Мысли же не измеришь и не потрогаешь, но наше счастье, что знание выражается в суждении. Слова обозначают вещи.

Надо заняться словами и предложениями. Философия что бы изучит познание надо заняться языком. И неопозитивизм - научный язык, а лингвист позит - обыденным языком. Проблема научного знания. Исходным знанием является мир в наших ощущениях. А за ними есть что-то? Знать мы не можем. Вопрос глупейший, неприлично философу заниматься этим. Суждения это есть высказывание, и они могут быть истинными или ложными. Есть ощущения, и они обязательно выражаются в предложениях. Вот это и надо изучать. Анализ языка науки.

    2 проблемы 1. Структура научного знания. 2. Демаркация научного знания

А что там бальше, мы будем подробно изучать в следующем году.
