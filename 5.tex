\section{Наука в эпоху Возрождения}

Первая часть этой лекции безвозвратно утеряна
\subsection{Протестантство.}

…Восстания имели территориальный характер, быстро подавлялись. Теперь же началось массовое движение которое охватило в начале Германию, потом перекинулось в Скандинавию, Швецию, Францию, Шотландию и т.п. Движение протестантов было не атеистическим, его целью было реформирование христианской этики, реформирование христианской церкви, церкви католической. Протестантство – новое, третье, направление христианства. Главным действующим лицом протестантов был Мартин Лютер (1483-1546). Его тезисы обличали церковь. Он выдвигал идеи реформирования христианской религии. Протестантство имело множество направлений и одно из них было названо в честь Мартина Лютера – лютеранство. Лютеранство также не было единым, оно было демократическим и не имело единого центра. Лютер выступал против схоластики и особенно обличал Аристотеля, которого называл безбожным оплотом баптизма. Другой крупный деятель Жан Каллин. Он был родоначальником ветви протестанства, которое называлось каллинизм. Особенностью этого направления была абсолютная предопределенность, божья воля предопределяет жизнь каждого человека. Он продолжал линию Августина Блаженного: никто не знает какова будет судьба человека, поэтому он стремиться служить Христу. Эти люди выступали против католицизма, однако, в тоже время по приказу Жана Каллина в Женеве сожгли знаменитого ученого. Так что была инквизиция не только католическая, но и протестантская. Каллинисты были умеренными в том смысле, что они не выступали против феодальных порядков и не участвовали в массовых восстаниях. В те же времена (1525г.) Германию охватило великое крестьянское восстание. Одним их самых активных сторонников этого восстания был Томас Мюнцер (1490-1529) он принимал самое активное участие в восстании и погиб в бою. Он сам не был атеистом, но был близок к пантеизму (нет личного бога, бог слит с природой, Христос – человек, чудес нет, ). Томас пропагандировал наступление стадии царствования божьего на земле. Считал, не нужно ждать, а нужно создавать это царствование на земле самим. Он не описал будущее общество в подробностях, но ясно одно: это общество без частной собственности, без эксплуатации и без государства. Самое главное отличие от Томаса Мора, это то, что Томас Мор просто описывал идеально общество, а Томас Мюнцер призывал к борьбе за это общество. Это была программа христианских выступлений. Т.е. нужно не ждать новых порядков, а нужно бороться за них.
\subsection{Деятели Возрождения во Франции}
\subsubsection{Франсуа Рабле (1494-1553)}

Мыслитель, знаменитый писатель. Роман “Гаргантюа и Партангрюэль” вошел навсегда в мировую литературу. Был монахм, но порвал с монашеством, его преследовали. Роман – борьба с схоластикой. Главное оружие против средневековых мировоззрений, против схоластики – юмор. В своей книге высмеивал существующие порядки. Философом он не был. Рисовал картину общества справедливого, но довольно ограничено.
\subsubsection{Пьер де ля Раме (1515-1572)}

Выходец из семьи плотника, поэтому с трудом поступил в университет, проявил огромные способности, защитил диссертацию на звание бакалавра, он обрушился на схоластику, предмет критики – Аристотель. Он присоединился к движению гугенотов и был убит в Варфоломеевскую ночь. Достаточно назвать название тему его диссертации, которую он защитил когда ему было 20 лет “Все что сказано Аристотелем - ложно”. Он сказал, что метод, предложенный Аристотелем не годиться для науки, а годиться только для схоластов. Он один из первых поставил задачу создать новый научный метод. Эту задачу он не решил, т.к. не наступило еще для этого время. Таким образом, он разрушал его сила не в созидании новых теорий, а в разрушении старых. Затем все же он нашел в трудах Аристотеля рациональное и немного пересмотрел свои взгляды.
\subsubsection{Мишель де Монтень (1533- 1592)}

Его отец был богатым буржуа, поэтому ему не приходилось зарабатывать на жизнь. Принимал участие в политической жизни, но потом отошел от всего этого. Это было время религиозных войн. И он поселился в замке вдалеке от мирской суеты. Самое знаменитое произведение Монтеня: “Опыты” (1580). Это огромная книга, 2 или 3 тома в различных изданиях. Не единая книга, а сборник набросков на разные темы. Главный мотив – мотив сомнения. Во всем надо сомневаться, ничего не принимать на веру. Его мировоззрение – скептицизм. Причем скептицизм довольно глубокий. Считал, что даже чувства обманывают человека, мышление тоже слабо и подвержено ошибкам. Главная идея – постановка под сомнения всего что существует. Подвергает сомнению не только средневековую, но и античную культуру. В этом он пошел дальше гуманистов возрождения. Нужно создавать новую мудрость и нельзя ограничиваться пошлыми знаниями. Называет теологию лженаукой. Восстанавливал идеи апикурейцев: погрузиться в себя, счастье – быть подальше от бурного потока, главная радость – познание, это и обеспечивает счастье.
Философско-историческая концепция Жана Бодена.

Жан Боден (1530-1596) Он является основателем науки государственного права. Впервые разработал проблему государственного суверенитета. Это он отразил в работе:”Шесть книг о государстве”. Он сделал крупный вклад в философию истории. Выделял стадии развития человеческого общества. Причем не конкретных обществ, а всего общества в целом. Ввел понятие исторической эстафеты, когда культура и знания передается от одних народов другим. Тем самым, народы исчезают, но то что ими накоплено сохраняется и передается. Он выделяет три великих эпохи истории. Три эпохи: первая – эпоха народов востока, народов жаркого пояса. Именно там возникло общество более высокого типа, цивилизованное, высокой культуры: Персия, Сирия и т.д.. Они внесли огромный вклад в развитие человечества. А потом они ушли в тень и на их смену пришли народы умеренного пояса, обитающие около средиземного моря. Т.е. вторая эпоха – эпоха античной культуры. Античная культура гибнет. Третий этап - народы севера, западная Европа. Т.е. идет поступательно развитие всего человечества в целом, не взирая на гибель отдельных народов и культур. По его мнению никогда не было золотого века. Объясняет более бурное развитие общества по сравнению с прошлым: Изобретен компас, что позволило свободно передвигаться по морям, а античные народы плавали вокруг берегов. И начались процессы интернационализации. Человечество становиться единым народом. Люди теперь не сомневаются, что земля - шар, открыт новый свет. Открыто книгопечатанье. В результате образование стало доступным. Изобретено огнестрельное оружие. Мощная промышленность. Т.е. нет сомнения, что имеет место прогресс и сейчас он намного выше. И мы движемся гораздо быстрее. Он разрабатывает идею географического детерминизма: климат определяет структуру организма, психику взгляды, а также развитие общества в целом. Обычно, разработка теории географического детерминизма приписывается другим философам, т.к. Жан Боден писал на латыни, а в то время когда эта проблема рассматривалась, латинский язык уже вышел из моды. Нужно критиковать не только средневековую но и античную культуру и нужно создавать новые знания. И это процесс становления науки назревал…
\subsection{Научная революция. Возникновение современного естествознания. Наука в эпоху Возрождения}

Была ли наука в средние века? И да и нет. Были накоплены знания, но никакой системы, конечно, не было. Не было никакой научной картины мира. Если и была, то она была философская. В основе средневековой философской картины мира была философия Аристотеля. Земля – центр мироздания, вокруг некие сферы звезд и т.д. Такая наука подходит для католицизма. Пример средневековой науки – алхимия. Они всего лишь проводили многочисленные опыты. Многое осталось из античности: Архимед, геометрия Евклида, физика Аристотеля. Наука начала возникать только в XVII веке. Можно отметить людей, которые внесли вклад в развитие науки и даже опередили свою эпоху:

Леонардо да Винчи (1452-1512) Был всесторонне развитым человеком. Был художником (Мона Лиза) и ученым. Больше всего ценил опыт. Был великим изобретателем (парашют, вертолет). Он занимался и математикой, механикой. К сожелению многие его изобретения остались на бумаге, т.к. не было еще технических средств для реализации его идей.

Коперник (1473-1543) Поляк по происхождению. Учился за пределами Польши. Сначала учился в университете Вене, затем в Италии Болония, в 1543 году решился обнародовать главный труд: “Об обращении небесных сфер”. Решился выпустить ее только перед своей смертью. Книга писалась с оговорками, т.е. это просто хорошая модель для упрощения математических расчетов. Опроверг идею о том, что Земля – центр мироздания, а лишь одна из множества планет. Это шло вразрез с идеями о земном и небесном мире, о боге архангелах и т.п. Рухнула вся устоявшаяся средневековая система.

Джордано Бруно (1548-1600) Бруно был материалистом. Считал, что природа и бог – одно и тоже. Придерживался концепции панпсихизма. Был сожжен на костре по приказу инквизиции. Он никогда не занимался экспериментальной наукой. Он был чистым философом. Говоря о Копернике нельзя вслед за ним упоминать Бруно, потому что Бруно начал философски толковать работы Коперника. Он пришел к выводу, что каждая звезда – это солнце, а солнце – всего лишь одна из множества звезд, планета – ничтожная пылинка. У него возникла идея множества обитаемых миров. Так получается, что на каждую планету Бог посылал по сыну!?? Тут то священники почувствовали, что учение Коперника – это страшная вещь, спохватились и в 1613 году запретили ее и приказали сжечь. Бруно угодил в ловушку инквизиции. Его предупредили, что если он покаится, то его простят. Он все равно настаивал на своем, зная чем это грозит. Был сожжен в Риме на костре.
\subsection{Великие ученые XVII века}

Галилео Галилей (1564-1642) Выступал простив схоластики и теологии, но не всегда последовательно. Открыл ускорение свободного падения. Доказал его экспериментально, лазил на башни и проверял его. Разрабатывал механику. Подготовил появление законов Ньютона, но не сформулировал их. Изучал движения маятника, что позволило усовершенствовать часы. Изобрел термометр. XVII век – век приборов: микроскоп, телескоп-рефрактор, телескоп – рефлектор, монометр, водный и ртутный барометр. Использовал телескоп, и сразу сделал массу открытий. Разбил миф о структуре поверхности планет, что горы есть и на других планетах, а не только на земле. Открыл спутники Сатурна, что вызвало всеобщее возмущение, ведь есть только 7 небесных тел. Исследовал движение пушечных ядер. Его приглашали для разработки пушек и усовершенствования полета снаряда. Открыл фазы Венеры. Отстаивал бескомпромиссно систему Коперника. В 1632 вышла книга “Диалог о двух важнейших системах мира” на итальянском языке, где спорят сторонник системы Коперника и сторонник системы Пталомея. Каждый приводил доводы в пользу своей теории, но в конце книге полную победу одерживает сторонник Коперника. В отличие от скучной и сложной книги Коперника со множеством цифр, формул и таблиц, это произведение было доступно широкому кругу читателей. В ней ярко была описана вся несостоятельность системы Пталомея. Ну а так как это произошло, когда книга Коперника была уже запрещена, книга привлекла внимание, и Галилей попал в тюрьму, подвергался пыткам. Ему предложили либо смерть либо отречение от своих идей. Галилео, в отличии от Бруно, решил отречься от своей концепции. Сейчас спорят, как он поступил правильно или неправильно. Ясно одно, что он внес большой вклад в науку.

Иоганн Кеплер (1571-1632) Немец, он открыл, что планеты движутся по эллипсам, и тем самым обеспечил полную победу системы Коперника. Теперь она давала совершенно точные расчеты движения небесных тел. Открыл свои три закона. Был астрономом, занимался и астрологическими прогнозами. Чтобы выжила астрономия приходилось заниматься и астрологией.

Уильям Гильберт Занимался исследованиями явления магнетизма и электрества. Но этим тогда мало кто интересовался.

Уильям Гордей (1578-1657) Основоположник научной теологии. Открыл кровообращение.

Эванжелисто Торричелли (1608-1686)
Ученик и секретарь Галилея. Основа механики жидкости и газов, изобрел ртутный термометр, впервые получил вакуум.

Отто Фон Герике (1602-1686) Ученый и изобретатель. Изобрел воздушный насос. И водный барометр.

Роберт Бойль (1621-1691) Положил основу научной химии. Закон Бойля-Мариотта

Эдм Марриот (1620-1684) Занимался газами. Основатель Парижской академии наук.

Антоний Ван Ливенгук (1632-1723) Ученый из Нидерландов. Биолог-любидель. Открыл микроорганизмы

Христиан Гюйгенс (1629-1695) Нидерландский ученый. Создал часы новой конструкции – сверхточные часы с маятником, открыл кольца Сатурна, открыл самый крупный спутник Сатурна Титан, изобрел монометр, а главное – создал волновую теорию света и объяснил явления дифракции и интерференции, был крупнейшим и одним из первых теоретиков.

Исаак Ньютон (1642-1727) Открыл закон всемирного закона тяготения, три закона Ньютона, завершил теорию механики твердого тела, и она оставалась таковой до начала 20-го века. Современник Ньютона написал небольшую эпиграмму: “Был мраком мир окутан, сказал господь: “Да будет свет”, и вот явился Ньютон”. Современник Эйнштейна дополнил эту эпиграмму: “Но Сатана не долго ждал реванша, пришел Эйнштейн, и стало все как раньше” Ньютон не был атеистом, даже на старости лет писал книги посвященные богословию, но они были благополучно забыты. Он допускал, что Бог сотворил солнечную систему, затем толкнул ее и уже больше в эти дела не лез, затем движение тел стало подчиняться законам механики. Но в целом никаких чудес нет.
Возникновение научных учреждений и научной печати

Стали появляться один ученый за другим, и стали появляться научные учреждения. Первое научное общество возникло в 1662 году – знаменитое Королевское общество Англии, английская академия наук, создана при короле Карле II. Оно возникло не на пустом месте. Ранее было собрание, ученые собирались еще и раньше (с 1645 года), чтобы обменяться мнениями, сообщить о новых достижениях, поспорить, подумать. Затем это собрание было официально переведено Карлом II в ранг академии наук. Затем в 1666 возникла Парижская коллегия по настоянию Марриота. 1700 – берлинская коллегия. Вместе с ними стали возникать и научные журналы, т.е. возникла специальная научная печать. Первоначально журнал был просто отчетом Королевского английского общества. С 1765 года журнал начал издаваться и во Франции. Так шаг за шагом стали развиваться и другие журналы и научная печать в общем. Возникает первая астрономическая обсерватория в Лейбнице, затем в Париже и в Гринвиче. Таким образом, возникла система научных знаний и отпала нужда в натурофилософии. Возникла научная картина мира. Изменилось мировоззрение людей. Среди ученый воцарилось понятие, что все имеет причину и следствие, что нет беспричинных событий. Возникло научное сообщество. Необходимо искать только естественные причины. Мы уже говорили, что еще в античном мире возникла идея абсолютного детерминизма и была оформлена в трудах Демокрита. Принцип простой: все имеет причину, в мире все абсолютно определено. Но теперь, когда стала известна механика, это было научно подтверждено. Случайность – результат нашего незнания. Это оказало огромное влияние на философию. Но принцип абсолютного детерминизма в дальнейшем был подорван появлением квантовой механики