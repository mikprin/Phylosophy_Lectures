\section{Философия Средних Веков.}


\subsection{Философская мысль в раннем средневековье. Возникновение схоластики (V-XII)}

На предыдущей лекции мы уже затронули эту тему, обсудив период от V-го до XII-го веков, и остановились на рассказе о таком крупнейшем схоласте XII-го века, как Пьер Абеляр. Все философы о которых шла речь в основном занимались проблемами онтологии и гносеологии, к социальным проблемам они почти не обращались. Но и в те времена были мыслители, занимавшиеся социально-философской тематикой.
Иохим Флорский (1130-1202).

Самым видным из них был современник Абеляра Иохим Флорский, Калабрийский (годы жизни 1130-1202). (Жил в провинции Калабрия в южной Италии, был настоятелем монастыря Фиори). Был богословом,писал свои труда на латыни. В прошлом году мы говорили об Августине Аврелии, который попытался дать общую схему развития мировой истории, развития человечества. В его схеме история развивается линейно, пожалуй он первый создал унитарную концепцию человеческой истории. Он понимал историю, как единый процесс, где история отдельных людей, стран это частички мировой истории, развивающейся по прямой линии. У него былы разделения на временные отрезки, но стадий не было. Следущий шаг сделал один из первых философов-схоластов, о котором мы уже упоминали, это Иоанн Скотт Эриугена. И Августин Аврелий и Иоанн Скотт Эриугена были сторонниками концепции провиденциализма, согласно которой вся история есть реализация намеченнного богом плана. Но Эриугена считал, что развитие истории - это три периода, связанные с тремя ликами божества(Бог-Отец, Бог-Сын, Бог-Святой дух).

Иохим Флорский дал развитие этой идее. Согласно его трудам, в истории человечества существет три поступательных этапа.

    Первый этап - эпоха Бога-Отца. Люди почти не думали о духовных потребностях, а стремились удовлетворить свои плотские потребности. Бог координировал деятельность людей путём страха, люди жили под игом Бога-Отца, неспособные сами отвлечься от своих плотских потребностей. Второй этап - эпоха Бога-сына. В людях стали просыпаться требования не только плоти, но и духа. Людьми управляло чувство сыновьего послушания, люди повиновались Богу не сколько из-за страха, сколько из-за уважения к нему. С зарождением монашества, зарождается и Третий этап, эпоха Бога-Духа*. Плотские потребности отходят на второй план. Люди свободны и руководствуются потребностями духа, и потому сами способны обеспечить порядок, отпадает необходимость в государстве, церкви.

Сам Иохим Флорский происходил из бедной крестьянской семьи и во многом разделял взгляды крестьян, боровшихся с непосильным трудом и нищетой, и их мечты о лучшем обществе. И эта схема вселяла в людей надежду. Сам он не был мятежником, но его идеи стали потом знамёнами крестьянских восстаний воимя установления нового времени. Поэтому труды его были осуждены Церковным Собором, с критикой выступил позднее такой авторитет как Фома Аквинский, так что в общем-то это была еретическая концепция. Он выразил идею унитарного развития, имеющего стадии(унитарно-стадиального), то есть наличествует идея прогресса.

На смену ранней схоластике приходит поздняя(13 - первая половина 14вв). Изменения в европейской философской мысли, выраженные в поздней схоластике, во многом обусловленны огромным влиянием Ближнего Востока, арабской философии.
Арабская средневековая философия.
Зарождение ислама. Образование арабской имерии

Как и схоластика, Арабская философия была религиозной и основана на религии, называемой Исламом, или Мусульманством. Ислам зародился на Аравийском полуострове, её основоположником был Мухаммед(Магомет) (580(прибл.)-632г.).

Ещё до распространения ислама на аравийском полуострове существовало два типа обществ. Одно из них – кочевые племена, жившие на стадии первобытно общинного строя (предклассвого существования), но были и крупные города, где были торгово-промышленные горда, где уже было классовое общество: Мекка, Якрит… Надо сказать, что в течении III, VI и V веков нашей эры через территорию Аравийского полуострова проходили оживленные торговые пути. Это способствовало обогащению верхушки кочевых племён, так как племена либо грабили, либо охраняли караваны. Но эти пути стали изменяться, идти не через Аравийский полуостров. В результате это сказалось на кочевниках и на городах. И единственным выходом стало завоевание соседних народов, захват соседних территорий.

Племена были разрозненными и без конца враждовали между собой. У каждого племени была своя религия, свои боги. Для их объединения нужно было единое знамя, единая идея. Но в то время такой идеей могла быть только религиозная идея. Притом она должна быть монотеистической - идея единого всеарабского Бога. Кстати, это было ещё до Мухаммеда - течение называлась ханифизм, но оно не стало всеарабским. Мухаммед был его сторонником, но со временем он начал разрабатывать свою религию. Однако её пропаганда встретила неприятие властей, и в 622м году Муххамед бежал из Мекки. Позднее он и его сторонники, мугаджиры, объединили весь аравийский полуостров под зелёным знаменем Пророка. Год бегства из Мекки, Хиджра, стал началом мусульманской эры. По зелёным знаменем пророка было создано единое государство в масштабах аравийского полуострова. А Мухаммед стал его правителем, и оставался им до своей смерти.

Ислам - религия строго монотеистическая. Коран, священная книга ислама, возникла из выступлений, пророчеств Мухаммеда, записаных свидетелями после его смерти. > Основные его положения ислама:

    Нет бога кроме Аллаха; Мухаммед - Пророк его. Каждый мусульманин должен молиться пять раз в день, соблюдая определённые ритуалы при этом. Каждый мусульманин должен платить налог в пользу бедных, так называемый закят. Кроме того нужно соблюдать пост в течение десятого месяца, т.н. рамазана, не есть до заката солнца. * Истинный мусульманин должен совершить паломничество (хаджж) в Мекку. Помимо прочего, провозглашался джихад - священная война против "неверных".

Нужно насаждать ислам, ибо в нём спасение, а кто не хочет его принять, должен быть уничтожен. Однако завоёванные народы, не хотевшие принимать ислам не всегда уничтожались.

Ещё при жизни Мухаммеда были завоёваны некоторые соседние с Аравией государства. После его смерти, халифы, правители правоверных после Пророка, продолжили его дело. Первым халифом был избран Абу-Бакр, один из ближайших сподвижников Мухаммеда, при котором Коран был собран в единую книгу. За ним последовал халиф Омар, третьим (после убийства Омара) стал Осман. Три правденых халифа, при которфых были завоёваны значительная часть Малой Азии, Египет, Ирак,пал Иран, войска арабов дошли до Инда, позднее ислам с индийскими и арабскими купцами распространился на территории современной Индонезии. С другой стороны, арабы завоевали всё средиземноморское побережье Африки, и в 711-м году арабские войска переплыли Гибралтар и завоевали территорию современной Испании, которая стала провинцией халифата Аль-Андалус. Надо сказать, что Испания была завоёвана практически полностью. Арабы хотели завоевать и всю западную Европу, но в VIII веке войска были разбиты и арабы вынуждены были отступить и остаться в Испании до 1492 года, когда было уничтожено последнее исламское государство на территории Испании.

С эпохой завоеваний начался не упадок, а рассвет арабской науки. Возникло великое арабское государство. Арабы были отличными мореплавателями, они смогли распространить ислам от Индонезии до Эфиопии. Не все арабские философы были не арабами по национальности, просто они жили в исламском мире и писали на арабском языке.

Ислам возник как учение единое, но в дальнейшем в исламе произошел раскол на шиитов и суннитов. Он произошел после смерти Османа третьего праведного халифа. Дело в том, что у Мухаммеда был двоюродный брат Али, кроме того, он был женат на дочери Пророка, значит его дети были внуками самого пророка. Когда умер Мухаммед, произошел своеобразный политический раскол, люди окружавшие Али настаивали, что бы Али стал халифом. Но большинство высказывалось, что бы им был Абу-Бакр. И только после смерти Османа Али стал халифом, и справедливость, по его мнению, восторжествовала. Но это длилось не долго, он был предательски убит. Потом был убит его сын Хусейн. И теперь встал вопрос кому стать халифом.

    Те кто хотел, чтобы после Али началась новая династия назывались суниты, а те, кто выступал за продолжение династии потомков Али, получили название шииты.

Дело в том, что Коран не просто святая книга, это руководство к действию для мусульман. Дело в том, что кроме Корана существуют ещё и священные предания о жизни Мухаммеда – сутры. Сторонники Али отвергали все сутры, кроме тех, которые были написаны ближайшими родственниками. Шииты тоже раскололись на множество групп. Когда-то один из потомков Али исчез. Но шииты верят, что он жив и вернётся и спасёт всех. И он руководит незримо всем исламским миром – незримый имам.

    \subsection{Имам – глава правоверных и прямой потомок Али.}

Но арабская велика держава через 2-3 века начала распадаться. Первым независимости добился Иран. Они обоснавали свое отделение от державы тем, что имам не из рода Али, и он не законен.

\subsection{ Расцвет исламской философии}

Характерным для арабского мира был рассвет материальной и духовной культуры. А для него требовались знания. И тогда арабы обратились к наследию античной культуры. Были переведены труды Птолемея, Аристотеля, Сократа и т.д. Таким образом, арабы воскресили античное знание. Труды античных философов стали доступны и получили широкое распространение. Возникли библиотеки, обсерватории. Надо сказать, что арабские правители поощряли такое развитие. Они умели использовать достижения античной культуры, не только воскресили, но и внесли что-то свое. Получила развитие физика, механика, алгебра, сказалось это и на философии.

Когда возникает религия, обычно вслед за ней возникает богословие. Дело в том, что религиозные книги не всегда написаны четким, ясным языком. Толковать это можно по-разному. Когда возникает истолкование священных текстов, это и называется богословие. Когда раскалывается религия, возникают разные точки зрения, нужно доказать кто прав. А когда начинается спор, нужно доказывать, обосновывать, опровергать. Возникает проблема - что такое истина, заблуждение.

Так постепенно из богословия начинают возникать элементы философии. В VIII-IX веках возникает исламское богословие. Возникают разные направления. > Исламская философия называется Калам. Расцвет арабской культуры связан с тем, что они обратились не только к античной науке, но и к философии. Труды Платона, Аристотеля и других, всё это огромное богатство было использовано. И так постепенно наряду с чисто религиозной философей возникла философия, которую нередко называют арабский перипатетизм(аристотелизм). Возник синтез ислама и Аристотеля. Появились два центра: Багдад и Кордова (на территории Испании).
Восточная школа арабской философии

Первым представителем арабской философии был Аль Кинди ((ок)800-879). Зарождение настоящей философии, хотя она всё же была связана с Кораном. Его филосфия разделялась на 3 части:

    Алгебра
    Естествознание
    Метафизика (философия)

    **Аль-Фараби. (870-950)**
    Необычайно уважал труды Платона и Аристотеля. Он о них отзывался, благодарил. И в то время не мог удержаться от колких замечаний в адрес Корана, что создавало ему не очень хорошую репутацию среди духовенства, хотя и безбожником он не был. Он не много знал об Аристотеле, и нередко под учением Аристотеля продрузмивал неоплатонизм.

В отличие от Аль-Кинди пытался дать истинного Аристотеля, освобождая его от неоплатонистких настроений, которые были антиаристотельскими. Написал комментарий к трудам Аристотеля. Показывал, что в Коран нужно верить, но разум тоже должен быть, и всё должно быть обоснованно. > Надо принимать во внимание не только то что написано в книгах, но и опыт человечества. Бросил вызов богословию, поставив философию выше богословия.

\paragraph{Ибн Сина. (Авиценна) (980-1037)}

Родом был из кишлака рядом с городом Бухаре. Родным языком был Фарси. Написал множество книг. Самый крупный труд – <<Каноны медицины>>. В нем он учил, как поставить диагноз и как лечить человека. Авиценна опирался на опыт, что делало его философию более приближенной к реальности. <<Книга исцеления>> - это книга, посвященная чисто философии, самый важный философский труд. Это был огромный труд, состоящий из 18 томов. Главные – логика, риторика, поэтика, физика и метафизика. > Он наконец-то освободил арабский аристотелизм от неоплатонизма. Считал, что мир вечен и никогда не был сотворён Богом, что противоречило Корану. Отстаивал детерминизм. Считал, что разум выше веры. Занимался проблемой общего и отдельного. Общее существует в виде божественных идей Аллаха, существует в вещах и существует в наших понятиях.

\paragraph{Омар Хайям (1048-1123)}

Родился в Иране, говорил на Фарси. Писал в основном на арабском, но писал и на Фарси. Был гениальным человеком: величайшим поэтом, великим ученым и не менее великим математиком. Сумел выйти на тот уровень, на который европейская наука вышла в XVII веке. Дал классификацию всех уравнений. Получил формулу аналогичную формуле бинома Ньютона. Раскрыл связь между алгеброй и геометрией, подошел к дифференциальному исчислению.

Писал под напором мусульман и его обвиняли в безбожестве. Его философские работы в основном разъяснения работ Авиценны с разнообразными поклонами в сторону духовенства. Написал большое количество стихотворений, которые неподписанные ходили по всему арабскому миру, они были написаны на Фарси. Оказал большое влияние на запанную культуру. Это произошло благодаря англичанину Эдуарду Фиджиральду, который перевёл 101 его стихов в 1859г. К 1926 г. вышла 156 изданий этой книги. В этих стихотворениях он писал то, о чем боялся сказать в трактатах, поскольку всегда можно было отречься от них. Не отрекается от вина и других развлечений вопреки исламу. Как он сам говорил, он полубезбоник-полумусульманин.

\paragraph{Аль-Газари (Аль-Газели) (1058-1111)}

Написал книгу <<Опровержение философов>>, в которой путем анализа философских текстов он пытается доказать, что Авиценна, Аль-Фараби сами себе противоречат. Опровергал всех философов, ставил на первое место веру, а не разум. Говорил, что мир был сотворён из ничего. Это было развитие в обратном направлении.
Западная ветвь арабской философии

\paragraph{Ибн Баджа. (?-1138)}

Жил на территории арабской Испании. Так же был известен на западе по именем Авен Пация. Написал книгу <<Руководсво одинокого>>, где пытался создать схему идеального государства, где нет войск, насилия и т.д. О философских взглядах почти ничего неизвестно.

\paragraph{Ибн Туфайль. (1100-1185)}

На Западе известен под имением Абубацер. Конечно, существует одна истина, и она выражается и в Коране и в философии. Но Коран написан для простых людей, неспособных абстрактно мыслить, а философия рассчитана на образованных людей. То есть философия выше Корана и богословия.

\paragraph{Ибн Рушд. (Аверроэс) (1126-1198)}

Родился в Кордове. Это вершина арабской философской мысли. Жил при дворе Аль-Мансура. Написал много книг, а самая известная <<Опровержение опровержения>>. Эта книга была направлена на защиту аристотелизма. Был страстным поклонником Аристотеля. Сам он считал, что он просто комментатор Аристотеля. Считал, что в Аристотеле заключена вся мудрость мира, и дальше Аристотеля развития быть не может. Но он говорил, что Аристотель часто искажен, извращен и его неправильно понимают. Написал массу комментариев к Аристотелю, в некоторых были и его собственные мысли на базе его трудов.

    Мир есть. Никакого творения, нет никакого загробного мира, душа умирает вместе с телом. Мир находиться в вечном движении, Бог в это не вмешивается, а Бог это мысль, познающая самого себя. Чудес не может быть, потому что Бог не вмешивается в жизнь и всё имеет причинно-следственной связи. И богословие ничем помочь не может.

Аверроэс понимал, что его работа противоречит Корану. Поэтому он создаёт принцип двух истин. Истина разума и истина души. Наука есть наука, религия есть религия и у них могут быть разные истины. Коран не совершенен, так как выражает истину образным языком, не в адекватной форме. Разум и философия должны быть независимы от религии.

И с тех пор начинается закат арабской философии. В 13 веке побеждает чисто религиозная философия, что вера выше разума, и далее следует долгое топтание на одном месте.


\subsection{Закат исламской философии}

В основном философия на Востоке получила развитие в IX, X, XI веках, начиная с XI она стала глохнуть. Второй всплеск в западном центре арабского мира (Кордова), где появилась такая крупная фигура, как Аверроэс, после чего она стала вновь загнивать, кончилась наука, а вместе с ней и арабская философия. На первый план вышли богословие, мистика и прочие не лучшие продукты духовной деятельности. В последующие века можно отметить только один временный подъем философской мысли, не столько гносеологической, сколько социально-философской.
Ибн Халдун (1332-1406)

На арабском Востоке один из мыслителей создал своеобразную концепцию философии и истории. Этим мыслителем был Ибн Халдун. Это северная Африка и арабская Испания. Ибн Халдун родитлся в 1332 году, умер в 1406г. Его идеи потом нашли отклик и в Западной Европе. Занимал, как и многие мыслители, разные посты при дворе эмира. Был историком, написал грандиозный труд, который называется <<большой историей>> (очень длинное полное название). В ней он подробно рассматривал историю стран ближнего востока на протяжении почти тысячи лет.

Но славу он приобрел не благодаря этой работе, а вступлению к ней, которое называлось <<Мукаддима>> (введение, вступление по-арабски). Объем его 2 тома по 700 страниц (на английском).

    В нем он попытался представить общий взгляд на историю. 2 формы общественной жизни: Сельская форма. Но так как он был арабом он говорил про кочевой образ жизни. Нужда в продуктах заставляла их держаться друг за друга. Был набор моральных норм, которые помогали им держаться друг да друга. Нужда вынуждала их объединяться в государтва и нападать на другие народы. Потом возникли крупные объединения и возникло новое качество – водник город, городской образ жизни. Возникла промышленность и роскошь. В результате этого началось падение нравов. Люди стали враждебны друг к другу. Кроме того, возникло государство, чтобы содержать госаппарат нужны были средства, которые получали путем сбора налогов, аппарат непрерывно рос, и надо было увеличивать налоговое бремя, которое в итоге стало настолько непосильно, что люди начали разбегаться из городов куда глаза глядят, и это подточило основу государства, и оно начинало рушиться. Он выделял пять статий такого государства. И после пятой стадии всё разваливается, под действием соседних племён.

Ибн Халдун выделял 5 стадий развития такого государства. 5 стадия - всеобщая болезнь, гниль общества. Даже если бедные соседние племена не нападали, внутренние пороки и изъяны ослабляли государство, и не могло противостоять обрушившимся врагам, наступал конец династии. Цикл длится приблизительно 120 лет.

Приходили новые дикие племена, завоевывали город, восстанавливали его, приходили к городскому образу жизни... через 120 лет новый крах. Ака <<белый бычок >> (прим лектора). Циклическое развитие, повторение одного и того же. Такие идеи были восприняты западными мыслителями, в т.ч. Макиавелли.

Кроме арабской религиозной философии, в то же время (во время Авиценны, в конце XII века), возникла еврейская религиозная философия, но писали философы по-арабски, поэтому их рассматривают как своеобразную часть арабской философии. Отличия от арабской в том, что арабы пытались применить ислам и Аристотеля или создать синтез языческой философии Платона и ислама. Евреи же примиряли греческих философов с иудаизмом. Одни соединяли иудаизм с неоплатонизмом, другие с аристотелизмом. Можно назвать двоих: Соломон Бен Габироль (1020-1080) (неоплатонизм) и второй крупный Маже Бен Маймон(1135-1204) (сторонник Аристотеля).