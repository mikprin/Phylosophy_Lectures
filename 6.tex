\section{Возникновение классической философии нового времени. Борьба эмпиризма и рационализма.}


Пранаука протонаука в средние века – смесь преднауки и протонауки. И вот здесь возникает наука... Возникновение этого нового естествознания сказалось и на истории философской мысли. Возникла классическая философия нового времени. С ней связан грандиозный взлёт философской мысли, в каком-то смысле сравнимый с античным. От Фалеса до Аристотеля, когда появление каждого нового учения было большим шагом в развитии философской мысли. Вот это время 17-первая половина 19го века – взлёт философской мысли, когда были сделаны великие открытия.
Возникновение классической философии нового времени. Борьба эмпиризма и рационализма.

В начале 17го века перед мыслителями возникли весьма насущные проблемы. Как заниматься научными исследованиями? Что должно быть руководящей нитью? Потребовалась система норм, которые позволяли бы учёным быстро находить правильные ответы на волнующие вопросы. То есть проблемы не столько теории познания, но метода. Ну а вопросы теории познания становились постольку поскольку без этого невозможно было выработать полноценный метод.

Стремление создать новый метод появляется ещё в 16м веке. Пьер Де Ля Роме чётко поставил проблему создания нового метода, которого раньше не было. Другое дело, что он не решил задачу, но проблему он поставил. И в 17м веке проблема встала во весь свой рост и началась разработка новых методов научного познания. Первый вопрос – источник научного познания. Возникли два направления, отличающиеся определением источника познания.

Эмпиризм(от лат. Эмпирио - опыт)– Единственным источником познания является опыт. То есть взаимодействие исследователя с внешним миром. Если такого взаимодействия нет, то знание не появляется.

Есть два вида знания, определённые ещё в древние времена: априорное и апостериорное. – знание, полученное до всякого опыта – знание, полученное в опыте.

Эмпирики считали, что априорным знание быть не может. Всякое знание является апостериорным. Всё уходит корнями в опыт. Надо сказать, что это широкое понятие. Возникал вопрос: <<а что это за опыт?>>. Многие считали, что опыт – это чувственное знание, полученное в процессе воздействия внешнего мира на органы чувств человека. Такие люди назывались сенсуалистами(от sense - чувство) . С их точки зрения нет ничего в уме, чего бы не было в чувственном познании. Иногда эмпириков упрекали в том, что они недооценивали значение разума и всё сводили к чувствам. Однако, большинство понимали, что есть познание чувственное, есть материальное, но источником материала, которым располагает мышление является чувственное восприятие.

Эмпририческому подходу противостоял (лат. рацио – разум). Рационалистов в свою очередь упрекали в том, что они пренебрегали опытом. Они не отрицали значения опыта, но они считали, что существуют априорные знания. В частности, они считали, что существуют врождённые идеи – человек рождается со знаниями в мозгу. И сам по себе разум может быть источником знания.

Рационализм имеет несколько смыслов.

1. рационализм противостоящий эмпиризму

    Разум – главная форма познания

2. рационализм противостоящий иррационализму

Разум имеет значение

3. рационализм противостоящий фидеизму

всё необходимо рационально обосновывать. Если что-то не доказано рационально, то 	принимать это нельзя. 

Рационалисты в первом смысле слова обычно были рационалистами во втором смысле слова, но не наоборот.

Таким образом, в этих веках основная борьба шла между этими двумя течениями:

Самые крупные фигуры:

Основоположником эмпиризма был выдающийся английский философ Френсис Бекон(1561-1625гг) Основоположником рационализма является выдающийся французский философ и естествоиспытатель Рене Декарт(1596 – 1650(?)). Последователем Бекона, его учеником был Томас Гоббс(1588 – 1679г).Надо сказать, что Т.Г. - фигура противоречивая. Он эмпирик, но взял кое-что из трудов Рене Декарта. Его последователь – Джон Локк(1636-1705) От него движение пошло по двум веткам: материализм и идеализм С одной стороны его выправлял, делали последовательным материалистом. Имена: Ламетрин, Гольдбах, Дидро

Идеализм: Джон Беркли,(1685-1753), Давид Юм(1711-1776), Людвиг Фейербах(1804-1872).

Последователи Декарта также разделились на две ветки: с одной стороны материалисты, возглавляемые трудами Бенедикта Спинозы.(1632-1677) (материалист-рационалист) С другой стороны, философия Лейбница(1646-1716гг) (идеалист-рационалист) От Лейбница пошло развитие к представителям той группы философам, которая называется немецким классическим идеализмом. Это прежде всего Иммануил Кант(1724-1804). Иоганн Фихте(1762-1814), Шеллинг(1775-1859), Гегель(1770-1831гг) Шеллинг пережил очень быстрый подъём способностей. Гегель был учеником Шеллинга, который был на 5 лет его моложе. Он бы учеником Фихте, потом Шеллинга и в 18м веке развил свою философию. Шеллинг же быстро выдохся. И наконец, это привело к философии, созданной Карлом Марксом(1818-1886) и Фридрихом Энгельсом(1820-1895), которая объединила в себе материалистическую и эмпирическую линию. Всю философию до Маркса и Энгельса принято называть классической. Всё, что после – неклассическая. До М и Э шёл непрерывный подъём философской мысли, после – либо топтание на месте, либо декаданс.
\subsection{Френсис Бекон(1561-1625)}

Статья из Википедии Выходец из родовитой дворянской семьи. Отец – последний лорд-хранитель печатей – очень высокая должность. В Англии в то время была абсолютная монархия. ФБ был адвокатом, был членом палаты общин парламента, был близок с одним из фаворитов королевы(несмотря на то, что её звали королевой-девственницей) – графом Эссекским. Последний способствовал продвижению Френсиса Бекона, подарил ему имение. Граф Эссекский поссорился с королевой, его отдали под суд и казнили, Бекон составлял судебное заключение. Александр сказал о нём <<самый мудрый, самый блестящий и самый низкий представитель рода человеческого>>. Подъём его карьеры начался, когда на престол Англии вступил король Яков I. Бекон становится лордом Хранителем Печатей, а затем – Лордом Канцлером Англии. Он становится бароном, виконтом. Однако в 1613м году его отдают под суд за взяточничество. Его карьера обрывается. Его заключают в тюрьму, налагают гигантский штраф и лишают права занимать любые государственные должности. Однако, Яков I освободил его от тюрьмы и от штрафа, а Карл I позвал на должность премьер-министра, однако тот отказался. 1 работа 1605 год – о преуспевании наук

     1623 – о достоинстве и увеличении наук
	Новая Атлантида – утопия об обществе, в котором существуют специальные 		учреждения, занимающиеся наукой <br />
	История Генриха VIII

умер в результате неудачного опыта по замораживанию курицы. Бекон понимал великое значение науки, хотя сам не был естествоиспытателем. Надо сказать, что в понятие науки тогда вкладывали и философию.
По поведению Бекон был верующим. Он совмещал религию и материализм следующим образом: Есть две независимые истины: Область разума, где не место вере и область веры, где не место разуму. Он ценил область разума. Объектом науки является природа. Целью науки является изучение природы, чтобы человек мог господствовать над природой. Знание-Сила. Но для этого нужно научиться познавать природу. Нужен метод. Но для этого не подходит метод, предложенный Аристотелем. И в противовес дедуктивной аристотелевской логике он вводит индуктивную логику, то есть логику, подразумевающую переход от частного к общему. Его работа называется <<Новый Органон>> в противовес Аристотелевской логике – Органону. Данные следует искать не в книгах, а из опыта. Все особенности явлений должны быть записаны, ничего не упуская, проверяя и перепроверяя. Если у нас есть прочный фактический базис, можно делать верные выводы. Он выделял три типа учёных: Паук – сам вытягивает из себя знания. Выводит все из своих суждений, не прибегая к реальности. Муравей – собирает весь материал, тащит в муравейник и не обрабатывает. Пчела – собирает нектар и перерабатывает их в мёд. - это единственный вид учёных, который он признавал. <<Пчелиный>> метод – и есть метод индукции. Важнейшая задача – выявление причинно-следственной связи. Он создает и обосновывает несколько методов для их выявления – в частности, методы сходства, различия и метод сопутствующих (?). Таким образом, Бэкон положил начало исследованию причинно-следственных связей. Важнейшая задача – анализ. Каждое явление надо анатомировать Например, золото. Золото жёлтое, поэтому в него входит жёлтая натура Золото твёрдое - твёрдая натура. Это так называемые простые натуры. Мы можем создавать новые вещества, объединением натур. Это примитивный подход, но он дал начало анализу. Главнейшее – это раскрытие форм и законов. Важнейшим свойством материального мира он считал движение. Материя находится в постоянном движении и в отличие от своих предшественников он считал, что существуют разные формы движения. Учение о четырёх идолах. На пути научного познания есть четыре идола. 1)Идол Рода Есть особенности организма, которые препятствуют познания 2)Идол Пещеры Человек всегда выращивается людьми, поэтому его мировоззрение навеяно другими людьми 3)Идол Площади или Идол Рынка Человек принимает решение, потому что оно общепринято и соответствует представлениям окружающих 4)Идол Театра – Всевозможные философские системы, которые, возможно, ошибочны. Иначе говоря, Идол Авторитетов.
\subsection{Рене Декарт(1596-1660)}

Статья из Википедии Родился во франции в семье чиновника. Его отдали в колледж иезуитов. У него было прекрасное знание латинского языка. Денег у солдат было немного, поэтому он стал наёмным солдатом. Посмотрев Европу, вернулся в Париж. Занялся наукой, уехал в Голландию – первое в мире буржуазное государство, славящееся своей терпимостью. Однако даже там картезианство было запрещено. Декарт был великим естествоиспытателем и математики. Положил начало аналитической геометрии. Занимался физическими опытами, открыл закон преломления света. Как и Бэкон, ценил роль науки и считал, что наука должна обеспечить человеку господство над внешнем миром. Важнейший принцип – отвергать мнение авторитетов. Во всём надо сомневаться и проверять любое утверждение, не важно, кем оно было сделано. В своих работах он разрабатывает свой метод познания для этой проверки. Правила для руководства ума. "Рассуждения о методе для хорошего направления разума и отыскания истинных наук" Работа была издана на французском языке, что вызывало некоторые опасения у власть предержащих, так как мысли, содержащиеся в ней, могли быть прочитаны и, что самое ужасное, поняты простыми людьми. Поэтому после смерти Декарта ео учение было запрещено к учению и преподаванию на всей территории Франции.

Вот основные положения работы "Правила для руководства ума":

    Подвергать всё сомнению, принимать только то, что очевидная истина. Признак истины – очевидность.
    Необходимо разделять все вещи на части и собирать целое из частей
    Необходимо восходить от простого к сложному.
    Необходимо ничего не упускать из виду.

Главный метод у Декарта – метод дедукции. Это было связано с тем, что область приложения методов Декарта, в отличие от Бэкона, была математика. Поэтому идеальный метод познания в его понимании был определение аксиом и <<накручивание>> вокруг них теорий. Первое основоположение: Я мыслю, следовательно я существую. Мышление есть духовный процесс. Человек – духовная субстанция. По Бэкону Бог – врождённая идея, следовательно Бог существует, а Он не может нас обманывать. Следовательно, существует и мир. По Декарту Субстанция бывает духовная и материальная. Духовная бывает бесконечной(Бог) и конечная(душа) и обладает свойством мышления. Материальная субстанция обладает свойством протяжённости. По Декарту Бог создал мир со своими законами и дальше этот мир развивается по своим законам. Таким образом, Декарт первым создал научную картину мира. Причём у Ньютона Бог создал Солнечную систему, а по Декарту - создал законы. Декарт ввёл реляционную теорию. Предложил механистическую картину мира. Он говорил о том, что человек реагирует на внешний мир рефлексами, причём ввёл понятие врождённого и приобретённого рефлекса. Животный организм – это сложный механизм, ничего чудесного. А в организме человека есть душа. Причём взаимодействие между душой и организмом происходит через шишковидную железу, находящейся в мозгу. Душа находится в Шишковидной железе и трясёт её.