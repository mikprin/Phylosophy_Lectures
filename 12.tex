

\section{Система созданная Гегелем. Философия Гегеля.}


Гегель писал настолько запутанно и сложно, что его очень часто называют шарлатаном. Шопенгауэр и Гегель постоянно спорили между собой, Шопенгауэр называл Гегеля шарлатаном, глупцом. На самом деле Гегель был величайшим философом, сделал много открытий, которые повлияли на развитие философии. Гегель был объективным идеалистом (как и Шеллинг, а Фихте был промежуточный между субъективным и объективным идеализмом). Первым создателем объективного идеализма был Платон. История объективного идеализма завершается на Гегеле.

Напомним, в чём различие между объективным и субъективным идеализмом: и те и другие кладут в основу мира некое идеально духовное начало. Субъективные идеалисты кладут некое простое элементарное единое начало и только. Немного сложнее у объективных идеалистов: есть субъективное сознание (<<моё сознание>>) и объективное сознание (<<ничейное сознание>>, <<абсолютный разум>>, <<абсолютный дух>>, <<абсолютная идея>>) последнее созидает объективный мир и рождает субъективное сознание.

Возникает вопрос: откуда возникло представление об объективном сознании? Существует ли объективное сознание? Знание, накопленное человечеством, есть объективное сознание, оно выступает как первичное сознание к субъективному сознанию (<<моему сознанию>>). Объективные идеалисты открыли это объективное сознание (как понятие) и <<пустили гулять по миру>>. Это открытие впервые совершил Платон (<<система эйдосов>>).

У Гегеля объективное сознание – <<абсолютный дух>>, субъективное сознание – <<дух человеческий>>.

Первая крупная философская работа Гегеля <<Феноменология Духа>> (1807г.). В ней Гегель исследует сознание человеческое, прослеживает, как оно накапливается. Процесс познания по Гегелю это процесс совпадения объективного и субъективного духа. Главное открытие Гегеля, посредством этой работы: он первый взглянул на истину как процесс. Истина есть результат и процесс одновременно. Поставил проблему абсолютной и относительной истины. На каждом этапе процесса истины, наши знания относительны, наше соответствие с миром относительно. Т.е. истина есть процесс, когда относительная сознание стремиться к абсолютному сознанию. Гегель считал, что для доказательства правильности теории надо знать не саму теорию, а как к этой теории стремились. Чтобы показать, что теория истинна, надо показать, что она - неизбежный результат человеческого развития к ней. В работе <<Феноменология Духа>> Гегель собрал весь материал для своего главного открытия: есть такая вещь как мышление, мышление на поверхности выступает как человеческая деятельность, как процесс наших действий, как процесс наших операций (мы берём понятия, соединяем эти понятия и получаем суждение). Ранее на мышление смотрели как на субъективную деятельность человека (Аристотель: <<формальная логика>>). А Гегель открыл, что мышление есть ещё и объективный процесс, ведущий к объективным, независящим от мышления человека, законам. Т.е. открыл, что есть два вида мышления. Ранее Шеллинг, Кант догадывались, что есть другая форма мышления (отличная от субъективной формы), но только Гегель смог показать это. Разум – это мышление как объективный процесс, рассудок – это мышление как субъективный процесс (диалектика и формальная логика соответственно). Не нужно смешивать рассудок и разум. Объективный процесс движется, развивается, что видно на примере: наука.

Гегель создал новый взгляд на мир. Раньше мир рассматривали, как совокупность вещей, особенно наглядно видно в картине мира Ньютона. С точки же зрения Гегеля мир есть процесс, причём саморазвивающийся процесс (независимый), этот процесс делится на несколько процессов, которые в свою очередь делятся на другие процессы (все они саморазвивающиеся – в философии это понятие <<спонтанные процессы>>!). Эту теорию Гегель выдвинул в начале 19 века, а наука пришла к этому в 20 веке (теперь это в науке называется <<самоорганизующиеся процессы>> и изучает их фенольгетика). Для Гегеля каждый реальный процесс называл <<историческим процессом>>.

Пример частного логического процесса: физический процесс, экономический процесс (развитие капитализма), <<Капитал>> (труд Карла Маркса) – описан процесс развития капитализма вообще.

Предельные общие понятия есть категории диалектики. Вопрос в том, каким законам следует логический процесс. Логический процесс, имеющий содержания самого себя есть процесс развития законов диалектики по законам диалектики. Мышление есть объективный процесс идущий по объективным законам.

Гегель, пожалуй, первый из тех, кто понял, что развитие философии есть закономерный процесс, происходящий в независимом сознании воли людей, которые участвуют в этом процессе. Всякое великое учение есть звено в этом процессе. Гегель развил диалектику понятий. . Есть диалектика мышлений (диалектика развития понятий) и диалектика вещей. Основной вопрос философии, что первичное, а что есть вторичное понятие, диалектика мышления или диалектика понятий. В книге <<Энциклопедия философских наук>> (3х томник) разработана эта система.

Первая часть системы Гегеля (<<Энциклопедия философских наук>>) – это <<Логика>> (первый том). <<А ГДЕ НАХОДИТСЯ МИР?>> - СПРОСИЛ ЛАН. <<Логика>> имеет 3 части: <<Учение о бытие>>, <<Учение о сущности>> и <<Учение о понятии>>. Как идёт развитие мысли у Гегеля: вначале появляется чистое бытие, которое настолько неопределенно, что оно равно ничему (небытию), далее что происходит – бытие переходит в ничто и ничто переходит в бытие, возникает качество, вслед появляется количество, потом появляется мера, при нарушении которой происходит превращение и возникает сущность. И ТУТ ЛЕКТОР ПОПРОСИЛ ЛАНА ЗАКОНЧИТЬ РАЗГОВОРЫ. Вторая часть учений Гегеля – <<Философия природы>> (второй том). Начинает с механики, потом переходит к физике, потом к органике, потом к человеку, а когда появляется человек, то и появляется субъективный дух, который, реализуя, абсолютные идеи в человеческом обществе плодит абсолютный дух. Все стадии развития природы находятся в абсолютном духе. Третья часть учений Гегеля – <<Абсолютный дух>> (третий том). Философия духа представлена в трёх частях: <<Субъективный дух>>, <<Объективный дух>> и <<Абсолютный дух>>. В <<Субъективном духе>> он рассматривает проблемы психологии (восприятия, ощущения, мышления, рассудок как явления психические не как в теории познания). <<Объективном дух>> он разделяет на разделы: a) право b) моральность c) нравственность:

aa) семья
bb) гражданское общество
cc) государство
	aaa) внутреннее право
	bbb) внешнее государственное право
	ссс) всемирная история

<<Абсолютный дух >> он делит на формы: 
\begin{enumerate}
\item искусство 
\item религия откровения 
\item философия
\end{enumerate}