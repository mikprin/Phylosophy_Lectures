
%Лекции по философии науки проф. Семенова Юрия Ивановича



\section{ПРОДОЛЖЕНИЕ ЛИНИИ ЭМПЕРИЗМА}
Лекция 8, семестр 9, 22.10.2005
\subsection{Джон Локк. (1632-1704)}

Английский философ. Родился в семье состоятельного адвоката. Окончил Оксфордский университет. Получил диплом врача. В 1667 году познакомился с Лордом Эшли. Стал у него домашним врачом, потом стал секретарём Лорда Эшли, который в 1672 стал лорд-канцлером Англии. В 1682 году бежал за границу из-за преследований Эшли. Спустя 3 года пришел к власти новый король Яков II. В том же году вспыхнул мятеж. После чего стал королём Вильгельм. И в Англии воцарилась конституционная монархия, и вся власть перешла к парламенту и премьер-министру. После революции Локк вернулся. Написал не мало работ по проблемам общества, проблемам государственного строя и т.п. и т.д., не особо оригинальны.

<<Исследование о человеческом разуме>> - самая крупная работа Локка написана в 1690 году. Его предшественники занимались разработкой метода научного познания, но Джон Локк начал исследовать сам процесс познания. Он начал разрабатывать гносеологию. Он исходил из того, что главный инструмент познания – это разум. Значит нужно сначала изучить работу разума, а потом понять как сущность процесса познания. Пытался обосновать эмпиризм, что опыт – единственный источник научный знаний. Врожденных идей нет. > Душа человека пуста, пока не начнется опыт, и только тогда душа наполнится знанием. Локк не был сенсуалистом, потому что допускал существование двух разных типов человеческого опыта – внешний и внутренний опыт. Внешний опыт – чувственное познание. Внутренний опыт – деятельность разума (рефлексия). С эти связано появление двух типов идей. Под этим он понимал ощущения и восприятия. Идея чувственная и идея рефлексивная. О внутреннем опыте писал не много и не четко. Разработал концепцию внешнего опыта.

В том, что существуют предметы внешнего мира, он не сомневался. Когда перед нами возникает идея вещи, эта вещь в нашем сознании обладает разными качествами. Когда эту вещь мы представляем, не все эти качества существуют в реальном мире.

    Вещь в нашем сознании и вещь как она сама по себе не полностью равны. Есть такие качества, которые равны (протяженность форма, вес…), а вот что касается цвета, запаха, вкуса – это качества вторичные, они возникают только в сознании (в многом повторяет Демокрита, Гоббса…).

В начале действует наш разум, возникает два вида идей: простые идеи чувств и простые идеи рефлексии. Практически рефлексия – это рационализм, и это была не последовательность в идеях Локка. Здесь ум не участвует, потом ум начинает их перерабатывает, и появляются сложные идеи. Это идеи создаваемые умом. Что соответствует эти сложным идеям. Он был концептуалистом, наши понятия существуют только в нашем сознании. А что стоит за тем, что мы прилучаем в ходе наших ощущений? А существует ли субстанция.

    Две субстанции, несомненно, есть – одна субстанция это вещи, а когда мы мыслим, мы имеем дело с духовной субстанцией. Потом встал вопрос, а какого соотношение между ними. Можно допустить, что одна порождает другую, впрочем, тут есть и элементы мистицизма. Когда он подводит итоги, он выбирает 3 вида познания:

    Интуитивное. (Истины разума Лейбница) на столько что мы не можем усомниться в этом. И не нужны доказательства. Вообщем-то это рационализм.
    Демонстративное. Когда нужно вывести из одной идеи другую, показать истину.

    Сенситивные познания.

\subsection{Джордж Беркли. (1685 – 1753)}

Жил во время, когда с абсолютизмом было покончено. Главное было сохранить существующие порядки. А для того, что бы сохранить существующий строй нужна религия.

    Об этом говорил даже Гоббс, что религия ложна, но нужна, чтобы управлять народом. Философия Беркли – борьба с материализмом. Материализм – основа атеизма. Он был воинствующий идеалист.

Родился в Ирландии. Окончился Христианский Колледж. Прекрасно знал латинский и греческий языки. Побывал в Америке и Европе. Добрался до Калифорнии, где город Беркли назван в его честь. Потом вернулся на родину, где был назначен священником.

Опровержение материализма по двум линиям:

Другая линия – сенуализм. Выступает как чистый сенсуалист. Мы имеем дело с огромным множеством ощущений и восприятий. Мы имеем дело с огромным количеством идей. Эти ощущения соединяются, и создаётся коллекция идей – это и есть вещи. Мы имеем дело с ощущениями и коллекциями ощущений. И никаких других вещей кроме как в ощущениях быть ничего не может. Работы <<Новая теория зрения>> – 1709 г. где он пытался использовать новые данные об устройстве зрения для объяснения своих теорий. Далее в 1710 году появилась работа <<Трактат о принципах человеческого познания>>. <<Три разговора между Геласом и Филонсом>> – популярная книжка, написанная в форме диалога, приводит возражения материализма и критикует их. Пример: боль. Есть такое как ощущение боли. А есть ли такой предмет? Нет, значит – боль это только ощущение. А что вызывает боль? Если взять горячую вещь, то она вызовет боль, значит – тепло существует только в сознании, а если вещи холодные и горячие, то и сами вещи существуют только в сознании>>. Кроме ощущений ничего нет. Но с другой стороны все вещи существуют на самом деле, то есть не отрицает существования мира. Быть – значит быть воспринимаемым. Тогда Беркли водит понятие мыслящих духов. И субъект один из мыслящих духов. То есть становится на позиции объективного идеализма. Еще одна проблема, что человек может вообразить того, что не существует и чего не может быть. Беркли решает эту проблему следующим образом: отличаются эти предметы четкостью. Есть мыслящий всемогущий дух. И то, что вкладывает высший дух, и то, что он творит – есть объективно. Так он вступает на позиции объективного идеализма. Беркли исходил из теории Локка о вторичных и первичных качеств, но потом доказывал, что качества являются только вторичными.
\subsection{Давид Юм (1711 — 1776)}

Продолжатель дела Беркли и Локка.
Биография

Родился в 1711 году в г. Эдинбург в семье юриста, владельце небольшого поместья. Юм получил хорошее образование в университете Эдинбурга. Работал в дипломатических миссиях Англии в Европе. Написал массу работ на разные темы, в том числе историю Англии в восьми томах.

Начал философскую деятельность в 1739, опубликовав первые две части <<Трактата о человеческой природе>>. Через год вышла вторая часть трактата. Первая часть была посвящена человеческому познанию. Потом он доработал эти идеи и опубликовал в отдельной книге – <<Очерк о человеческом познании>>.
Философия.

Является эмпириком и сенсуалистом. Никаких врожденных идей быть не может. Всё познание образуется в процессе человеческого опыта. Вводит понятие концепции. Все концепции делятся на основные категории:

\begin{itemize}
    \item Впечатления (для них характерна необычайная точность и ясность)
    \item Чувственное впечатление.
    \item Рефлексивное впечатление.
    \item Идеи (представления)
\end{itemize}

После восприятия материала, познающий начинает обрабатывать эти представления. Разложение по сходству и различию, далеко друг от друга или рядом (пространство), и по причинно-следственной связи. Всё состоит из впечатлений. А каков источник ощущения восприятия? Юм отвечает, что существует, по меньшей мере, 3 гипотезы.

    Есть образы объективных предметов (теория отражения,материализм).
    Что мир – это комплекс ощущения восприятия (субъективный идеализм).

    Что ощущение восприятия вызывает в нашем уме Богом, высшим духом (объективный идеализм).

    Юм ставит вопрос, какая же из этих гипотез верна. Для этого надо сравнить эти типы восприятий. Но мы закованы в черте нашего восприятия и никогда не узнаем что за восприятием. Значит вопрос о том, каков источник ощущения – принципиально не разрешимый вопрос. Всё может быть, но мы никогда не сможем это проверить. Никаких доказательств существования мира не существует. Нельзя ни доказать ни опровергнуть.

В дальнейшем через 100 лет его концепцию стали называть агностицизмом. Создаётся ложное впечатление, что Юм утверждает, как будто знать вообще ничего не возможно, но это не совсем так. Содержание сознания мы знаем, значит мир в сознании известен. То есть мы знаем мир, который является в нашем сознании, но мы никогда не узнаем сущности мира, мы может узнать только явления. Такое направление носит название феноменализма. На этой основе построено большинство теорий современной западной философии, утверждающих неразрешимость основного вопроса философии. Причинно следвенные связи в теории Юма – это результат нашей привычки. А человек – это пучок восприятий.
