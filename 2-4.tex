
Лекции по философии науки проф. Семенова Юрия Ивановича
Лекция 4, семестр 10, 4. 3.2006
Психофизиологическая проблема в истории философской мысли.

Иероглифический материализм. Название возникло очень поздно, но такого рода концепции были раньше, начиная с самого Демокрита, хотя он и не употреблял таких понятий, как знак, иероглиф, символ, но по существу это был иероглифический материализм. Мы уже выяснили, что материализм в завершенном виде говорит, что ощущения не похожи на внешний мир, являются символами, иероглифами – такая трактовка совершенно не укладывается в понятия семиотики, науки о знаках. С точки зрения семиотики, знак – это материальный иероглиф(?). То есть знак - не идеальный. Таким образом вопрос здесь вопрос не в отношении между разумом и материальными явлениями.

Вопрос об отношении между материальным и идеальным. И здесь ни в чем разобраться нельзя.

    Можно все же попытаться разобраться в том, что такое идеальное, раскрыть природу идеального и прежде всего отношение между идеальным и всем миром, а также между идеальным и тем органом, который рождает идеальное – человеческим мозгом.

Когда мы берем то, что называется сознанием, идеальным, психическим, и говорим о том, что оно, скажем, вторично, то здесь выступают две стороны:

    Отношение идеального к внешнему миру. Идеальное здесь выступает как нечто явно отличное от материального. Жареная курица в нашей голове и реальная жареная курица – не одно и то же! Сытым от жареной курицы в голове не станешь. Принципиальное различие между бытием в сознании и бытием вне сознания. С такой точки зрения сознание, ощущения, восприятие, мысли никак материальными не назовешь. Есть нестираемая грань. Это отношение идеального и мира, образом которого является идеальное - это гносеологическое, теоретико-познавательное отношение.
    Всякое идеальное порождается мозгом, находится в определенном отношении к мозгу, здесь возникает проблема соотношения идеального и тех процессов, которые происходят в мозгу и, бесспорно, являются материальными процессами – физиологическими. Это отнологическое отношение материального к идеальному. Обычно это рассматривается как отношение между психическим и физическим, причем под физическим подразумеваются природные, материальные явления. Когда возникла наука о коре головного мозга, стали исследоваться процессы, эта проблема стала называться психофизиологической. Она и сейчас обсуждается; в англоязычной литературе она называется mind-body problem или mind-brain problem. Проблема ума-тело или ум-мозг. До сих пор никто не предложил такого решения, какое бы всех устроило.

Эта проблема возникла еще в античные времена, а с 16-17вв она пришла к более конкретной форме. Рассматривать все течения не будем, самые основные предлагаемые решения этой проблемы.

Исходили из того, что существует какое-то однозначное соответствие между процессами психическими и физиологическими. Это констатация факта. Каждому изменению физиологических процессов соответствует изменение психических и наоборот. В этом сходятся все. Но какая связь – непонятно.

    Дуализм (Психофизиологический дуализм) - первое толкование: психическое и физиологическое это две разные субстанции: духовное и материальное, которые друг от друга не зависят. Самостоятельные виды бытия. Наиболее четко это было выражено в философии Декарта, который прямо говорил о двух независимых субстанциях. Внутри есть два течения
    Психофизическое взаимодействие. Согласно нему, две субстанции друг на друга влияют. В частности, душа управляет телом. Этой концепции по сей день придерживаются крупные западные физиологи. Психофизиологический интеракционизм (то же самое).

    Психофизический параллелизм. Эти процессы текут параллельно и никогда не могут друг на друга влиять. Бог – истинная причина, он приказывает всем, его все слушаются. Эту точку зрения пропагандировал последователь Декарта Николя Мальбранш и все окказионалисты.

    Психофизический монизм: духовное и физическое – два проявления одной и той же субстанции. Эту идею пропагандировал Спиноза. Можно трактовать по-разному (Спиноза придерживался примата(первичности) материи, однако допускал, что вся материя одушевлена). Можно считать, что примат за духом, или же считать, что первично третье, которое никак не назовешь, а двумя проявлениями являются психическое и физическое.
    Эпифеноменализм. Эта концепция рассматривает духовное как эпифеномен. Суть его в том, что духовное есть сопровождение материальных процессов, ни на что не влияющее и никакой роли не играющее.

Ученые пришли к выводу, что эта проблема неразрешима силами человеческого ума. Эмиль Дюбуа-Реймон написал книгу о границах человеческого познания <<Семь мировых загадок>> (1872). В число этих загадок вошло взаимоотношение психического и физиологического. Эрвин Шредингер в 1946г. написал книгу <<Что такое жизнь>>. Больше всего его внимание привлекла проблема идеального, духовного. Эту проблему мы не можем решить, она аболютно неразрешима, по Шредингеру.

Материалисты 17-18вв. решали эту проблему, исходя из первичности материи. Духовное, психическое – свойства материи. Поиск более конкретного решения приводил их либо к тому, что духовного нет, либо это свойство никак не влияет на материю. Некоторые пытались более конкретно решить проблему: например, Пьер Кабанис. Он считал, что мозг выделяет мысли так же, как печень выделяет желчь. Вульгарный материализм. Материалист Карл Бюхнер не соглашался, т.к. желчь можно собрать в сосуд, а мысль не является веществом и в этом смысле не является материей. Её нельзя потрогать, проанализировать. Мысль относится к мозгу, как магнетизм к магниту.

Последнее время (лет 15) интерес к теме пробудился у западных философов, когда рухнул неопозитивизм, который считал, что проблемы соотношения духа и материи – псевдопроблемы, это бессмысленная вещь и заниматься ею ни к чему. После этого появились философы, которые говорили, что некоторые из старых проблем имеют смысл, в т.ч. mind-body problem, и попытались ее решить. Все они пытались ее решить с точки зрения материализма. Это направление получило название научного материализма. Научные материалисты занимались только mind-body problem. Они называли себя научными, т.к исходили из конкретных данных о работе мозга.

Направления научного материализма:

    Элиминативный материализм.
    Редуктивный материализм.

И те и другие сводили духовное к материальному - физиологическим процессам – физическим процессам. Поэтому их точка зрения называется физикализм. Сведение без остатка идеального к физическим процессам.

    Элиминативный: все слова <<сознание>>, <<ум>>, <<воля>>, <<радость>>, <<любовь>> упразднить, заменив нейрологическими терминами, как ненаучные.
    Редуктивный: отказываться от слов не следует, зачем? Заменять надо только при научном описании.

Со временем, когда стало понятно, что так просто все не объяснишь, возникло новое течение внутри научного материализма - функциональный материализм: духовное – это функциональное состояние нервной системы. Но как не пытались раскрыть, что такое функция, состояние, вещество, ничего не получалось. Но некоторые поняли, что и это слишком упрощенно и возникло четвертое направление:

Эмерджентистский материализм. Когда мы видим развитие животного мира, мы видим непрерывное возрастание сложности. Возникают все более разнообразные процессы, новые качества, не сводимые к старым. Материальное – это элементы, а духовное – единство. Например, культура - эмерджентный признак.
Советская философия:====

С 1932г. по 1953г. – полный застой, исчезание науки. После смерти Сталина и XX съезда мысль слегка раскрепостилась, ограниченная свобода.

Большой популярностью пользовалась работа И.П.Павлова <<О высшей нервной деятельности>>. Павлов – великий физиолог, занимался проблемами работы коры головного мозга. Создал законченное учение о высшей нервной деятельности, развивал его, но после его смерти ученики стали повторять одно и то же. Его учение прекрасно объясняет поведение животных. Поведение животных определяется инстинктами. У него есть условные рефлексы, которые ведут его. Они меняются, и животное гибко реагирует на изменение окружения. Нет никаких психических процессов у животных. И, кстати, с его точки зрения, у человека тоже.

У животных нет души, а у человека есть. И когда он распространял свое учение на человека, это было ошибкой.

Некоторые его сторонники считали, что то, что мы раньше считали психической деятельностью есть просто высшая нервная деятельность и нужно отказаться от терминов типа эмоция и пользоваться только физиологическими терминами. Были те, кто говорил - пользуйтесь терминами, но считали физиологическую деятельность единственным видом деятельности. Были и те, кто считал, что психическая деятельность есть, но ее основа – физиологическая деятельность, но сама она все же существует. И они делились на 2 течения: одни считали, что психическая деятельность нематериальна, а другие – что это форма материальной, только другая, нефизиологическая. Все эти люди были в затруднении – с одной стороны все это материально, а как ввести идеальное?

Некоторые говорили, что есть два отношения, в гносеологическом отношении ощущение идеально, а в онтологическом материально. По отношению к миру восприятие, ощущения идеальны, а к мозгу – материальны, вот и все. Было обилие точек зрения, вплоть до такой: вообще нет психофизиологической проблемы, это псевдопроблема. Потом он пересмотрел, что эта проблема есть, но она второстепенная, научная и к философии отношения не имеющая.

Что мешало философам решить проблему: когда говорят о духовном, идеальном, психическом, существуют два качественно отличных вида идеального: 1. Присущий животным. Идеальное – биологическое. Биоидеальное. Это есть и у человека. 2. Присущий только человеку. Высший вид идеального, проявлением которого является мышление. Является явлением социальным. Социоидеальное. Раз существуют 2 вида идеального, надо отдельно решать их вопросы! Займемся исследованием только биоидеального. Мышлением пока мы не занимались.
Тема 4. Природа низшей формы идеального – биоидеального

Предположим, на какой-то рецепторный прибор организма подействовал внешний раздражитель. Возник нервный процесс, по нервным волокнам достигший коры головного мозга. В ней возникает ощущение, восприятие. Чтобы понять природу этого идеального, нужно поставить вопрос: в чем сущность этого процесса? Чем он отличается от других процессов в организме – пищеварения, кровообращения? Принципиальное отличие есть. Процесс пищеварения важен сам по себе. А нервный процесс важен тем, что несет в себе информацию о внешнем раздражителе. Кроме физиологического содержания, есть информация, сообщение о внешнем предмете, который вызвал этот процесс. Как телеграфные провода против силовых – энергетический процесс и информационный. Сигнал важен не сам по себе, а как носитель информации.

В середине XX века информация была открыта как особое явление, возникла наука, изучающая процессы возникновения, хранения информации – теория информации, информатика. Понятия теории информации давно пытались использовать для решения психофизиологической проблемы. Дубровский, в последней книге писал, что <<невозможно понять природу идеального, не прибегая к теории информации>>. Он не завершил теорию, ему могло помешать, что он не смог отделить 2 вида идеального - социального идеального и биологического. Другие занимались этим, особенно отечественные. Попытаемся разобраться, что же там происходит.

Информация – это данные, сведения о чем-то. Легко понять, что информации не может быть без материального носителя. Информация предполагает существование этого материального носителя. Что может им являться? Два вида носителей:

    Какой-то физический процесс. К примеру, электричество.
    Физический объект, тело. Например, фотография.

Особенности, которыми должен обладать объект или процесс, чтобы нести информацию: процесс должен обладать многообразием состояний. Физический объект должен обладать множеством элементов. Это множество называется кодом, шифром информации.

Требуется информационный объект, носитель с кодом и получатель информации. Будем рассматривать те случаи, когда получатель – человек.

Все начинается с того, что объект действует на носитель информации и происходит кодирование информации. Возможно кодирование и перекодирование информации. При перекодировании меняется коды, объекты, процессы, но информация сохраняется, причем сохраняется как порядок расположения элементов множества. Норберт Винер в книге <<Кибернетика>> сказал, что <<информация есть не дух, не энергия и не материя>>. На что советские материалисты начали обвинять его в идеализме. Идеализма нет. Сейчас с этим все, кто этим занимается, соглашаются. Порядок – объективная вещь.

Оказывается, в мире существуют две формы объективно реального:

    Явления, существующие сами по себе. Это самобытие. Они являются материальными.
    А есть явления, которые существуют в чем-то через что-то. Например, порядок. Если нет шашек, нет порядка их расположения. Семенов ввел термин <<Объектальное>>. Объективное, нематериальное. Самого по себе не существует.

Как видно, существует два вида нематериального: идеальное - субъективный образ объективного мира, и объектальное. Если мы запишем голос человека, а потом его прослушаем, то отличить от оригинала его будет очень сложно. В этом процессе после кодирования, перекодирования, и декодирования мы получим дубликат голоса. При декодировании получается копия, которая ничем не отличается от оригинала. Всегда ли так? Фотография. Фотопортрет несет информацию о человеке, он, кстати, объективен. Сам портрет материален, а информация, заключенная в нем, объектальна. Фотография - есть единство носителя и инфомации, единство материального и объектального. Однако, на фотографии мы имеем дело лишь с образом человека, в отличие от записи голоса, что есть дубликат.

Когда нервный процесс достигает коры головного мозга, как он существует для организма? В организме есть материальный процесс. В коре головного мозга извлекается информация, возникает восприятие. Сам нервный процесс не важен для организма, важна информация, в которой заложено сообщение об объекте – источнике информации. Для организма эта информация существует, как вне находящийся объект(например, яблоко), иначе говоря, нервный физиологический процесс для организма существует как вне находящийся объект. Организм становится также и субъектом, происходит раздвоение на объект и субъект. Одновременно раздваивается и внешний мир – возникает его отображение в сознании.

Нервный процесс является субъективным образом объективного мира. То, как внешний объект будет выглядеть для меня, зависит от двух факторов –

    От природы объекта
    От того, как устроена нервная система и в каком состоянии она находится.

В информатике есть понятие полноты описания, полноты информации. Если нет цветного зрения, полнота ниже, чем если есть. Существа с цветным зрением лучше отображают, понимают мир. Когда мы видим, что существует понятие полнота описания, т.е. соответствие может быть более полным и менее полным, рушится теория символов, т.к. символ не может обладать полнотой описания, символ не может быть более похожим. Раз есть полнота описания, значит это не символ, а образ внешнего мира. В развитии животного мира возникают все новые органы чувств. С точки зрения Гельмгольца, чем больше органов чувств, тем хуже – много лишнего. А на самом деле, с каждым новым рецепторным прибором мы с все большей тонкостью видим мир. Неплохо, например, сейчас обладать встроенным дозиметром. (Лучше спиртометром (прим. ред))

Тут довольно долгие споры (гдето 40:00-50:30). Кто хочет, может послушать и все важное записать Идеальное - свойство материального процесса, несущее информацию, существовать как тот предмет, о котором идет информация. Нервный процесс преобразует информацию в предметы, во внешний мир.

У животных есть субъективный мир, есть идеальное, но оно существует не как какой-то процесс, а как особое свойство материальных нервных процессов.
Тема 5. Время и пространство мира для нас и мира в себе

Все предметы в нашем субъективном мире занимают определенные места. Они упорядоченны, и их порядок называется пространством. Кроме вещей, в мире существуют события. События упорядочены во времени. Это 2 формы порядка. В мире для нас это формы нашего чувственного созерцания. Кант считал, что это формы нашего чувственного созерцания, при помощи которого упорядочивается наш хаос ощущений. Ошибка его в том, что он считал, что они упорядочивают только мир для нас. А на самом деле они же упорядочивают и мир в себе.

Человека давали очки, которые переворачивали картинку. В результате образовался разрыв между миром в себе и его восприятием, и человек все время падал. Потребовалось несколько дней, чтобы он приспособился. Потом очки снимали и все начиналось сначала. Он подгонял мир в нем к объективному миру. Гельмгольц утверждал, что гармония между миром в себе и миром для нас возникает в результате опыта. Пример из Эдгара По про жука на окне, который казался Ктулху на горе.

Бывали случаи, когда человек рождался слепым, а потом в результате операции начинал видеть. Такой человек не мог ориентироваться с открытыми глазами! С закрытыми мог. Очень трудно подгонять свой мир к новым ощущениям. Так формируется мир для нас под влиянием опыта.

Итак, порядок есть в объективном мире, и в нашем сознании он возникает под влиянием его в реальности. А как существует время и пространство? Они сами по себе нематериальны, не существуют. Объектальность!

    Субстанциальная концепция времени и пространства. С этой точки зрения и время и пространство существуют самостоятельно в отрыве от материи, как контейнер существует независимо от содержимого.

    Реляционная концепция с ее точки зрения, время и пространство существуют только как порядок материальных тел и событий. Т.е. они не самостоятельные сущности, не субстанции. Возникла как зачатки у Аристотеля, Декарт четко произнес, Лейбниц, Гегель, окончательно утвердилось в науке после возникновения теории относительности. Как сказал старик Альберт, когда к нему пристал журналист с вопросом <<А как в двух словах объяснить для ламера, что самого классного в Вашей теории?>>, <<Я открыл, что если из мира исчезнет материя, с ней же исчезнет время и пространство!!>>

Сайт создан с помощью CSS фреймворка Amazium
