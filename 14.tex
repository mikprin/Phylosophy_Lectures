
%Лекции по философии науки проф. Семенова Юрия Ивановича

\section{Возникновение марксизма}
Лекция 14, семестр 9, 3.12.2005
\subsection{Кризис материализма на рубеже XVIII–XIX вв}

Материализм проявил ряд слабостей. Прежде всего, он не смог решить проблемы теории познания и запутался в неразрешимых противоречиях. Отрицал то, что было очевидно и не мог дать материалистическое описание тех или иных явлений познания. А, следовательно, был материализмом незавершенным, недостойным, материализмом только во взглядах на природу, но идеализмом, когда обращался к проблемам общественной жизни. Не мог обнаружить общественного бытия, не мог найти социальную материю. Он не мог разобраться в проблемах теории познания. И вот материалистическое учение, которое смогло справиться со всеми этими проблемами, дать ответ на все эти вопросы, было новым материализмом, который был создан двумя выдающимися мыслителями Карлом Марксом и Фридрихом Энгельсом.
\subsection{Предпосылки возникновения марксистской философии}

Марксистскую философию можно назвать марксистским материализмом. Марксистский материализм возник как часть марксистского учения. Это учение состоит из трех частей. Первая часть марксизма - марксистская философия. Именно эта часть нас с вами и интересует. Она является определенным философским учением, концепцией. Вторая часть - марксистская политическая экономика. Относится к другой науке, науке экономической. Третья часть - по современной терминологии - марксистская политология, научный социализм (ну научный коммунизм) - учение о том, как нужно действовать рабочему классу в современных условиях и какова задача рабочего движения. Все эти три части возникли вместе. С чем связано возникновение марксизма? Это не просто работа двух философов. Марксизм возник как ответ на те проблемы, которые поставила сама жизнь. В первой половине XIX века. в западной Европе окончательно утверждается капитализм. Во второй половине XIX веке во всей Европе происходят так называемые промышленные революции. Тем самым, для капитализма возникла адекватная технологическая, промышленная база. До этого капитализм был мануфактурным и основывался на ручном труде. Выдвинулись две противоборствующие силы: с одной стороны буржуазия, а с другой рабочий класс. Начало XIX века выделялась ужасающим обнищанием народных масс. Люди, которые в свое время штурмовали Бастилию, практически ничего не получили. Вместо братства, равенства и свободы пришло господство капитализма. Было ужасающее положение рабочего класса, рабочий день доходил до 16 часов в сутки. Крайне низкая зарплата и ужасающие жилищные условия вызывали возмущение. Это выливалось в классовую борьбу. В начале это принимало дикие формы. Например, движение лудитов в Англии. Они считали, что во всех их бедствиях виноваты машины. Поэтому они ломали станки и поджигали здания фабрик. Были введены соответствующие законы, и людей уличенных подобных действиях, вешали. Затем стали вспыхивать вооруженные восстания рабочих. Самое знаменитое восстание 1831 и 1834 года во Франции в городе Леон. Восстание леонских ткачей. Они захватывали город, устанавливали свою власть, но дальше не знали что делать. Также в Англии развернулось чартистское движение. Это движение началось в 30-м году, достигло развития в 40-х и заглохло в 50-е. Они исходили из того, что рабочих большинство в стране и они имеют право получить место в парламенте. Требовали, чтобы все взрослые мужчины старше 21 года могли участвовать в выборах. Они считали, что тогда изберут парламент из рабочих, а парламент изменит общественный строй. Они требовали принятие народной партии. Собирали 3-4 миллиона подписей, но парламент им отказывал. Они действовали по разному, проводили грандиозные демонстрации (собирали до 300тыс. человек), были и бои, восстания, вот такое мощное движение. Стали возникать организации чартистов. В 1836 возникла лондонская организация, потом была создана Национальная ассоциация чартистов - первая партия рабочего класса (50тыс. человек). В 44-м году в Германии вспыхнуло знамение восстание каких-то там ткачей. Таким образом, рабочие могли требовать повышение зарплаты и уменьшение рабочего дня, а что дальше в целом-то? Т.е. поступил социальный заказ - нужно было создать идеологию рабочего класса. Учение, которое бы выражало интересы рабочего класса, указывало, что нужно рабочему классу делать, какие у него объективные задачи. Именно ответом на эти вопросы и появился марксизм. Т.е. марксизм не мог не возникнуть. Причем сразу три его части. Теперь немного об основоположниках этого учения.
\subsection{Карл Генрих Маркс (1818-1883) и Фридрих Энгельс (1820-1895)}

Маркс родился в городе Трире (прирейнская Пруссия). Это был самый передовой район Пруссии и Германии. Родился в семье адвоката, советника юстиции. Отец был раввином, но став советником юстиции перешел в христианство. Карл поступил в гимназию, потом в Боннский университет, через год перебрался в Берлинский университет. Там он тесно связан с группой гегельянцев (объективные идеалисты). Активно участвовал в оппозиционной борьбе. В результате ему пришлось защищать докторскую диссертацию по философии в другом университете. Называлась диссертация: “О развитие между философскими системами Демокрита и Эпикура”, защитил в 1841 году. Т.е. в то время он был идеалистом. Он мечтал о карьере академического направления. Но ему судьба не улыбалась, т.к. он прославился своими оппозиционными выступлениями. Он занимался публицистикой. Был редактором оппозиционной газеты.

Энгельс родился в семье крупного, богатого фабриканта. В городе Бармене той же прирейнской провинции. Он учился в гимназии, но отец не дал ему его доучиться, т.к. с детских лет готовил его занять свое место. Его призвали на военную служу и определили в Берлин. Он 2 года пробыл в Берлине и получил возможность посещать занятия в университете. Там он сблизился с гегельянцами и активно принял участие в их деятельности. Затем он поехал в Англию поучиться на коммерческом факультете, где он нагляделся на все прелести жизни рабочего класса. В 1843 году стал склоняться в сторону материализма. Стал переходить на сторону рабочего класса, коммунизма. Начал отвлекаться от философии Гегеля и на этот поворот повлияла книга Фейербаха. В 1844 году Маркс и Энгельс встречаются и выясняют, что у них схожие взгляды на многие вопросы. С этих пор начинается прочная дружба и совместная деятельность. Совместные работы "Святое семейство" и "Немецкая идеология", где они четко формулируют принципы материалистического понимания истории. В Германии произошла революция 48-49 годов, которая объяла всю страну. В это время Маркс становится редактором "Новой рейнской газеты", которая выступала за республику, конституцию, свержение всех монархий. Маркса хотели отдать под суд, и ему пришлось бежать. Что касается Энгельса, он принимал участие не только в газетной борьбе, но и в вооруженных восстаниях. После поражения одного из таких восстаний, вынужден был покинуть страну. Таким образом, они оказались сначала во Франции, а потом перебрались в Англию.
\subsection{Возникновение философии марксизма}

43-48гг - время формирования марксизма. Самая знаменитая совместная работа "Манифест политической партии" Была такая международная рабочая организация "Союз коммунистов", которая ставила целью помочь рабочему классу в борьбе за социалистическую революцию. Маркс и Энгельс к ним присоединились и написали программу партии. Это была работа уже зрелого марксизма. Это не значит, что развитие марксизма завершилось. Нет, Маркс и Энгельс работали дальше, но это было уже развитие созревшего марксизма. Они обратили внимание на учения социалистические, но ранее этот социализм носил риторический характер. Это были всего лишь мечтания о социализме. Маркс и Энгельс решили создать научный социализм. Для этого нужно было понять законы развития общества. Нужно было создать материалистическое понимание истории. Забегая вперед можно сказать, что основа общества - экономика. Поэтому для того, чтобы понять куда идет развитие общество и на этой основе строить программу и тактику рабочего класса, нужно детально изучить экономику, поэтому Маркс написал “Капитал”.
\subsection{Французские историки эпохи Реставрации}

Основное положение в учении Гегеля - история есть объективный процесс, происходящий независимо от деятельности отдельных людей. Но он не смог нащупать механизмы, приводящие к движению истории. Но к ответу на этот вопрос пришли мыслители из других стран. Эти людей называю французскими историками эпохи реставрации. К ним относятся 4 человека. Все они родились в конце XVIII века и прожили всю первую половину XIX века. Огюстен Тьерри Франсуа Гизо, историк и государственный деятель.
Франсуа Минье Четвертого у нас не очень любили упоминать. Если вы хоть немного помните историю, то помните, что была такая Парижская коммуна. Она потом была задавлена силами версальского режима. Глава этого режима - Адольф Тьер. Его назвали палачом Парижской коммуны. Именно он и был одним из четырех выдающихся историков. В чем их заслуга? Если сказать коротко, то они открыли общественные классы и классовую борьбу. Т.е. эти понятия были открыты еще до Карла и Энгельса. Что такое эпоха реставрации? Почему она так называется? Были реставрированы новые порядки после революции. Наполеон вернулся во Францию, снова захватил власть. К власти пришло снова дворянство, которое вынуждено было считаться с интересами буржуазии, но сама буржуазия до власти не допускалась. И началась классовая борьба между дворянством и буржуазией. В 1789 году началась революция, в результате которой власть перешла в руки буржуазии. Франция раскололась на два лагеря, между которыми велась вооруженная борьба. Возник вопрос: А эти два лагеря возникли в результате революции? Тут историки легко поняли, что вся история Франции с момента возникновения феодализма была историей борьбы классов - дворянства и крестьянства.

Вся земля во Франции принадлежала дворянам. Цель крестьян была освободить землю и получить ее в собственность. Выяснилось, что эти классы имеют свои объективные интересы, не зависящие от конкретных людей. Где же корни этих интересов? Ясно, что они не биологические. Корни этих интересов в системе отношений к собственности, к системе имущественных отношений. Но это не только отношение к земле, но и к любому виду имущества вообще. Хорошо, существую классы, существует система социально-экономических отношений. Почему такая, а не иная? Почему возникли именно эти классы? Историки эпохи реставрации решить эту проблему так и не сумели. Существуют два вида имущественных отношений: Первый вид, очевидный, бросающийся в глаза - волевые или политические отношения собственности, закрепляются законом (как заключаются сделки и перевод имущества...); 2 - экономические отношения собственности проявляются в распределение и обмене. Французские историки эпохи реставрации видели только волевые отношения собственности.
\subsection{Открытие экономических отношений и возникновение политической экономии.}

Итак, существуют два вида отношений собственности, из которых французские историки знали только один - правовые или волевые. Они зависят от воли и сознания людей. Отношения экономические были всегда, на всех стадиях развития экономического общества. Но их впервые открыли только в XVII-XVIII веке. Именно в эти годы и возникает наука об экономических отношениях, которая называется политическая экономия. Она изучает экономику не домашнего хозяйства, а страны. Политэкономия возникла как наука не вообще об экономических отношениях, а как наука о капиталистических экономических отношениях. При капитализме возникают такие отношения, которые начинаю определять волю людей прямо непосредственно. И мораль и право, т.е. существует прямая детерминация экономическими отношениями воли и сознания людей. Капиталистическое общество - это общество промышленное. Каждый крестьянин может жить самостоятельно и независимо. А когда возникает промышленность, то возникают промышленные предприятия с высокой степенью специализации. Т.е. каждый производитель должен поддерживать множество связей с другими предприятиями. Т.е. необходимо для промышленного общество непрерывное циркуляция материальных благ между всеми ячейками производства. Этот обмен можно обеспечить методом централизованного государственного управления либо рыночным методом. Рынок диктует людям как действовать. Открытые экономические отношения не зависят от сознания людей, а являются объективными. Политэкономия, исследовав рыночную экономику, открыла первый в истории человечества общественный закон. Было ясно, что экономические отношения - этап развития общества. Это ясно было видно в работах Адама Смита. В своих работах он показывал, что отношения капиталистические это не просто отношения обмена, а отношения распределения. Экономисты открыли, что, существуют отношения, независящие от воли и сознания людей. И вот эти отношения диктуют людям как поступать не только в области экономической деятельности, но и за ее пределами. Откуда эти капиталистические отношения взялись? На этот вопрос отвечал Адам Смит. Он считал, что никаких экономических отношений, кроме капиталистических не существует. Они вытекают из природы человека. Но есть страны, в которых нет капиталистических отношений. Ответ - это результат насилия. Т.е. капиталистические отношения - естественные, а все остальные искусственные.

В XVIII-м веке появлялось много работ посвященных античности, и делался вывод, что античное общество базировалось на рабстве. Феодальное общество базируется на труде крепостных крестьян. Затем появилось капиталистическое общество, основанное на труде рабочих. Франсуа Бернье (1625-1688) Был философом, человеком широких взглядов. Путешествовал по странам востока, был в Египте, в Турции, попал в Индию в разгар гражданской войны за власть. Пробыл там 12 лет. Высказал мысль, что общественно-экономический строй восточных стран коренным образом отличается от европейских порядком. Специфика заключается в том, что в Европе правят деспотические монархии, общество делилось на группы людей, и там не было частной собственности.
\subsection{Философия истории А. Сен-Симона}

Анри Сен-Симон (1760-1825), утопический социалист. Он своеобразную систему философской истории. Он знал, что было общество первобытное, есть общество древнего востока, но не принимал за общество самостоятельное и присоединял к античному. Существует три эпохи: 1 - античное общество, основанное на рабстве, 2- средневековое общество (феодальное), основанное на крепостном праве; 3 – индустриальное, основанное на работе наемных рабочих. Смена эпох связана со сменой общественно - экономической системы. В античной эпохе было рабство - это своего рода прогресс, т.к. пленников раньше убивали, а теперь стали брать в рабство, а в последствии они могли освободиться. В феодальную эпоху, основанную на работе крепостных крестьян, возник тоже большой прогресс по сравнению с рабством. В индустриальном обществе рабочий сам решает к кому наняться на работу. Существует три формы порабощения человека человеком. Что лежит в основе смен этих эпох - разум, но разум не абсолютный и не разум отдельный людей, а разум человеческий - есть такой общечеловеческий разум. Разум развивается, приходит в противоречие с существующими порядками, затем наступают новые порядки. Разум - движущая сила истории. У Сен-Симона были догадки о роли классов и классовой борьбы, после его смерти появились такие экономисты, которые начали исследовать докапиталистические системы и сыграли немалую роль в развитии общественной мысли и славили, кстати, марксизм, таким был Ричард Джонс.
\subsection{Р. Джонс и его вклад в развитие экономической науки.}

Ричард Джонс (1790-1853) Во многом забытый. Его заслуга была в том, что вопреки Адаму Смиту и другим экономистам пришел к выводу, что капиталистическая система - это не единственная, а одна из немногих экономических систем. И все они тоже естественные. Существуют разные способы производства и распределения. Их много, и капитализм - только один их таких способов. Джонс был человеком консервативным. Считал, что капитализм - приходящая стадия экономического развития. Существуют системы экономических отношений, и эта система является фундаментом каждого конкретного общества. И определяет взгляды и представления людей, общественный строй. Для того чтобы понять природу любого общества нужно изучить систему экономических отношений. Джонс не мог ответит на вопрос: сколько таких систем и как они сменяли друг друга. Ответ на эти вопросы дал Карл Маркс.
\subsection{Открытие К. Марксом и Ф. Энгельсом социальной материи и создание ими материалистического понимания истории}

Карл Маркс был великим экономистом. Прекрасно знал экономику современной Европы. Понимал, что есть определенная экономическая система, центром ее является наемный рабочий. Новое время - время господства такого способа производства. Раньше наемны рабочих не было, но производство все равно существовало, следовательно, существует разные формы производства. Возникает вопрос, а как быть с древним востоком? Он сказал, что на востоке особый азиатский способ производства. Общественное производство всегда было и является основа жизни людей. Стоит остановиться процессу общественного производства, как люди неизбежно погибнут. Ввел понятие способ производства – производство, взятое в определенной общественной форме. Перед нам три способа производства - античный, феодальный, и капиталистический. Существуют и другие, но ими еще не занимаются. Там, где возникает машинное производство, там рушатся старые способы производства. Результат - отмена крепостного права. М и Э вводят понятие производительной силы. Эпохи отличаются не только тем, в какой общественной форме происходит производство, но и кто и как создает... Работники - производительная сила общества. В разные эпохи производительные силы общества были разными. Растет производительная сила. Чем определяется характер экономических отношений? Уровнем развития производительности. Производительная сила определяет характер общественно-экономических отношений. Возникают такие отношения, которые дают простор для развития производительных сил. Производительная сила уходит вперед и отношения экономические, которые были двигателем, начинают мешать развитию производства, возникает конфликт. И единственный выход - смена экономических отношений другими, которые дают простор для развития производительных сил. Т.е. Маркс выявил, откуда берутся социально-экономические отношения, которые являются объективными, не зависящими от воли людей, а зависят только от уровня развития производительных сил общества. Все отношения людей были разделены на две части - отношения социально экономические, независящие от сознания, и все остальные, волевые. Была создана социальная материя и материалистическое понимание истории.
