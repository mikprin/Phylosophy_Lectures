
\section{Экономические отношения}
\subsection{Введение}

Мы уже рассмотрели, как возникла материалистическая картина истории, как человеческая мысль искала объективный источник общественных идей, т.е. социальную материю. Основные этапы этого – философия Гегеля, концепция французских историков эпохи Реставрации, открытие того, что отношения собственности двух видов(отношения волевые, которые в цивилизованном обществе приобретают форму юридических и правовых отношений, которые диктуются волей людей, и экономические, которые не зависят от воли и сознания людей). Экономические отношения были открыты в 16-17вв. В 18в, когда сложилась политическая экономия, стало ясно, что есть экономические отношения, которые давят на людей, что существуют от них независимо, определяют волю и сознание людей, что эта система объективна и в ней действуют объективные законы. Экономические отношения были окрыты как капиталистические отношения, а там действовало экономическое принуждение. Возникал ворос – откуда же возникают экономические отношения?

Обнаружилось, что существуют экономические системы разных типов, все эти системы объективны и что капиталистическая – одна из них. Возникал вопрос: почему существует та или иная систему, почему происходит их смена и тем самым смена типов общества, тем самым смена мировых эпох. Ответ на этот вопрос смогли дать только Карл Маркс и Фридрих Энгельс.
Экономические отношения по Марксу и Энгельсу

Обнаружили, что основой жизни является производство, без него люди жить не могут. Это одно из отличий человека и животных – люди создают то, чего в природе нет. Производство – основа бытия людей, существа людей. Стоит прекратиться процессу производства и человечество исчезнет. Оно было всегда. На всех стадиях развития человечества, всегда происходило в определенной общественной форме, которая и есть системой социально-экономических отношений, или социально-экономическим укладом. Когда мы берем производство в конкретной определенной форме, то это называется способом производства. И вся человеческая история – это развитие производства и развитие смены способов производства. Разные эпохи отличаются не только общественной формой производства, но и уровнем развития производства. Производительность в рамках одной системы отношений развивается, перерастает эти рамки и требует новых социально-экономических отношений. Происходит смена социально-экономических отношений, способов производства и коренное изменение самого общества.

Тем самым стало ясно, что социально-экономические – это отношения объективные, определяющие волю и сознание людей, но от воли и сознания не зависящие. Так было открыта социальная материя, создано материалистическое понимание истории.

Человечество никогда не было одним обществом, было разделено на отдельные общества. Историки как раз и занимались типологией социально-исторических организаций. Использовались разные признаки: деление на республики и монархии, по политическому режиму(авторитарные и демократические общества), по конфессиям(религия). Но это не главные признаки. С точки зрения материалистической теории в основе лежит социально-экономическая структура общества. Типы общества, разделенные по этому признаку, получили название общественно-экономических формации(тип общества, выделенный по признаку социально-экономической структуры). Каждый такой тип представляет собой более высокую ступень развития. Все способы производства не равноценны, а относятся как высшие и низшие. Каждая формация есть не только тип общества, но и стадия экономического развития. Марксистская теория формации – одна из разновидностей унитарно-стадиального понимания истории, в основе которой положено развитие общественного производства.
Стадии развития человечества

У Маркса было выделено несколько эпох: азиатская(эпоха Древнего Востока), эпоха Античности, эпоха Средневековья и Новая эпоха. Для эпохи Нового Времени характерны капиталистические отношения, Средние Века – феодальные общественно-экономические формации, Античность – рабовладельческие формации, Древний Восток – азиатские(условно, т.к. ни одна из выше перечисленных, особый тип производства). Древнему Востоку предшествовал первобытный коммунизм(дикие варварские цивилизации, первобытное общинное общество).

После капиталистической формации по Марксу и Энгельсу следует коммунистический способ формации. Вывод был сделан на основе анализа экономики капитализма.
Диалектический материализм

Когда была открыта социальная материя, был разорван порочный круг и стало ясно как развивается человеческое общество. Марксом и Энгельсом материалистическое решение основного вопроса философии было распространено и на область общественных отношений. Так что диалектический материализм(такое название получила эта философия) был материализмом историческим. Надо сказать, что создание материалистического понимания истории не просто привело к тому, что был достроен материализм, а к тому, что он был заново перестроен. Открытие материалистического понимания истории дало возможность преодолеть слабые стороны материализма и его противоречия. Создание материалистического понимания истории дало возможность понять и раскрыть причины активности сознания. До этого материалисты не могли понять причины и шли по пути отрицания, когда занимались миром в целом, но признавали, когда занимались к обществу. Есть два вида человеческой активности: идеальная (активность мышления и сознания, духовная) и материальная. Все склонялись к тому, что первична активность идеальная. Любой практической материальной активности предшествует активность духовная. Даже самые убежденные материалисты к этому склонялись.

Пример: Николай Гаврилович Чернышевский, выдающийся русский мыслитель, который старался быть материалистом, был последователем Фейербаха, пытался использовать диалектику Гегеля. На вопрос, как возникла способность человека к производству отвечал так: по-видимому стадо обезьян попало в благоприятные условия, где было мало врагов и много пищи, поумнели и пришли к выводу, что можно и работать. Т.е. в начале поумнели, а потом начали создавать производства.

Среди марксистских мыслителей эта мысль преобладала почти у всех. Догадка о том, что не мысли порождают дело, а дело порождает мысли, была пожалуй только у знаменитого немецкого философа, естествоиспытателя и поэта Гете(<<Фауст>>). <<И говорю я смело - вначале было дело>>. Догадка гениальна. Когда Маркс и Энгельс открыли, что основой развития человечества – производства, им стало ясно, что именно производственная деятельность и есть основа человека и что именно ее развитие вызвало к жизни мышление. Отсюда и возникла идея, что именно производство породило человека. И Энгельс еще в 1876 году написал небольшую статью <<Роль труда в процессе превращения обезьяны в человека>>, которая 20 лет пролежала у него в архиве. Была опубликована в 1896 году. В ней он четко сказал, что возникновение производственной деятельности породило мышление и породило человека. Эту работу мало кто знал из антропологов. Выяснилось, что человек есть только там, где есть производство. Там где его нет – там обезьяны.

Первыми существами, о которых достоверно известно, что они ходили на задних лапах были австралопитеки(южные обезьяны). У них были свободны верхние конечности, и они систематически использовали камни и палки для борьбы с хищниками. Но это не люди, т.к. производством не занимались. Их часто называют предлюдьми. 2,5 миллиона лет назад возникли существа, которые изготовляли орудия труда. Их называли homo habilis(человек умелый). Их не считали людьми, поскольку физиологическими они были животными. И эти орудия труда все были разными. У них не было мышления и языка. 1,8 млн.лет назад появились существа, которых назвали Homo Erectus(человек прямоходящий), которого уже морфологически походил на человека, у них зарождалось мышление и язык. В это время появляется первое типовое орудие – ручное рубило, которое было одинаково везде. Человек начинает изготовлять орудия при помощи орудий и это открывает возможность развития. А 40 тыс. лет назад возникли homo sapiens. Сформировалось мышление, речь. Следующим шагом было изготовление вещи, которая сама по себе не нужна, но из которой можно изготовить ценное орудие. Для того, чтобы успешно развиваться производству, необходимо создавать образы вещей, которых еще нет, а для этого необходимо знание общего, знание существенных необходимых связей. Люди стали фантазировать – воображать связи, которых нет в чувствах. А потом проверять. И если понятия соответсвуют реальному миру, то это дает новые знания о мире, и идеальное превращается в материальное. Не только мир действует на сознание, но и сознание действует на мир. Это внес диалектический материализм, показал как общее в сознание. Возникает проблема – нет ничего в разуме, чего не было бы в чувствах? И верно и неверно. Если оно верно, то мышление ничего нового не дает. А если оно неверно, то есть какой-то канал из сознания в мир? А такого канала нет и быть не может.

Свобода есть господство человека над миром, основанное на двух моментах – знание необходимости(т.е. законов), и наличие материальных средств, которые позволяют изменить мир в нужную сторону. А от абсолютного детерминизма позволяет избавиться положение Гегеля – случайность случайна, но она необходима, а необходимость необходима, но она случайна. И поэтому зная необходимость, мы подчиняем себе случайности, направляем их в нужную для нас сторону. С одной стороны понятие свободы исключает понятие необходимости. В каком случае возможна свобода(выбор,умение разобраться, какое из массы действий ведет к результату)? Во-первых это знание внутренних связей, когда можно предвидеть результат, и во-вторых наличие средств, с помощью которых можно добиться цели. Нет абсолютной свободы, есть только относительная, а этой свободы нет без необходимости, т.к. если бы был только хаос случайностей либо только необходимость – это была бы не свобода. Когда есть необходимость, которая при этом случайна, то можно видеть как будет развиваться ход событий, и какой будет результат.

Основой познания человека является практическая деятельность. Процесс развития бесконечен. А значит бесконечно развитие познания человека. А поэтому мышление – это объективный процесс. Т.е. у Гегеля была заимствована диалектика, диалектика мира и диалектика понятий, но была пересмотрена проблема отношения между ними. С точки зрения Гегеля диалектика понятий определяет диалектику вещей, а Маркс и Энгельс пришли к выводу, что и диалектика вещей определяет диалектику понятий. Это дало возможность решить проблему истины. Между миром и сознанием существует и соответствие и несоответствие. И поэтому на каждом этапе знания о мире относительны, неполны, а в процессе развития идет накопление знаний, и приближение к истине, которая никогда не будет достигнута. И идет процесс непрерывного сближения мира и сознания. Проблема суверенности человеческого сознания – можно ли познать весь мир? Да – это значит что когда-то весь мир познают, и все. Нет – это значит, что когда-нибудь мы придем к пределу, что существуют вещи непознаваемые. Правильный ответ – и да и нет. Т.е. мир полность познаваем, но никогда не будет полностью познан.
Развитие западной философии после возникновения марксистского материализма.

Историки выделяют два этапа в развитии философии новейшего времени выделяют 2 основных этапа.

    начало 17в – середина 19в. Классическая западно-европейская философия.
    после середины 19в – неклассическая или постклассическая эпоха. Между этими двумя этапами существует коренное отличие – этап быстрого прогресса философского знания. Эпоха постановки новых проблем и новых открытий. Вторая же эпоха характеризовалась топтанием на месте, повторением старых систем. Однако новые проблемы все же возникали. Одной из таких проблем была проблема соотношения эмпирического и теоретического знания. Она была замечена, но не решена. В это время ни одна проблема не была решена.

Это связано со следующими причинами: возникла марксистская философия, которая ответила на множество вопросов. Чтобы двигаться дальше, необходимо было усвоить достижения марксистской философии, ведь философия – наука кумулятивная, чтобы двигаться дальше, необходимо познать то что было. А этого не происходило главным образом по идеологическим причинам, ведь марксизм – философия революции. Даже философии Гегеля приписывали то, что он создает революционную теорию. Поэтому марксизм нигде не изучали. А значит философия прекратила развитие. Возникали учения типа неокантианство, неогегельянство итп, просто новые прочтения гегеля, канта с добавлением современных элементов. Неотомизм. Была попытка воскресить Спинозу. Возникло учение под названием <<позитивизм>>, которое просто является обновлением Юма. Но была одна система, которая до этого не существовала – иррационализм.
