

\section{Философия нового времени (продолжение)}



Прошлый раз мы остановились на рассмотрении двух основоположников (двух основных линий) философии нового времени: Френсиса Бекона (английский философ, основоположник эмпиризма) и Рене Декарта (французский философ, основоположник рационализма). А на этой лекции мы рассмотрим их последователей.
Томас Гоббс (1588 - 1679)

Хотя он и старше чем Декарт (но надолго пережил его), но мы рассматриваем его как его продолжателя. Гоббс много полемизировал с Декартом (нельзя понять философию Гоппса, не зная его полемики с Декартом).
Историческая справка. (отступление)

Основная часть жизни Гоббса приходится на удивительно сложное бурное время: на английскую буржуазную революцию. После Якова I (VI) на трон взошел его сын Карл IX (I), который пытался править самодержавно (даже не собирал парламент). В конце концов, страна была разорена. Разразился конфликт между королем и парламентом (который впервые за период царствования Карла I был созван лишь в 1640г., для получения денег на новую военную кампанию), в результате которого Карл покинул Лондон, собрал армию и объявил войну. Началась первая гражданская война (1641-1646гг.) В начале успех был на стороне королевских войск (потому что у него были наемные солдаты и офицеры, имевшие боевую выучку и стремившиеся к компромиссу с королем). В конце концов, армия парламента была организована, во главе которой стали выдающиеся полководцы (Оливер Кромвель). В результате король был разбит, попал в плен, был приговорен к смертной казни и казнён в 1641 году. И в Англии была провозглашена республика. На смену республике пришло авторитарное правление Оливера Кромвеля. После смерти Кромвеля наступил раздор в республиканском лагере. В 1660г. в Англию вернулись Стюарты и на престол взошел Карл II, сын казненного короля (реставрация монархии). Была произведена реставрация старого стоя, но парламент действовал постоянно и контролировал короля. И Карлу Второму приходилось с этим считаться. Взошедший после смерти Карла на престол Яков II из династии Стюартов, попытался восстановить абсолютизм и его опору - католическую церковь, что вызвало большие протесты среди англичан. И в 1682 году произошло восстание герцога Монмаунта (незаконнорожденного сына Карла II) против короля Якова II. Восстание было крайне жестоко подавлено, что вызвало большое недовольство среди англичан. И в ходе государственного переворота в 1688-89 (т.н. Славная революция) на английский престол был призван нидерландский король Вильгельм III, который после победы над королем Яковом II правил Англией до 1694 года совместно с женой Марией II Стюарт. Государственный переворот 1688-89 закрепил доступ буржуазии к государственной власти. И с 1689 года в Англии воцарилась конституционная монархия.
Биография

Родился в семье сельского священника, поступил в школу, затем в Оксфордский университет. К этому времени университеты стали совсем иными (чем когда они возникали). Тогда университеты были оплотом мысли, а теперь они стали оплотом консерватизма и схоластики. Поэтому тогда крупные ученые в них уже не работали, возникали конфликты. Получив степень магистра, Гоббс не стал преподавать, а попытался получить настоящее научное образование. Так как он был не слишком обеспеченным, то позволить себе путешествия не мог. Поэтому он нанимался преподавателем в семьи обеспеченных людей и вместе с ними путешествовал. Во время подобных путешествий он побывал во Франции, в Германии, в Италии, где он знакомился с крупнейшими представителями науки. Так он познакомился с трудами Эвклида, которыми в Англии пренебрегали. Навсегда сохранил уважение к математике, считая её царицей наук. Так же он познакомился с Беконом, много раз с ним беседовал. Таким образом он путём самообразования приобрёл гигантские знания. Дальше он начал заниматься философией.

    Работы: 1642г. <<О гражданине>>. Где он высказал критику применения силы и поддержал позицию короля и в результате должен был покинуть Англию. Когда Карл I был обезглавлен, Гоббс стал проявлять лояльность к новому правительству, доказывая необходимость применения силы, тем самым привлек симпатию республиканцев и получил возможность вернуться в Англию. 1651г. <<Левиафан>> – его главный труд (был написан до возвращения на Родину). Левиафан – сказочное, огромное библейское чудовище, которое пожирало людей и корабли. Под этим чудищем он понимал государство. В этой работе он говорит о необходимости сильного государства, во главе которого стоит человек обладающий всей полнотой власти, и он имел ввиду Оливера Кромвеля. Так же примирился с провозглашенной в Англии республикой, что вызывало недовольство в кругу эмигрантов и Гоббс вынужден уехать из Голландии в Англию.

Когда произошла реставрация, то он попытался не ссориться с властью. Сам Карл относился к нему благосклонно, но его окружение начало травлю. Постепенно все его работы были запрещены. После смерти книгу (Левиафан) торжественно сожгли во дворе университета.
Философия

Его философскую систему выражает трилогия: "О теле", "О человеке" и "О гражданине".

    Он был восхищен успехами науки и техники. В этом почти смыклся с Беконом, писал почти тоже самое что и он. Только научные, объективные, полноценные знания дают возможность подчинить себе природу и поэтому важнейшая задача – это развитие науки, а наука является плодом разума. Сам Гоббс был эмпириком и считал, что нужно исходить из опыта, но, тем не менее, сам по себе опыт знания не дает. Знания получаются только, когда все опытные данные обработаны разумом. Научный разум и только он дает истину. И никакой другой истины, кроме истины разума, кроме истины науки нет и быть не может (никакой истины откровения нет). Доказывал, что теология - не наука. Человеческий разум дает истину, но а для того, чтобы он давал истину (ведь мышление у всех людей разное) нужна новая логика, новый метод, так как логика Аристотеля безнадёжно устарела. Для получения истины мышлением нужно управлять.

Большое внимание уделяет, как и Бекон, причинным связям. Тут возможен двоякий путь. Первый путь: движение от следствия к причине, а другой путь: движения от причины к следствию. Нужно познание закономерных связий природы.

Практически он сводил познание к математическому познанию (уделял огромное значение математике). Разум должен заниматься вычислением (складыванием и вычитанием), а складывать и вычитать можно не только цифры, но и углы, вещи, законы… И в результате этого разношерстные знания, получаемые из опыта, превращаются в строгую законченную систему.

Учения о метках и знаках (родоначальник семиотики). Гоббс развивает свою теорию мышления. > Существует бесконечно много мыслей и их нужно уметь направлять. Для этого необходимы опоры - <<метки>>. <<Метки>> это некие материальные объекты, наводящие нас на определенные мысли (например, грозовые тучи - это <<метки>> дождя). Для <<меток>> характерна важная особенность, а именно, они субъективны. Но при общении мы тоже начинаем использовать эти метки и тогда они становятся общезначимыми. И тогда метки становятся знаками. Распространенная форма знаков – это слова (язык является общезначимым). Ставит вопрос о соотношении понятий и знаков – проблема общего и отдельного. Понятие и слово имеют нечто общее и являются продуктом нашего мышления (существуют только в нашем мышлении). Был номиналистом, а именно сторонником того направления, которое получило название концептуализм.

Мы оперируем знаками и словами, но сами по себе знаки и слова истину не дают. А для получения истины, мы соединяем знаки в суждения. (Пример: Кошка – черная). Проблема истины и заблуждения становится тогда, когда мы имеем дело с суждениями. > Истина – это характеристика наших суждений.

В мире есть материя. Мир – это есть совокупность тел. В мире существует телесная субстанция и эта телесная субстанция – единственная существующая. А что такое дух? А это и есть проявление той же самой телесной субстанции (<<Телесная субстанция способна мыслить, а мысль – это свойство телесной субстанции>>). Эта субстанция находится в движении, причем конкретные вещи возникают при взаимодействии. А телесная субстанция – вечна и заполняет весь мир. Таким образом, Гоббс высказывает сугубо материалистическую точку зрения.

В отличие от Бекона, Гоббс сводит все движение материи к перемещению тел, просто к механическому движению. Говоря о развитии мира, придает огромное значение изучению причины и следствия. С его точки зрения, нет явлений беспричинных и нет явлений, вызванных неестественными причинами! Случайны только те следствия, причины которых мы не знаем, так что случайность – явление субъективное. Иначе говоря, Гоббс был сторонником абсолютного детерминизма. Впервые идея детерминизма выдвинута Демокритом, а вот теперь эта идея возрождается в философии нового времени. Учение Гоббса об абсолютном детерминизме основывается на механике (науке того времени). Все подчиняется законам и отсюда никакой свободы не существует. > Свобода – это просто обман нашего самосознания.

Пытался разработать концепцию ощущений. Есть предметы внешнего мира, они воздействуют на рецепторный прибор человека, идут нервные процессы, дальше эти процессы приходят в мозг человека и тогда человек видит перед собой тот предмет, который действовал на него, у него возникает призрак предмета. То есть мы видим Солнце, но то Солнце, которое мы видим – это не то Солнце, которое существует: мы видим кружок, но на самом деле это огромная звезда. Таким образом, есть ощущения, которые нам передают свойства предмета, но есть и такие, которые являются также результатом воздействия, но существуют только в нашем сознании. > C его точки зрения цвета и звуки возникают и существуют только в нашем сознании. А мир в нашем сознании получается с помощью синтеза: дедуктивно из простых вещей мы строим окружающий мир (например, движение точки = прямая, а плоскость = движение прямой). Мир - результат нашей деятельности. И получается он путём дедукции, путём составления вещей из элементов не имеющих качеств, из геометрических точек.

Науки бывают как дедуктивные (геометрия, наука об обществе), так и индуктивные (физика). Гоббс чувствует, что эмпиризма не достаточно, что он чем-то пренебрегает, что чувств не достаточно и уходит к рационализму. Но у него эмпиризм и рационализм не сливаются в единое целое, а существуют параллельно. Поэтому как видно, Гоббс берет кое-что у Бекона, но и заимствует кое-то у Декарта, пытается приблизить то и другое, но синтеза у него, конечно же, не получается.

Рассматривая естествознание, он показывает, что опытного познания не достаточно (все зависит от опыта). А вот что касается геометрии, тут, как говорится, добавить нечего. Поэтому геометрия дает абсолютную истину, физика дает истину во многом случайную, где не докажешь что все должно быть так, а не иначе, мы просто это знаем, а в геометрии ничего не может быть иначе, может быть только так. В этом смысле геометрия выше физики.

И вот, к числу дедуктивных наук он относит и учение об обществе и пытается, исходя из элементарных вещей, вывести всю картину общественной жизни. Почти все философы говорят об общественной жизни, а вот у Гоббса есть своя концепция, которая сыграла важную роль и повлияла на дальнейшее развитие творческой мысли. С чего он начинает?

    Существуют два движения материи человека: одно движение к мозгу (движение по нервным клеткам, результат - познание), а второе – к сердцу (объясняют чувства). Так вот эти движения к сердцу могут находиться в различном отношении к процессам, происходящим в нашем теле. Есть такие движения к сердцу, которые способствуют жизненным процессам и поэтому они доставляют нам удовольствие и наслаждение. А есть такие движения, которые мешают жизненным процессам и в результате приводят к страданиям. Каждый человек стремится к тому, чтоб было приятно, и отвращается от неприятного. Здесь он вводит два понятия морали: добро и зло. То, что приятно – это добро, а то, что неприятно – это зло. Таким образом, понятия этики и морали он выводит прямо из физиологии.

Он не отличает людей от животных. В начале люди жили в состоянии, которое называющемся естественным: не было никаких законов. И поэтому каждый человек стремился к приятному и отстранялся от неприятного. Скажем, он хотел есть, ну он имел право взять все, что ему захочется (запретов же не было). В то время никакой собственности не было. И это приводило к бесконечным столкновениям (<<война всех против всех>>) и уничтожению рода человеческого.

Для решения этих конфликтов нужно было договориться, или иными словами создать государство. И возникает второе состояние – гражданское. И государство, прежде всего, вводит частную собственность. Каждый получает свою долю, которую другим трогать нельзя! Таким образом, прекращаются споры, конкуренции. Возникает могущественный аппарат, и люди все свои права передают этому аппарату. Ради чего? Да чтобы жить, потому что иначе без регулирования отношений выжить невозможно!

Кому они передают эту абсолютную власть?

    Здесь Гоббс выделяет три формы государства. Одна форма – республика, когда вся власть переходит парламенту. Вторая форма – это монархия, когда вся власть передается одному человеку. И, наконец, третья форма – это аристократия, когда выделяется круг людей, который в своих руках сосредоточивает власть. Какая же из этих форм лучше? По мнению Гоббса – монархия. А для этого монарх должен быть умным человеком, назначать хороших чиновников, заботиться о бедных и т.д. То есть он был приверженцем абсолютной власти.

Гоббс был атеистом, потому что он считал, что существует только одна единственная телесная субстанция. Но это не значит, что он был воинствующим атеистом, он, конечно, доказывал что религия – выдумка, но религия полезна, потому что народ нельзя держать в повиновении только с помощью армии и полиции (физической силы не достаточно). Народ нужно не только запугать, а убедить его (заковать его еще и духовно), что государство служит ему и поэтому протестовать не надо, а надо повиноваться.
Спиноза (1632 - 1677)

Спиноза - настоящее имя Барух Спиноза. Писал он на латыни, как многие философы того времени по-этому его назвали Бинедикт Спиноза. Родился в 1632 году в Амстердаме умер 1677 году. Произошел из купеческой еврейской семьи, его семья бежала из Португалии от инквизиции после реконкисты. Ислам был более терпимым к другим религиям, нежели католицизм. Если человек принимал ислам, то с него снимали часть налогов. В следствии терпимости на Пиренейском полуострове жило много евреев. После реконкисты многие евреи и мусульмане перешли в христианство. За ними следили, чтоб он не претворялись, а если обнаруживали, что они все-таки сохранили старую религию, их жгли на костре.

Спинозу отдали в еврейскую школу - хедер. Он изучал там Библию, Талмуд, древеневрейский язык. Семья Спинозы, как и большинство евреев из Испании говорила на спаньёле - смесь иврита и испанского. Евреи из центральной Европы говорили на идыш - смесь немецкого и иврита. На древеневрейском со ветхозаветных времен ни кто не говорит, однако современный иврит практически полностью идентичен древеневрейскому. Древеневрейский используется сейчас как церковнославянский в православии - на нем написан самый священный текст иудаизма - танах, который практически весь вошел в Ветхий завет, а так же иудейские молитвы и некоторые другие иудейские тексты.

Спиноза был очень любознательным. Он все время задавал много вопросов. Его родители считали, что из него выйдет превосходный богослов, светоч иудаизма и т.п. Но Спинозе показалось, что иудаизма ему мало. Он записался в голландскую школу Ван Дер Эндена, который был свободомыслящим человеком. Там Спиноза познакомился с естествознанием, математикой, овладел латынью. Он познакомился с работами философов. Его очень впечатлили работы Декарта и Бегеля (???)

Еврейская община всячески старалась вернуть Спинозу к еврейской культуре. Сначала на него пытались давить, в частности придали его малому отлучению - постановили, что целый месяц ни кто не имеет права общаться с ним. Потом решили его подкупить - обещали ему приличное жалованье, при условии, что он будет ходить в синагогу. Он категорически отказался. Он стал критиковать иудаизм в своих работах. Он взял некоторые факты из Библии и показал, насколько все эти данные были "не достоверны". В результате еврейская община решила придать его великому отлучению и проклятию. В 1656 году его объявили изгнанным и проклятым. Ни кто не имел право с ним имеет ни каких дел, в том числе и его ближайшие родственники. Но он не пошел ни на какие компромиссы. Тогда еврейская община добилась постановления муниципалитета Амстердама об изгнании Спинозы из города. Он поселился в маленькой деревушке, потом переехал в другую, но к концу жизни поселился в Гааге. Везде его преследовали за его прямые выпады против религии. Он был атеистом, а в то время это было диким. Он мог быть протестантом, католиком, иудеем, но не атеистом. Он умер от туберкулеза, не дождавшись выхода в свет главной своей работы. Работа называлась "этика" - труд всей его жизни.

Работы Спинозы публиковались и его имя было восславлено, как имя одного из ведущих мыслителей. К нему обращались с очень выгодными предложениями. Например ему предложили место в Гекельбергском университете с очень большим жалованием, но при условии, что он не будет ругать Бога с кафедры. Спиноза не пошел ни на какие уступки. Принц Кондель предложил ему от имени Людовика XIV большую ежегодную пенсию, если один из трудов он посвятит Людовику XIV. Спиноза написал в ответ: "Я свои труды посвящаю только истине". Это показывает, что "самым благородным" среди всех философов был Спиноза. Другие могли отмалчиваться, шли на уступки, но только Спиноза был абсолютно безкомпромиссным.
Философия Спинозы

На первых порах своей деятельности Спиноза был очарован работами Рене Декарта. Но потом он пришел к выводу, что Декарт был в основном не прав. В Декарте Спинозу раздражал дуализм, когда Декарт считал, что есть 2 субстанции, материальная и духовная, которые между собой не соприкасаются. Спиноза отвергал этот факт. > Рене Декарт придерживался принципа психофизиологического параллелизма, а Спиноза психофизиологического монизма, т.е. психическое неразрывно связано с физическим и без физического не существует.

Как противовес дуалистической модели Декарта, Спиноза разработал концепцию где существует только одна субстанция. Если Декарт был дуалистом, то Спиноза был до конца последовательным монистом.

    Так главное понятие в философии Спинозы - понятие субстанции. Субстанция - это бытие, которое ни от кого не зависит. Чтобы быть оно не нуждается ни в чем постороннем. Субстанция есть причина самой себя. У ней нет причин. Основные признаки субстанции: 1. Вечность - субстанция ни когда не возникала и ни когда не исчезнет. 1. Бесконечность в пространстве. Субстанция занимает все, и за пределами субстанции ни чего нет. 1. Субстанция не делима на чести.

Субстанция есть мир, а мир есть субстанция. И эту субстанцию Спиноза называл Богом. Спиноза был последователем пантеизма - Бог есть природа и природа есть Бог. У Спинозы был материалистический пантеизм. Это следует из того, что Бог у него это, не тот кто карает, помогает и т.д., а природа и Бог понятия тождественные. Это был способ избавиться от религии и от Бога, не отказываясь от слова.

В средневековой философии было 2 понятия natura naturans и natura naturata. Natura naturans - Бог, творящая природа. Natura naturata - все созданные, земные вещи. В нашем понимание природа это производное от Бога, она не может творить. > С тем, чтобы доказать, что природа не только производное, но и творящее он и назвал природу, субстанцию Богом.

Субстанция она natura naturans. Кроме субстанции, у Спинозы было понятие атрибутов. У субстанции есть атрибуты. то в чем выражается сущность субстанции. У субстанции бесконечное число атрибутов. Нет атрибутов, нет субстанции. Субстанция существует только в атрибутах. > Хотя атрибутов бесчисленное множество, нам известно только 2 атрибута: протяженность и мышление.

Здесь прямой вызов Декарту. У Декарта 2 субстанции - одна духовная, способная мыслить, а другая материальная, обладающая протяженностью. А у Спинозы мышление и протяженность это два атрибута одной субстанции.

Еще Спиноза вводит понятие модусов. Субстанция проявляется в атрибутах и проявляется в модусах. Если субстанция и атрибуты это natura naturans, то модусы это natura naturata. Субстанция порождает модусы. Модусы это все конкретные вещи. Камень это проявление субстанции, модус субстанции. У камня проявляется только один атрибут - протяженность. Человек тоже модус субстанции, но у человека проявляются два атрибута. Субстанция не уничтожима, а модусы могут появляться и исчезать. Субстанция это то, чтего не может не быть, а модус может быть, а может не быть. Субстанция и атрибуты это то, что не меняется, а модусы меняются. Модусы находятся в движении и движение присуще только модусам.

Субстанция порождает модусы, следовательно, все вещи подчинены закону причинности. А раз существует закон причинности, все предопределено. Нет вещей которые бы не имели причины и нет вещей, которые имели бы сверхъестественную причину. Но свобода все же есть. Свободу и предопределенность Спиноза совмещает следующим образом: Свобода это осознанная предопределенность. Когда мы понимаем, что иначе быть не может и идем, не сопротивляясь, это и есть свобода. Резко отличается от свободы Декарта. Это чем-то напоминает суждения стоиков, что свобода это осознанная необходимость. Однако стоики считали, что предопределен только конечный результат, а метод его достижения можно выбирать. А у Спинозы предопределено все. Немного перекликается с марксизмом.

Поскольку сознание это модус субстанции, а другой модус это тело, то связь сознания должна быть такой же как связь тела. Из этого следует, что мы автоматически должны выдавать истину. Но с другой стороны люди заблуждаются. Спиноза считал, что причина заблуждений - слишком многое взятое из прошлого и т.п. Однако здесь он не углублялся.

Спиноза развивает теорию познания.

    Он выделяет 4 формы познания: 1. По наслышке - что-то где-то слышал. Так говорят, значит так оно и есть. Все принимается на веру и не проверяется. Тут ни чего нельзя доказать, по-этому надо переходить к другим, более сложным формам познания. 1. Эмпирическое. Мы соприкасаемся со внешним миром и получаем опыт. Это познание более совершенное, чем познание по наслышке. Однако оно не обладает общностью. Если мы получаем определенный факт, мы не можем его доказать, принимаем как данное. Опыт не дает полного и адекватного знания. Он слишком неполон. 1. Рациональное познание. Мы обобщаем опыт. Анализируя следствие приходим к причинам. Но этого знания тоже не достаточно достоверное, потому, что мы не имеем прямого дела с причинами. Мы их устанавливаем. 1. Интуитивное. Когда мы знаем, что что-то так и иначе быть не может, однако, мистического тут ни чего нет. Это знание приходит само собой. Мы вдруг понимаем, что иного быть не может. Это полностью полноценное познание. Знание к нам приходит т.к. мышление есть атрибут субстанции.

Спиноза очень любил геометрию. Вся его "этика" построена как учебник геометрии. Сначала предпосылки, потом теорема и выводы. Спиноза был скрытый атеист, прикрытый теологической оболочкой. Он много занимался проблемой возникновения религии, сущностью религии. Он считал, что религия это есть заблуждение. Все понятия религии это заблуждения. Причины заблуждений: страх перед смертью и явлениями природы и невежество. Следовательно, нужно нести знание в массы. Нужно показывать как на самом деле происходят явления. И когда люди это поймут, религия сама собой исчезнет. Религия это особо вредное заблуждение. Религия держит людей в духовом страхе, чтобы оправдывать несправедливые порядки, хищных и тупых правителей, которые издеваются над людьми. Победить религию можно путем просвещения и убеждения - воинствующий атеизм.

Спиноза изучал Ветхий завет, особенно детально рассматривая Пятикнижие. Он приходи к выводу, что эти книги, вопреки общему мнению написал не Моисей. Эти книги писались на протяжении двух тысяч лет. Он показывает, что это компиляция разных вариантов, и что авторы не хотели соперничать друг с другом, а наоборот соединить все воедино. Следовательно "верить в это нельзя", это "чистейшей воды мифы". Например в конце пятой книги говорится, что Моисей умер и его похоронили. Где его похоронили неизвестно, но эти строки должны принадлежать Моисею.

Далее вставка из ((Лекция8))
Лейбниц, Готфрид Вильгельм. 1646-1716
Биография

В дальнейшим линию рационализма проводил Готфрид Вильгельм Лейбнец. Родился в 1646 году в воремя 30-летней войны, в городе Лейпциге. Отец - профессор права Лейпцигского университета. Лейбнец получил хорошее образование. Потом был дипломатом, а последняя должность была - заведующий государственной библиотекой герцога Наверского(?). Тогда это была очень престижная должность. Заведующи библиотекой университета был примерно равен ректору. Умер в 1716 году. Считается, что предки Лейбнеца были славянами. Он был философом, но также и математиком и естествоиспытателем. Работал в разных областях, был энциклопедически эрудирован.

Он пытался соединить разум и веру, науку и религию. В то время на территории настоящей Германии была феодальная раздробленность с одной стороны и 30-ти летняя война с другой. Поэтому Германия разлагалась, не было ни единой научной школы, ни единого класса буржуазии. Эта картина легла в основу компромисной философии Лейбница. Также она была во многом протестом против того понимания мира, которое предлагал Декарт. По декарту мир принимается как бесконечное перемещение в пространств готовых частиц по законам механики. Мертвый, застывший мир. Как реакция на этот механицизм возникла философия Лейбнеца. Он пытался показать, что мир живой, бушующий. Но в то время была только механика и отсюда масса фантастики. Фантастическое представление, с тем чтобы увидеть находящейся в вечном движении мир где все связанно, где друг без друга ни чего не существует. В этом пафос философии Лейбница.
Философская система

Монадология. Написал одноименную книгу. Исходил из того, что существует множество субстанций, которые он называл монадами.

    Монады это незримые чувствам, не физические, не материалистические, по сути, духовные, существуют вне времени и пространства. Всё это чистейшей воды фантазия. Своеобразная разновидность объективного идеализма. Все монады замкнуты и не имеют связи. Каждая монада отображает в себе весь мир. Между монадами существует определенное согласие, которое исходит из воли божества – божественная гармония. Все монады друг от друга отличаются. Монады – духовные сущности. Одни монады обладают перцепцией (восприятием), но существуют и монады апперцепции (сознания). Все монады обладают способностью к действию и к страданию. Они всегда деятельности, они всегда активны. Есть монады простые (голые), в которых существует только перцепция, да и то не явно, есть вторая категория монад с ясной перцепцией – животные, третья категория (апперцепция) – это духи (люди). Монады лежат в основе мира. На поверхности лежит материя, но в её основе лежат монады. > Лейбниц выделял 2 формы материи. Первая материя, обладает свойствами: Протяженность и непроницаемость. 2ая материя (<<материя физиков>>) находится в вечном и бесконечном движении переданной от монад. Нет никакого абсолютного покоя. Всякий покой только относителен. Представление о времени и пространстве было реляционная. (Развил теорию Декарта.) Нет времени и пространства как самостоятельных сущностей. Время и пространство – просто порядок сосуществования вещей в мире.

Развивал гносеологию. Он был рационалистом, хотя и не отвергал опыта. Пытался примерить рационализм и эмпиризм. Но на первый план выступал рационализм. Рационализм проявлялся в теории врожденных идей, что есть такое рациональное знание с которым нам врожденно и эти истины настолько очевидны, что мы не может не признать их. У нас эти врожденные идеи очень смутно существуют в начале. В нас в уме есть смутные врожденные идеи и надо их выявить и обработать. Много писал о сознании.

    Выделял 2 категории истины. Истины разума. Это такие истины, которые не могут не быть истинными, она необходима и не может не быть и иной быть не может. То, что она истина мы доказываем с помощью дедукции. Всякая любая противоположность ложность. (Примеры: у треугольника 3 угла). истины фактов, получаются путём индукции. Это обнаруживается в опыте. (Пример что тело падает). Мы знаем что это факт, но не можем доказать что иначе быть не может. Истины разума ставит выше истин фактов. Не был приверженцом абсолютного детерминизма в смысле как у Гоббса, что случайность – результат нашего незнания и что свободы не существует. Лейбниц вводит понятие объективной случайности и понятие вероятности. Настаивает на объективном характере вероятности. (Лаплас настаивал, что вероятность есть субъективная величина). Конечно же, признаёт свободу человека, свободу выбора.

В обществе существует зло, грех человеческий, пытается ввести в науку религию. Признает наличие Бога-творца, – который творит монады. Проблема в том, что в мире существует зло, и значит оно тоже создано Богом. Бог создал мир со злом для того, что бы было добро. Мир, в котором мы живём – самый лучший из всех возможных миров, потому что зло в конце концом обращается добром.
