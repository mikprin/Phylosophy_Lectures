
Лекции по философии науки проф. Семенова Юрия Ивановича
Лекция 3, семестр 10, 25. 2.2006
Проблема природы знания и познания

Отформатировано: Проверено:

Мы с вами проследим, как философы шаг за шагом пришли в к мысли, что мир существует в нашем сознании. Мы начали с Элиадов и дошли до Беркли, который довёл идею до абсолюта. Потом процесс пошел назад – Юм, допустил, что мир существует не только в сознании, а следующий, Кант, допустил уже существование другого мира, однако другой мир уже совершенно не похож на мир для нас. Настала необходимость для обозначения термина для вещей, которые существуют в нашем сознании. Надо было взять только содержание сознания, отвлекаясь от субъективной формы. Решающий шаг сделал Кант, который ввёл термин <<вещь в себе>>. Затем Гегель, который ввёл термин <<вещь для нас>>. Ну а дальше последовали Энгельс и Ленин, которые активно использовали эти термины. Возникла необходимость назвать совокупность всех вещей в себе. Первым употреблял эти термины Фейербах, который назвал это <<миром в себе>>. Не просто вещь в себе, а мир в себе - вообще все объективно существующие вещи. Потом возникла необходимость в термине <<мир для нас>> - мир, как он существует в нашем сознании. Эти термины вышли за пределы кантианской философии. Эти важные термины любой современной теории познания. Когда выяснилось, что мир существует в нашем сознании, стало ясно, что такое знание – иметь знание о вещи значит иметь её в сознании. Познание – это нахождение вещи в сознании. Такой позиции придерживаются все философы, не зависимо от ориентации, материалисты, идеалисты, дуалисты...

Но ведь мы не просто имеем готовые знания, мы их получаем. Напрашивается вопрос: а откуда они возникают? Это вопрос о том, что такое познание как процесс. Для того чтобы разобраться в этой проблеме, надо разобраться в другом вопросе. То, что мир существует в сознании это бесспорно. Но существует ли он вне сознания? И пока мы не ответим на этот вопрос, мы не можем ответить на вопрос о познании.
Основные решения вопроса о существовании мира и вне сознания

    Первый ответ – мира нет, мир существует только в сознании. Это ответ Беркли, субъективный идеализм. Быть – значит быть воспринимаемым.
    Второй ответ – агностики и феноменалисты, существует ли мир вне сознания или нет, абсолютно не разрешим.

    Третий ответ – мир существует не только в сознании, но и вне сознания. Но этот третий ответ распадается на два разных ответа, на кантианство и материализм. Но за этим сходством скрывается огромное различие.

Два основных решения вопроса об отношении вещи в себе и вещь для нас

Нарисуем разные точки зрения:

    Беркли. Рисует на доске круг - мир для нас. Кроме него больше ничего нет и быть не может.
    Юм. Рисует круг - мир для нас. То что снаружи - непонятно, есть там что-либо или нет, мы не можем заглянуть за границу круга. Так что снаружи рисует много знаков вопроса - там могут быть вещи в себе, но они там могут и не быть.
    Кант. У Канта бесспорно вещи существуют в сознании. Рисует круг - мир для нас. Но также кроме мира в сознании существует и мир вне сознания. <<Мир в себе>> и <<мир для нас>> разделены непроницаемой стеной. Не могут вещи в себе войти в мир для нас и наоборот. Вещь для нас она только есть для нас. По ту сторону мир – трансцендентный. А в чём отличие от материалистов? Попробуем изобразить точку зрения материалистов.

    Материалисты. Хотя материалисты признают мир в сознании, но начинать нужно с мира самого по себе. Этот мир бесконечен – бесконечный во времени и пространстве, объективный мир Вселенной(рисует полукруг). Чтобы понять взаимоотношение мира в себе и мира для нас и изобразить мир для нас, проведём эксперимент. > Вот видите кусочек мела. Пусть я убираю его за спину. Вы его не воспринимаете. Он является вещью в себе. Теперь я его достал, и вы его видите, он стал содержанием вашего сознания, стал вещью для нас. Теперь вопрос, остался ли он вещью в себе? Он стал одновременно и вещью для нас и вещью в себе; он перестал быть вещью в себе и не перестал. Понятие "быть" вне сознания имеет два смысла – просто быть, существовать и второе - быть не познанным. Отсюда видно, что понятие "вещь в себе" имеет два смысла – просто объективная вещь, второй непознанная объективная вещь. А ведь вещь в сознании тоже имеет два смысла. Первый – это существовать только в сознании, а второй – быть объективно познанной вещью, т.е. существовать и в сознании и вне сознания. У Канта только одна вещь для нас - только в сознании, а для материалиста обе. Есть вещи только в сознании - ангелы, черти, лешие. Однако, некоторые вещи для нас могут переходить в вещи в себе, это есть человеческая деятельность – <<чертёж>> превращается в продукт. Таким образом, не только вещи в себе превращаются в вещи для нас, но и вещи для нас превращаются в вещи в себе. Рисует на доске что-то, что объяснить нельзя. Мы делаем шаг вперед, познаем, и мир для нас увеличивается, все больше приближается к миру в себе. Есть вещи, которые существуют только в сознании и никакого отношения к внешнему миру не имеют.

Проблема понимания процесса познания

C точки зрения материалистов познание – это процесс превращения вещей в себе в вещи для нас, при котором вещи в себе перестают быть вещами в себе и остаются вещами в себе. Мир в себе превращается в мир для нас. А как же быть с идеалистами, которые не признают вещей в себе? С точки зрения Канта мы сами создаём мир из хаоса ощущений, при помощи категорий мы это всё рассовываем по местам. Между прочим, в этом есть большой смысл, мы не просто глазеем на мир – мы мыслим. Другое дело, что у него это создание мира для нас, но не хочет он принять, что мы создаём мир в себе. А субъективные идеалисты? Ведь знать – это иметь в сознании, но поскольку все вещи уже в нашем сознании, то всё уже познано и процесса познания нет и быть не может. Но ведь он идет! Приходиться Беркли крутиться. Откуда берутся вещи и куда деваются? И его точки зрения, никуда они не деваются, вещи продолжают существовать, но в сознании других мыслящих духов. А еще есть Бог, который вкладывает и выкладывает информацию о вещах. Агностикам проще – мы не знаем и знать можем и не хотим. Они не только отрицают возможность распознания сути мира, не только отрицают возможность проникновения за круг сознания, но и отрицают возможность раскрытия природы самого познания.

Возникает такой вопрос. С точки зрения Юма знать о том, есть ли вещи в себе мы не можем. Что значит знать о вещи? Иметь её в сознании. Что значит знать о вещи вне сознания? Знать о вещи, о которой мы заведомо ничего не знаем. С точки зрения формальной логики это неопровержимо. Так вообще нельзя доказать, что мир существует? Рассмортим этот вопрос в следующем разделе.
Можно ли доказать существование мира вне сознания?

С точки зрения Юма нельзя. И с точки зрения формальной логики Юм неопровержим. Но формально-логический вид доказательств - не единственный вид доказательств, есть ещё и другие виды мышления, где есть другие способы доказательств. Например, никакая теория не никогда не выводится логически из фактов. Но это не значит, что эта теория неверна, её можно подтвердить другими способами. Есть масса видов доказательств, одно из них – это практическая деятельность. Мы преобразуем мир как нам нужно исходя из знаний о нем. Значит, мир существует независимо от нашего сознания. > Обратимся к истории человечества. Когда появились люди? Тут есть две точки зрения – одни говорят, что 2.5 млн лет назад, другие - что 1.8 млн. Тогда начало возникать сознание. Всё стало возникать. Окончательно возникло сознание 40000 лет назад. Спрашивается, был ли мир до этого? А Вселенная? Большой взрыв был 12 млрд лет назад. Оно было, а где оно было – вне сознания. Или проще. В 1897 был открыт электрон. А при Аристотеле были электроны? Были, а потом они вошли в сознание, то есть стали вещами для нас. Уран вычислили теоретически, потому что нашли расхождения с ЗВТ для других планет. Рассчитав они вычислили массу и указали координаты где её надо искать. А потом открыли ещё и Плутон. Так спрашивается, эти планеты то были или не были до того, как их открыл человек? Наука тем самым подтверждает, что существует мир вне сознания и входит всё больше и больше в сознание. Значит мир есть вещи в себе, которые шаг за шагом входят в мир для нас. С этим связана эволюция аналитической философии. Есть такая её разновидность – неопозитивизм, который всегда себя объявлял философией науки, которая должна познать картину научного знания. Но сами они были агностики – феноменалисты и не допускали мысли о том, что можно знать, есть объективный мир или нет. На первых порах они объявляли себя защитниками науки, но в ходе развития всё более и более уходили от этого. Они понимали, что их философия противоречит элементарным открытиям. И результат - крах неопозитивизма, приход постпозитивизма, и всё приходит к тому, что разницы между наукой и сказками никакой нет. И кто прав? Да все правы и все неправы, так как нет никакой объективной, независящей от человека истины. Мир придумывается учёными, а не открывается. Что наука, что придания, что Библия – одно и тоже всё. И аналитическая философия из попытки объяснения научного знания пришла к этому, поскольку наука находится в вопиющем противоречии со всеми положениями нео- и постпозитивизма.

Возникает проблема соотношения мира в себе и мира для нас. Ясно одно, что мир в себе и мир для нас не совпадают по содержанию, потому что никогда мир в себе не войдёт в мир для нас, так как мир бесконечен. Процесс познания в этом смысле бесконечен. Пока нет никаких преград для познания, мы узнаём, что еще больше остаётся непознанным. В этом смысле мир в себе всегда шире мира для нас. И мы возвращаемся к проблеме восприятия и предмета восприятия. Кто близорук, видит по-разному, оденет он очки или нет. …Далее идёт отжиг про бабушку… Мир в себе и мир для нас и одно и обязательно не одно и тоже. Стоит нам выдернуть один момент и мы оказывается либо во власть Беркли, либо Канта.

Доступна фотография этой лекции в приложенных файлах. &nbsp;

Сайт создан с помощью CSS фреймворка Amazium
