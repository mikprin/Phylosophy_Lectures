
Лекции по философии науки проф. Семенова Юрия Ивановича
Лекция 1, семестр 10, 11. 2.2006
Введение

Отформатировано: Проверено:

История философии закладывает фундамент для понимания философии, но это еще не сама философия. В этом семестре мы будем слушать курс самой философии. На том фундаменте, который был создан в течение года, мы попытаемся возвести здание философии.

Ознакомимся с основными проблемами и как они решаются. Мы опустим важный раздел философии, который называется философия истории социальной философии. Мы будем заниматься тем, что называется теорией познания, гносеологией, эпистемологией и, конечно, выводами из этой теории познания, а именно методологией. Мы рассмотрим основные философские проблемы, как и когда они возникли, какие решения предлагались, в чем достоинства и недостатки предлагаемых решений, почему шли дальнейшие попытки решить, чем не устраивали предлагаемые решения и т.д., и какое решение предлагается сейчас.
Вступительная тема(повторение). Общие понятия теории познания

Любая наука – это систематическое целенаправленное исследование. Люди всегда познавали мир, без познания жить невозможно. Тем более, что люди не просто приспосабливаются к миру, а они его преобразуют, переделывают, поэтому стали необходимы знания об этом мире. Поэтому люди всегда во все времена начиная с самых первых людей занимались познанием мира. Но это познание мира носило своеобразный характер – оно было не специальном видом деятельности, а было вплетено в практическую деятельность людей. Такое познанием обычно называется обыденным познанием, житейским. А потом рано или поздно познание вылилось в особый вид деятельности. Появились люди, для которых познание стало профессиональным видом деятельности, т.е. они занимались познанием и только им. И главным для них было познать, а уж потом это знание они, но чаще другие, применяли в практической деятельности. Это и есть научное познание – систематические целенаправленное познание. Его обычно называют исследованием. Каждый ученый является исследователем.

Спрашивается: а философия занимается таким систематическим целенаправленным исследованием? И еще один вопрос: наука дает знание такое, которое нельзя получить никаким другим знанием; а философия, дает ли она такое знание? Т.е. занимается ли она открытиями, получением другого знания? Можно сказать, что философия тоже занимается исследованием и в этом смысле является наукой, правда своеобразной наукой. Наукой, существенно отличной от всех остальных наук. И тем не менее занимается исследованием и в этом смысле занимается наукой. Появляется вопрос: а что является объектом исследования философии(в прошлом семестре этот вопрос был поставлен в форме: есть ли объект, которым занимается только философия и никакая другая наука? )?

> Надо сказать, что если такой объект есть, то уже существование философии оправдано и необходимо. Только она может дать такое знание. Правда если такого объекта не обнаружилось – это еще ничего не говорит о бесполезности философии. Почему? Ведь есть науки, которые занимаются не каким-то конкретным объектом, а какой-то стороной реальности действительности.

Объектом исследования философии является истина(противоположность – заблуждение). Можно сказать, что все науки занимаются поиском истины. Возникает вопрос: нужно ли изучать что такое истина, какие пути ведут к истине, какие – к заблуждению? Конечно надо! Спрашивается: специалисты в какой науке занимаются этим? Специалисты в области конкретных наук ищут истину о чем-то, отличном от истины. А философия как раз и изучает истину, природу самой истины, т.е. поясняет что такое истина, что такое заблуждение, как идет процесс постижения истины, какие препятсвия здесь возможны и прочее. Самое краткое определение философии – это наука об истине. Философия исследует процесс постижения(познания) истины, поэтому она является теорией познания.(можно добавить конечно <<теория познания истины>>, но это не нужно, т.к. если мы не идем к истине, то это не познание) Это конкретизация положения о том, что философия есть наука об истине.

О том, что философия – наука об истине, писали многие крупнейшие философы. В самой яркой форме эту мысль выразили 2 крупнейших представителя философской мысли – Аристотель(написал об этом в своей главной книге по теории познания <<Метафизика>>) и Гегель. Для обозначения теории познания(русское название) в философских текстах используются 2 термина(заимствованы из греч. языка) – гносеология(от греч. слова гносис – знание) и эпистемология(от греч. слова эпистема – знание). Познание истины не бесполезно: процесс познания изучают чтобы прийти к цели, к истине. И вот когда философия выступает как практическое руководство к мыслительному дейтсвию, то она выступает как метод(от греч. образ, способ, путь, дорога) – путь к познанию истины. По этому пути мы ходим мыслями.

Существуют две ступени познания: чувственное познание и мышление(разум, рассудок). Так вот чувственным познанием мы управлять не можем, поэтому метода чувственного познания не существует. Существует только метод мышления.

Существуют разные методы в различных науках. Философия есть предельно широкий метод познания. Любой предельно широкий метод познания есть метод мышления. Предельно широкий - означает применимый во всех областях, всюду. Философия, вооружая методом, исследует метод, улучшает его, поэтому она является учением о методе, методологией.

Истина(определить истину значит определить и заблуждение) – это соответствие между миром и сознанием(впервые это определение дал Платон, но в очень нечеткой форме; его идею подхватил и развил Аристотель). После понятия истины выступили два понятия: мир и сознание(эти понятие присутсвуют в любой философской системе, хотя их могут называть по-разному: мир и сознание, бытие и мышление, психическое и физическое, духовное и материальное, идеальное и материальное. Поиски определения с помощью формальной логики(поиск более общего понятия и определения мира и сознания через него - например, негр - человек с определенными признаками) тщетны и бесполезны. Определить понятия мира и сознания можно только одним способом – раскрыть отношение между ними, а именно, определить, какое из них является первичным, а какое вторичным. Определяя что из них первично, мы таким образом определяем вторичное, и наоборот. Так вот нельзя дать определение миру не определив сознание, и наоборот. Это значит, что философия, давая нам эти определения, дает нам самый общий взгляд на мир. Таким образом, философия является наукой о познании, которая пытается раскрыть сущность познания и должна обязательно обладать общим взглядом на мир, и она всегда дает самый предельно общий взгляд на мир. Иначе философия является предельно широким мировоззрением.

Для обозначения философии как взгляда на мир существует термин онтология(от слова онто – сущий).

Из истории философской мысли мы знаем, что философия кроме онтологии еще была общей картиной мира. Раздел философии, где давалась картина мира, носит название натурфилософии(от слова натура – природа, буквально – философия природы). Впервые философию природы мы встречаем у Анаксимандра, потом Анаксимена, потом у других .... Ну и пожалуй такую яркую картину мира пытался нарисовать Гегель во 2м томе своей энциклопедии <<Философия природы>>. Надо сказать, что сейчас философия природы как раздел философского знания исчезла. В ней нет нужны(так как наука сейчас развита намного сильнее, чем раньше). А онтология не исчезла. В чем ее особенность? Вот скажем, философия занимается проблемой отношения мира и сознания, а следовательно и проблемой о свободе и необходимости. А её нельзя решить, не решив проблему, есть ли в мире случайность, или одна только необходимость - это проблема онтологическая. Проблемы онтологии – это такие проблемы, без которых нельзя решить проблемы гносеологии. Т.е. когда философия обращается к онтологии, она решает такие проблемы мира, без которых проблемы гносеологии абсолютно не разрешимы.

Когда говорят, что человеку нужно иметь целостное мировоззрение, спрашивается, зачем? Чтобы лучше разбираться в мире. Разбираться головой, мыслями, т.е. познавать. В этом смысле общий взгляд на мир является методом познания и нужет для того, чтобы лучше познавать мир. И в самом деле, как можно развить теорию познания мира, если мы не имеем ни малейшего представления о том, каков мир сам по себе независимо от сознания. Так что философия всегда является онтологией.

    На основной вопрос философии есть два главных ответа, прямо противоположных: мир первичен, сознание вторично и наоборот. В связи с этим имеются 2 основных течения по главным признакам(течений много, но все они не по главным признакам): материализм и идеализм.

Материализм

Что означает, что мир первичен, а сознание вторично и производно? Это значит, что мир существует вне и независимо от сознания. Мир является не просто реальностью, а объективной реальностью (определение: Мир есть объективная реальность), причем стоит подчеркнуть: объективная реальность – это такая, которая существует вне сознания, не как содержание словами, и которая существует независимо от сознания.

Можно спросить: разве <<вне>> и <<независимо>> - это не одно и то же? Не совсем. В прошлом году мы рассматривали французских материалистов, их взгляды на общество, историю. Они прекрасно понимали, что есть система человеских взглядов, т.е. общественное сознание, и называли его общественным мнением. У них возник вопрос: откуда возникает общественное мнение, которое определяет поведение, поступки людей и т.д. И, будучи сенсуалистами, эмпириками, они пытались найти первоисточник этого мнения. Искали они его в том, что называли общественной средой. Т.е. они прекрасно понимали, что в обществе есть не только взгляды, но и нечто, что находится не в сознании – различного рода социальные институты: государства, партии, различные группы... Ясно одно – эти институты существуют вне сознания, т.е. среда существует вне сознания. А потом возникал вопрос: а почему такая среда, а не иная? Исходили из того, что общество создается людьми, <<такими людьми>> с <<такими взглядами>>. И получается, что общественная среда создается общественным мнением(сознанием), т.е. общество существует вне сознания, но зависит от него.

Если мир существует вне и независимо от сознания, это значит, что мир был и тогда, когда не было сознания, что мир для того, чтобы быть, не нуждается ни в каком сознании. И из этого вытекает, что мир существовал всегда. Мир – это все, за его пределом ничего нет.

Для обозначения мира, существующего вне и независимо от сознания, есть термин – материя. Другое дело, что этим термином иногда пользовались и идеалисты и вкладывали иногда разное содержание. Материалисты вкладывали всегда одно и то же: материя – это то, что существует вне и независимо от сознания. Это единственный признак материи, который присущ всей материи. Все остальные признаки присущи каким-то формам материи.

    Надо сказать, что до этого взгляда подняться материалисты смогли не сразу. В какой-то степени такую точку зрения остаивал Гольбах (франц. материалист 18 века). Но он был не до конца последовательным, т.к. он считал, что это главные признак материи, но есть и другие, также присущие все материи. Эту точку зрения отстаивали многие другие материалисты. Ну какие другие признаки материи есть? Материя – это то, что можно постучать, потрогать, нажать...

Если взять 18 и весь 19 век, то как представляли себе картину мира естествоиспытатели и материалисты? Есть кирпичики мира – атомы(неделимые). Имеют форму шариков, как всякие материальные тела подчиняются законам Ньютона. От сочетания атомов и возникает главная материя, фундаментальная. Так что свойствами материи были не только объективность, но и непроницаемость, наличие массы и т.д. Именно с этим и связаны все те злоключения, которые претерпела физическая наука на рубеже 19-20 вв. В это время произошел перелом, который получил название революция в естествознании. После этого резко изменился взгляд на мир. Когда узнали о волнах, о делимости атома, о том, что атом – это электроны и протоны, а они в свою очередь – просто заряды, которые нельзя потрогать, пощупать, то материалисты закричали <<караул>>. За это уцепились идеалисты, точнее агностики, которые пытались создать третью линию философии, отличную от материализма и идеализма. Они заявили, что раньше наука опровергла идеализм, теперь наука опровергла материализм. Но возникает вопрос: а поля? Они же существуют независимо от сознания. Обладают свойствами, которые мы изучаем и которые не зависят от людей. Они воздействуют на приборы и т.д., т.е. существует объективная реальность. И как только мы говорим об этом, становится понятно, что понятие материи не исчезло, просто усторело представление о материи как о носителе нескольких необходимых признаков. А признак, присущий всей материи, есть только один и других быть не может.

Стоит уделить внимание тому факту, что материя – это не просто общий признак, присущий всем материальным вещам. Материя есть обязательно нечно цельное, т.е. все материальные образования образуют некую цельную систему и за ее пределами ничего не существует. Есть один единый мир и кроме него никакого другого мир существовать не может – материалистический монизм. Так что философия материалистическая является философией монистической, она доказывает, что существует один единственный мир. Правда здесь возникает вопрос: а как быть с сознанием? Оно материально или нет? Это разновидность материи или нет? Подчеркнем, что сознание не материально, а идеально. > Это легко продемонстрировать. Например, на Канте. Как-то заметил, что есть нестираемая грань между идеальным и материальным. Пример: вы приходите в трактир и хотите выпить пива. Для этого нужны деньги. Но кроме этого нужно еще иметь деньги в голове. Я же думаю, значит деньги существуют как содержание моих мыслей. Кстати, у меня может быть очень много денег в голове, но эти деньги никто не будет принимать, а принимать будут только материальные деньги. На этом простом примере Кант показал, что есть нестираемая грань между бытием вещей вне сознания и бытием вещей в сознании.

Возникает другой вопрос: если в мире есть нематериальное, идеальное, то какое право мы имеем говорить о материальном единстве мира, когда есть не только материя? Дело просто в том, что идеально существо никогда не образует самостоятельного мира. Это просто свойство материального мира и существует не как мир, а как свойство материальных явлений. Так что допущение сознания идеальным не колеблет принципов материального монизма, потому что с такой точки зрения духовное, отличное от материального мира, отдельного мира не образует, а просто существует в этом мире в качестве продукта высшего материального органа – материи.

Но все же, какой конкретный смысл вкладывается в слова вторичность сознания? Мысли человека существуют в голове, как порождение мозга. Иначе говоря, сознание есть продукт материи.

Возникает вопрос: что есть сознание, в чем его суть? Сознание есть отражение объективного мира. Копия, образ реальных вещей мира. Вряд ли имеет смысл спорить, что первично, а что вторично - копия или оригинал. И вот второй смысл понятия вторичность сознания - образ, копия внешнего мира. Короче говоря, это такой продукт материи, который является образом внешнего мира. Такое порождение материи, которое есть отражением материи, и такое отражение материи, которое является порождением материи.

Чтобы до конца понять материализм, нужно еще остановится на таких понятих как объект и субъект. Этими понятиями пользуются все философы, но мы рассмотрим только такие понятия, которые используют материалисты. Объект – это то, что познают, субъект – тот, кто познает. Объект – то, что познается субъектом. Теперь можно сказать в чем отличие материального от идеального. Материальное может быть только объективным, а идеально может быть и объктивным и субъективным, т.к. оно есть образ внешнего мира, оно есть порождение мозга индивида(единство объективного содержания, которое всегда содержится в субъективной форме).

    Существует несколько разновидностей материализма. Материализм Фалеса, Анаксимена, Анаксимандра. Атомистический (Левкипп, Демокрит, Эпикур, Лукреций Карр), деистический материализм 17-18 века (Локк, Толанд, Пристли). Механистический (метафизический) материализм франц. 18 века и т.д. Они много в чем отличались, но в все сходились в одном: мир существует независимо от сознания и отражаятся в их сознания.

Когда излагались основные идеи материализма, были даны и общий взгляд на мир материалистический(мир есть объективная реальность, которая существует вечно, никогда не возникала и не исчезнет) и общая теория познания материализма(теория отражения, наше сознание есть образ, копия, отражение реального мира). Эти положения так или иначе принимают все материалисты.

Материализм есть одновременно и онтология и гносеология. Невозможно изложить материализм, не давая либо гносеологии, либо онтологии.

Самое важное деление внутри материализма – это материализм до Маркса и материализм марксистский. В чем их отличие? Дело в том, что весь материализм до Маркса был материализмом только в понимании природы, т.е. эти люди не сомневались, что природа существует вне и независимо от сознания человека, а человек познает и отражает эту объективно существующую природу. Но когда материалисты обращались к обществу, то приходили к выводу, что общественная среда есть производная от общественного сознания, и что ход идей определяет ход истории, и что характер идеи определяет характер общественного строя. А что это такое? Это – идеализм. Т.е. есть общественное бытие и оно есть вторичная производная от общественного сознания. Иначе говоря, они переходили на позиции идеализма(очень своеобразного, который можно назвать социоисторический идеализм). И только Маркс и Энгельс открыли объективное социальное бытие, нашли объективный источник общественных идей и создали впервые материалистическое понимание общества и истории.

Как видно из вышесказанного, материализм старый был ограничен взглядом на природу, поэтому его можно назвать натурматериализмом, а материализм марксистский – это материализм всеобъемлющий(пан(то)материализм). Забегая вперед, надо сказать, что сознание является социальным продуктом, хотя до этого говорилось, что сознание - порождение мозга. Это правда, но сущность его не в этом. Вот скажем, если взять человеческого детеныша и изолировать от общества, и он будет воспитан животными, он не будет обладать мышлением, а значит, человеческий дух, сознание есть продукт общества. Отсюда вывод: если мы не обладаем материалистическим понимания общества и природы, то мы не можем раскрыть природу человеческого сознания, ибо оно своей сущностью есть продукт социального. Поэтому в стане материалистов были определенные непоследовательности в трактовке сознания. Иногда они склонялись в некоторых вопросах к идеализму. А мы уже говорили, что проблемы гносеологии и онтологии неразрывно связаны, поэтому когда они отходили от материализма в области гносеологии, то нередко отклонялись от него и в области онтологии, и отходили от материализма. В этом смысле материализм натурматерилистический был ограниченным не только в том смысле, что не было материалистического понимания общества, но были определенные непоследовательности, неправильная трактовка(нематерилистическая) и проблем гноселогии, и проблем онтологии.
Идеализм

Для идеалистов характерно то, что они в основу мира кладут сознание, и выводят мир из сознания. Для них мир есть производная сознания. В их рядах нет полного единства. Существуют несколько разновидностей: субъективный и объективный. Бывают также промежуточные линии, пытающиеся соединий эти два направления.
Субъективный идеализм

Крупнейшим и классическим представителем этого направления в истории философии был Джордж Беркли. С точки зрения субъективных идеалистов(си) мир существует в нашем сознании как содержание нашего сознания. Но ни один си никогда не отрицал, что есть природа, вещи, предметы...Они с торжеством опровергают, когда им это приписывают. Главные вопрос, который ставит отличие между си и материалистами: где существует природа? Для си ответ: природа существует в сознании, она есть содержание нашего сознания, а в том, что она есть, никто не сомневается.

    Иногда им приписывают взгляд, что мир существует в голове человека. Но могут ли си утверждать подобные вещи? С точки зрения си голова(как и все вещи) – это комплекс ощущений, восприятие, т.е. для них вещь и восприятие – одно и то же. А если так, то значит голова есть комплекс наших восприятий. Тело существует в сознании как комплекс ощущений. Я - это дух, сознание

Если эту точку зрения продолжать последовательно, то она приведет к той точке зрения, что один я существую, и больше никого нет, все остальное – порождение моего сознания. Это доведенный до предела субъективный идеализм называется солипсизм.

Эта точка зрения настолько нелепа, что все си пытались от нее откреститься и где-то, не доходя до конца, свернуть в сторону. У них возникало много неприятностей(зачем читать книги, писать лекции, если это все в моем сознании; почему не получается так, как я хочу...). Согласно принципу Беркли: быть – значит быть воспринимаемым. То, что не воспринимается, - того нет. И что получается? Пока он смотрит на нас, мы существуем, когда закрывает глаза – исчезли, открыл – опять появились, т.е. мир существует кусками. Беркли пытался выпутаться из этого тупика, доказав, что мир существует непрерывно. Он вводит мыслящих духов. <<Я - один из мыслящих духов, но кроме меня есть и другие мыслящие духи>>. Таким образом он признает существование чего-то вне его сознания, этим подрывая субъективный идеализм.

Перед ним стала еще одна проблема: отличие между восприятиями(ощущениями) и представлениями. Предметы действуют на наши органы чувств и возникают ощущения. Предметы действуют на разные органы чувств. Когда различные ощущения идут от одного предмета, они сливаются и возникает восприятие(в мозге образуется энграмма). В этом смысле восприятие есть комплекс(не сумма!) ощущений. Особенность восприятий и ощущений в том, что они существуют пока предметы действуют на нас. Но можно воспроизвести образ предмета, который когда-то действовал на органы чувств – это называется представлением. Таким образом, представление – это тоже чувственный образ предмета, который существуют в отсутствии воздействия этого предмета.

С точки зрения материалиста???(мне кажется, что все-таки идеалиста, но он так сказал на лекции) что восприятие, что предмет восприятия существуют только в сознании. Во внешнем мире ничего нет, т.е. нет самого объективного мира. Есть мир в сознании. Так вот как отличить? Провести грань между тем, о чем я думаю, но чего сейчас нет, и тем, что сейчас есть, т.е. между вещью и ее представлением. Ведь и то, и другое находится только в сознании.

Представление не поддается воздействию нашей воли, т.е. нельзя его изменить, оно объективно. Откуда береться это бессилие? Беркли вынужден ввести высшего духа. Этот высший дух творит мыслящие духи, мир сущесвует в сознании мыслящих духов. Часть содержаний вкладывается духом и поэтому не зависит от воли субъекта. Таким образом Беркли решает этот вопрос путем перехода к объективному идеализму. Таким образом наглядно показано, что субъективный идеализм несостоятельный.

Кстати, си - убежденные монисты.
Объективный идеализм

Согласно этой точке зрения в основе мира лежит некое объективное сознание(мировой разум, абсолютная идея, абсолютный дух...) Сознание ничье, существующее само по себе. У объективных идеалистов(ои) существует 2 сознания: один дух абсолютный и еще человеческий дух, объективный дух и субъективный дух.

Среди компонент систем объективного идеализма – объективный мир(объективный в смысле независящий от субъективного духа, но который может отражаться в нем(Фома Аквинский)).

В чем отличие объективного идеализма и материализма – ведь и там и там есть объективная реальность. С точки зрения и тех и других мир объективен в том смысле, что существует независимо от субъективного духа. Но с точки зрения материалистов никакого другого сознания кроме субъективного нет. И мир существует независимо от такого сознания. А у идеалистов мир не зависит от субъективного сознания, но зависит от объективного сознания, он так или иначе есть его порождение. Объектиные идеалисты являются монистами, потому что объективный мир есть порождение объективного сознания. Можно выделить 2 разновидности объективного идеализма. Одна: когда существуют рядом объективный дух и объективный мир(Платон). Мир вещей и мир идей(Эйдесов), но мир вещей порожден миром идей. Другая точка зрения: мир объективно существует, существует объективное сознание, но не рядом, а в самом мире как его основа, каркас. Первичной является духовная сущность этого мира, порожденная объективным сознанием.

У ои появляется понятие объективного сознания. Откуда оно? На чем основывается представление, что есть некое объективное сознание? Есть такое объективное сознание – знание, накопленное человечеством за всю его историю. Это сознание заметили независимо от субъективного сознание. Это сознание даже в большей степени определяет мышление человека, чем сам объективный мир. Таким образом есть объективное сознание, но оно исторически возникло из работы субъективного духа, является его производной. Это объективное, независимое от субъективного сознания, но зависимое от объективного мира, оно является его отражением. Это опыт человеческого познания мира.

    Все идеалисты основываются на чем-то, объективное существующем. Другое дело, что они толкуют это по-своему, абсолютизируют. Поэтому существует коренное отличие между идеализмом и религией. Религия основывается на вере, а идеализм никогда не аппелирует к вере. Идеалист никогда не требует, чтобы вы поверили в него, он стремится доказать, что его построения верны и приводит факты, которые, если почитать книгу Беркли <<Три разговора...>>, покажутся довольно умелыми и приводят к тому, что материалист поднимает руки и сдается(другое дело, что использовались разные подтасовки). Так что идеализм всегда аппелирует к разуму, к рассудку, с доказательствами и прочее.