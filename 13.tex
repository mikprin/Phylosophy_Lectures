
\section{Философия истории Гегеля.}

Одной из частей философии Гегеля была философия духа. В разделе об объективном духе есть подраздел <<всемирная история>>, т.е он создал свою концепцию философии истории. Гегель читал курс лекций по ней и после его смерти на основе конспектов, набросков была воссоздана концепция философии истории.

Великая французкая ревоюция ясно показала, что существуют некие объективные силы, которые управляют ходом исторических событий. Жозеф Де Местр – реакционер, ярый противник революции, подчеркивал объективную предопределенность хода Великой французской революции. В его книге – <<Размышления о Франции>> - отчетливо показано, что это объективный процесс, что революции не могло не быть. Есть бог, который определяет ход истории и революция – божье предустановление. Г.Гегель в своих попытках разработать собственную историософскую концепцию с идеалистических позиций, разработал и обосновал учение о противоречии как движущей силе всякого развития, то есть предложил ученым инструмент, позволяющий объяснить процессы развития, прогресса в истории человеческого общества.
\subsection{Суть философии Гегеля.}

Существуют некая объективная сила, определяющая ход истории, не зависящая от воли и сознания людей, определяющая общественное мнение, общественную среду, взгляды и волю людей. Эта объективная основа не остается неизменной, она меняется по линии прогресса, и когда она это происходит, меняется и общество в целом, т.е. развитие общества поступательно. Эта сила – Абсолютный Дух. Абсолютный Дух воплощается в народном духе. Дух народа – совокупность представлений народа – философских, научных, религиозных, эстетических. Абсолютный дух воплощается на каждом этапе развития в дух какого-либо народа. Такой народ становится передовым, и определяет развитие общества в целом. Он не только прогрессирует во времени, он перемещается. Однако, есть неисторические народы, где дух никогда не воплотится. Происходит историческая эстафета, когда новый народ усваивает все достижения предыдущего и становится двигателем прогресса. При этом критерием прогресса Гегель считал степень осознания свободы. Каждый новый шаг – движение к свободе человека.
\subsection{Периодизация истории.}

Гегель вводит понятие исторических миров. Он выделяет 4 исторических мира.

    Восточный – т.к. считается, что цивилизация зародилась на Востоке. Для него характерно полное отсутствие свободы – все рабы одного человека. Он делится на 3 подэтапа: Китай, Индия и Персия.
    Греческий мир – часть населения осознали свободу и стали свободными, а те, кто не осознал, остались рабами.
    Римский мир – больше вызревает понятие свободы. Начинает зарождаться христианство и подготавливает этот мир к приходу следующего.
    Германский мир. В него входили территории, которые входили в состав западно-римской империи, завоеванную германцами – Франция(завоеванную германскими племенами - франками), западная Германия, Испания, Италия, Англия.

В германском мире христианство получило полное развитие, которое требует любви, братства и свободы людей, значит только в германском мире свобода стала достоянием людей. Германский мир делится на 3 эпохи:

    Христианско-германский мир. 6, 7, 8 и половина 9 века. Империя Каролингов.
    Эпоха средних веков. 9-16вв. Для этой эпохи характерна феодальная раздробленность.
    Начало нового времени. Движение, направленное за реформацию христианской церкви. Гуггеноты, протестанты. Французкая революция.

Гегель считал германский мир – последним миром, а Пруссию – самым идеальным государством. Развитие закончилось, т.к. все обрели свободу. Развитие философии, исскуства закончилось. Дух развивался во времени, но он сам себя познал, поэтому теперь развитие остановилось. В этом Гегель противоречив – с одной стороны вечное развитие, с другой все закончилось.
Развитие абсолютного духа

Но возникает вопрос – как развивается дух? С точки зрения Гегеля развитие происходит руками людей. Дух влияет на их сознание, он внушает необходимость перемен. Великий человек раньше и лучше других понимает потребности абсолютного духа. Человек может действовать в согласии с логикой абсолютного духа, руководствуясь иллюзорным сознанием исторической необходимости. (Сравните девиз французской революции <<Свобода, равенство и братство>> и ее истинную цель – установление капиталистического строя). Это называется хитростью абсолютного духа. Гегель показал, что все либо предопределено, либо не предопределено. Ход истории не предопределен в деталях, но в глобальных процессах – обязательно. Он говорил: необходимость необходима, но она случайна. Случайность случайна, но она необходима. Например, если б не Наполеон, так кто-то другой захватил бы власть и установил диктатуру.

Гегель открыл, что есть <<черный ящик>>, который определяет ход истории, но забраться в него, раскрыть механизм он не смог.
\subsection{Младогегельянство.}

Гегель умер в 1831г. Спустя несколько лет школа распалась, наступил раскол в стане гегельянцев. Они раскололись на два течения – право-(старо-)гегельянцы и лево-(младо-)гегельяцны. Этот раскол был неизбежен. Германия к началу 19в была страной, в которой капиталистические отношения развивались, но медленно, плохо. Надеялись, что старый король Вильгельм III, и придет новый король, и на трон взойдет новый, который поймет необходимость реформ. Но на трон взошел Фридрих Вильгельм 4. Но никаких преобразований он не провел. Нарастало недовольство. Возникла оппозиция. Но она не была политической, поскольку в то время была абсолютная монархия, и демонстрации, манифестации просто были невозможны в то время. Поэтому оппозиционные течения действовали в области литературы, поэзии, драматургии, богословия, философии.

А господствующим философским течением стало учение Гегеля. Но оно было крайне противоречивым, в связи с чем последователи раскололись на два лагеря. В философии Гегеля можно было найти все – и оправдание революции, и оправдание реакции. Возник раскол в философии, который был связан с поляризацией сил охранительных и радикальных. Старогегельянцы ухватились за его систему, они оправдывали существующий строй, а младогегельянцы видели в учении прежде всего диалектику, призывы к движению, обновлению, что нет ничего вечного. Младогегельянцы были и довольно умеренные, и которые требовали свержения монархии и утверждения республики. Они выступали как революционные демократы.
\subsubsection{Давид Фридрих Штраусс(1808-1874)}

Был доцентом университета, занимался богословием. Написал книгу <<Жизнь Иисуса>>. Штраусс проанализировал всю евангиельскую литературу, и сделал вывод, что все это – просто миф. Он говорил, что Христос был как человек, но вокруг него выросли мифы. Миф же Штраусс рассматривал как результат бессознательного творчества группы людей. За свою книгу был изгнан из университета.
\subsubsection{Бруно Бауэр(1809-1882)}

Как и Штраусс, был богословом, теологом. Однако к моменту написания книги <<Критика евангельских историй, синоптиков и Иоанна>>, Бауэр уже был республиканцем, безбожником, и он этого не скрывал. В своей работе он пришел к выводу, что Христа не существовало, он чисто мифический персонаж. Бауэр положил начало мифологической школе исследования евангелия, где евангелие рассматривается как миф. <<Молчание века>> - ни один автор 1 в н.э. ничего не говорит о Христе. Кроме того есть некоторые несоответствия в датах: по Евангелию Христос родился при правлении Ирода в год, когда проводилась перепись населения, но Ирод умер в 4 г. до н.э., а перепись проводилась в 7 г. н.э. Кроме того убийство всех новорожденных также не могло не оставить следа в исторических хрониках. Происхождение мифа о Христе Бауэр обозначил как прямой осознанный вымысел отдельных людей. В работе <<Трубный глас страшного суда над Гегелем>> - Бауэр разоблачает Гегеля, как страшного потрясателя общественных основ, почти антихриста. Враг религии, церкви, государства, поклонник французкой революции.
\subsubsection{Людвиг Фейербах(1804-1872)}

Фейербах родился в Баварии. Отец – крупный криминалист. Учился в Гейдельбергском университете, но услышал о славе Гегеля и перешел в Берлинский университет, который и окончил. Там Фейербах слушал его лекции. Стал работать доцентом в Эрландонском университете. В 1830 Фейербах написал работу – <<Мысли о смерти и бессмертии>>, в которой он приходит к выводу, что никакого бессмертия нет. Книгу он писал с позиций пантеизма. Книга вышла без фамилии афтара, но вскоре его нашли и выгнали из университета. После чего Фейербах женился на наследнице фарфоровой фабрики, поселился в деревне, где жил безвылазно и работал над последующими книгами.

Но постепенно он переходил к более радикальным позициям. Стал материалистом. В 1839г написал <<Критику философии Гегеля>>. Подвергает жесточайшей критике объективный идеализм Гегеля, отрекается от него. Самая знаменитая работа – <<Сущность христианства>>. В этой книге он отстаивал материализм. Стал первым крупным немецким материалистом. Показал, что материализм не только не уступает идеализму, но и превосходит его. Фейербах говорил, что это искаженное представление реальности, а вовсе не глупость. В основе религии лежат чувства зависимости человека от каких-то сил, будь то природа или социальное общество. Она – оплот неравенства в обществе. И с религией необходимо бороться. Альтернатива – создание новой религии, в центре которой был бы человек, религию без богов. Антропологический материализм. Человек – центр философской мысли.

Сознание существует только человеческое и является продуктом деятельности мозга человека, т.е. является продуктом материи, т.е. вторично по отношению к ней, вторично по отношению к природе. Природа вечна, бесконечна, материальна. Природа движется. У Фейербаха появляются идеи о развитии, но он их далеко не разрабатывает. Он находится на позиции сенсуализма. Объективное мышление является источником знаний, а восприятие, ощущения – лишь образы, отражения этого объективного мышления. Не отвергал идею разума. Все познаваемо. Ум дает знания о том, что органы чувств дать не в состоянии – например ИК, УФ излучение.

Человек стремится к тому, что ему полезно, приятно, и стремится отстраниться от того, что ему неприятно, вредно. Человеком движет эгоизм. Понятие коллективного эгоизма: человек счастлив, когда стастливы и остальные. Главное для счастья – любовь ко всему. Человек занимает место в обществе, живет в нем, и все стремления человека имеют корни в структуре общества. Критикуя Гегеля, Фейербах не понял, что он открыл мышление как объективный процесс, открыл диалектику. Он считал, что абсолютные вещи – это плохо.
\subsection{Кризис материализма на рубеже 19-20вв.}

Начало XIX века было отмечено расцветом науки, подтверждавшей материалистические воззрения. И казалось бы, материализм должен был восторжествовать. Однако происходит оттеснение материализма на задний план. На это повлияли две причины:

Буржуазные революции прошли успешно, у власти встала крупная буржуазия, но необходимо было этот успех закрепить. Для этого нужна религия, чтобы опять вернуть народ к послушанию. Лояльной к религии философией был идеализм, в связи с чем появляется критика материализма.

Другая причина – слабость самого материализма. Он не мог решить ряд проблем в теории познания и мировоззрения. Материалисты исходили из того, что существует объективный мир. Он воздействует на органы чувств, и отражается в сознании человека. Они были сенсуалистами. Нет ничего в разуме, чего нет в чувствах. Но в разуме появляются законы, человек раскрывает сущность явлений, которая недоступна органам чувств. Откуда? Значит мышление активно, работает, извлекает законы откуда-то. Где корни активности мышления? Материалисты отвечали, что нет никакой активности(некоторые материалисты стали агностиками – человек знает явления, но сущность нет).

Человек знает законы, исходя из них изменяет мир, но еще существует не только активность мышления. Человек проявляет волевую активность – строит планы, ставит цели, рисует картину того, что должно быть, приступает к реализации. При этом он заглядывает в будущее, а чувства в этом не участвуют, т.к. они относятся к прошлому и настоящему. Значит, человек – свободен, может выбирать из массы вариантов. Но материалисты – сторонники детерминизма. В результате материалисты стали отрицать и активность сознания, и свободу, и пр.

Но и вопросы о понимании общества также заводили в тупик. Знаменитый круг – среда определяет мнение, а мнение определяет среду. Если не разорвать этот круг, развитие невозможно. Великие люди – полные творцы истории, а почему не такие идеи – а потому что не такие умы. Откуда такой ум – непонятно. В результате они переходили на позиции полной свободы. Французские материалисты были материалистами лишь в понимании природы. Создать законченное материалистическое мировоззрение, которое охватывало бы не только природу, но и общество, они не смогли потому, что не сумели, несмотря на все усилия, обнаружить объективный источник общественный идей (общественного мнения). То, что они именовали общественной средой, таким источником никак не могло быть названо. Фейербах говорил о познаваемости мира, но не объяснил, как мы узнаем о сущности явлений.
