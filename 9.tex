\section{ВЕК ПРОСВЕЩЕНИЯ.}
\subsection{Основные философско-исторические идеи эпохи просвещения}

Век научной революции, когда утвердилось научное естествознание и научный взгляд на мир, состоящий в том, что нет никаких чудес, все имеет свою причину и причина лежит в самом этом мире. В то время во Франции назревала Великая французская революция.

14 июля 1889 – штурм Бастилии.

Во Франции никакие компромиссы между буржуазией и дворянством, как в Англии, возможны не были. Просвещение в то время достигло наивысшей точки. Франция была эпицентром развития просвещения во всей Европе. Но наряду с Французским просвещением, была ещё Шотландская школа просвещения. В Германии был Гербер. А к концу века стала возникать немецкая школа классического идеализма.

    Основные идеи просвящения: Культ разума, возвеличивания разума, борьба против слепой веры, разум способен познать всё, но только в случае, если ему будет предоставлена полная свобода. Все представители были свободомыслящими, боролись за то, чтобы все имели право на собственные взгляды и имели право их высказывать. С их точки зрения разум является высшим судьёй. Только то что будет доказано можно принять во внимания. Эти идеи сказались и в России. Екатерина II заигрывала с просветителями. Идеи просвещения распространялись среди русской элиты.

Идея природы человека. Эта природа требует для развития – братство, равенство, свобода. Но сейчас не существует такого, значит – существующий порядок приговорён разумом и должен быть уничтожен. Феодальные отношения пришли в упадок, и буржуазия требовала равенства всех перед законом, освобождение ото всех феодальных уз. Разум требует, возникновения прогрессивного общества. Здесь появляется идея прогресса, и эта идея в 18ом веке восторжествовала. Она была основанная на тех открытиях, которые тогда были сделаны.

В то время началась эпоха великих географических открытий. Была открыта Америка, морской путь в Индию и совершено первое кругосветное путешествие. Европейцы обнаружили, что индейцы жили на состоянии первобытно-общинного строя. Занимались охотой и собирательством. Но самое удивительное, что у них не было частной собственности. Каково же соотношение дикого состояния и европейского состояния? Дикость – состояние, через которое пошло всё человечество. И сравнивая писание римских авторов, просветители поняли, что европейцы тоже раньше были в диком состоянии. Варварство – второе состояние. И третье – цивилизованное состояние. Так впервые стала складываться первая научная систематизация истории. А именно 3 стадии: дикость, варварство и цивилизация.
\subsection{Адам Фергюсон.1723-1816}

Представитель Шотландской школы. В 1767 году появилась книга <<Опыт истории гражданского общества>>. Тогда Фергюсон разделил всё историю человечества на 3 стадии, по состоянию частной собственности:

\begin{itemize}
    \item Грубое состояние 
    \item \begin{enumerate}
        \item Дикость. 
        \item Собирательство, нет частной собственности, нет эксплуатации и власти одних людей над другими. (Первобытный коммунизм) 
        \item Варварство ( Возникает частная собственность, и общество раскалывается на классы)
    \end{enumerate}
    \item Отёсанное состояние (цивилизованное)
\end{itemize}
Это была первая научная стадиальная концепция развития человечества.

Параллельно с этим понимаем истории, возникла другая концепция, здесь в основу легло формы обеспечения формы людей:
\begin{enumerate}
    \item Собирательская
    \item Пастушечья
    \item Земледельческая
    \item Торгово-промышленная
\end{enumerate}
Эту схему разработал Жак Дюрго (1727-1781) в своей работе <<Рассуждения о всеобщей истории>>. Одновременно с ним концепцию разработал Адам Смит (1723-1790), величайший экономист своего времени. Написал труд: <<Исследование о природе и причине богатства народов>> (1776). Представил историю как процесс, а не набор случайных событий.

Наиболее ярко изложил идею прогресса Мари Кондросе (1743-1794). Человек, который заслуживает упоминания. Он был ближайшим другом Адама Фергюсона. Был крупным естествоиспытателем и математиком. Принял активное участие в Великой французской революции. Был казнён во время якобинской диктатуры. Написал работу во время изоляции. <<Эскиз исторической картины прогресса человеческого разума>>. Картина прогресса разума, которая сметает все барьеры и определяет развитие человечества. Человечество подчинит себе природу. Рисовал радужную картину всеобщего неизбежного прогресса, что в его положнии выглядило странно.
\subsection{ Жан-Жак Руссо 1712-1778 }

Были люди, которые смотрели на идею прогрессу с сомнением, среди них был Жан Жак Руссо. Родился в бедной семье. Рано покинул дом. Познал нужду, тяжелый физический труд. Знал проблемы народа изнутри. Турды: 1.<<Рассуждения о происхождения и неравенства между людьми>>.1755. 1.<<Об общественном договоре>>. (Знамя якобинцев)

    Дикость – естественное состояние. Частная собственность это богатство одних и нищета других, с эти связан социальный гнёт, эксплуатация, падение нравов, криминал, разврат, пытки. Это страшное бедствие для человечества. Частная собственность развращает души людей. Это страшные последствия цивилизации. Нужно добиться равномерного распределения всего богатства.> Государство – продукт общественного договора. Правители должны править для блага всего народа, на смену монархии должны возникнуть органы власти подотчетные народу. Если нарушен общественный договор, то народ имеет право на революцию. Эта книга пользовалась популярностью в радикальных кругах.

То что цивилизация несёт с собой зло – это признавали многие, в том числе Морелли.(о котором ничего не известно). Он рисует естественного состояния – золотого века человечества. Нужно ликвидировать частную собственность, и тогда все источники бед человеческих исчезнут. Габриэль Мабли. Выпустил целый ряд работ похожих на Морелли, тоже социалист-утопист.

Встала проблема движущей силы истории. Стали возникать концепции, которые определяют ход человеческой истории
\subsection{ Шарль Монтескье 1689-1755. }

Французский просветитель, правовед, философ. В отличие от Руссо был человеком умеренных взглядов. Стремился вскрыть причины возникновения того или иного государственного строя, анализировал различные формы государства, утверждал, что законодательство страны зависит от формы правления. > Средством обеспечения законности считал принцип <<разделения властей>>. Именно Монтескье выдвинул идею <<разделения властей>> на законодательную, исполнительную и судебную, причем эту идею обосновал.

Работа <<О духе законов>>, где он выдвигает идею о том, что развитие человечества определяет география. Разработал концепцию географического детерминизма. Жаркие страны – климат не располагает к работе. Значит – единственный способ заставить людей работать – возникновения рабства. Дух рабства пронизывает эти общества. Деспотизм – продукт жаркого климата. А вот Европа в умеренном климате. В этих условиях люди свободны. Ещё, конечно ландшафт. Европа, где всё порезано горами, там возникают страны, а в степях – деспотические империи.

Проблема географического детерминизма в том, что невозможно объяснить, почему меняются порядки. > Поэтому появилась концепция демографического детерминизма. Превращение одного строя в другой определяется плотностью населения. Сначала было собирательство, и этого хватало. Но население растёт, и жить то не на что. Значит нужно, чтобы часть населения переселилась или изыскать новые источники жизнеобеспечения, и приходит скотоводчество и земледелие, а потом и промышленность.
\subsection{ Клод Адриан Гельвейций. 1715-1771 }

Одним из первых идею демографического детерминизма ввел Гельвейций. Две больших работы <<О человеке>> и <<Об уме>>. В работе <<О человеке>> набросал абстрактную модель под влиянием роста населения. Потом эта модель была подхвачена Артуаном Барнафом (1761-1793). Был казнён якобинским трибуналом. Написал работу <<Введение во французскую революцию>>. Пытался осознать картину революции и её последствия на ход истории. И главным образом обращал внимания на рост населения.
\subsection{ Вольтер. 1694-1778 }

Это один из его псевдонимов, один из 137. Настоящее имя – Арруэ Франса Мари. Был неистовым борцом за свободу человеческой мысли. Ненавидел религию. Поселился в Швейцарии рядом с границей с Францией, из-за гонений на родине. Когда он 1772 году вернулся в Париж, его встречали так, как не встречали ни одного короля. Он был и поэтом, драматургом, прозаиком, философом. Все свои знания он обращал на борьбу с деспотизмом. Был пантеистом. Религия – главное зло. Католическая церковь – оплот феодального порядка. Он высмеивал христианскую религию. Главный лозунг – раздавите гадину. Вся история христианской церкви – плутовство и грабёж. Вальтер боролся и делом. В 1762 году в одном из провинциальных городов был арестован Жан Калласс. Ему приписали, что он убил сына за то, что сын хотел перейти в католицизм. Когда об этом узнал Вальтер, он приложил все силы, чтобы оправдать Калласса. Он без конца вмешивался в конкретные дела. Надо сказать, что в борьбе религии (не с христианством, а вообще), действовал так, как Гоббс. Считал, что вопрос объективной реальности не может быть решен. Но, какая-та вера нужна. > Без деления общества не классы – общественная жизнь не возможна. Общество без борьбы не возможно.
Французкие материалисты XVIII
Основные идеи и представители

Представители:

    Жульен Люметри. (1709-1751) Основной труд: <<Человек-машина>>.
    Дени Дидоро. (1713-1784) Издание <<Энциклопедия наук, искусств и ремесёл>>. Пытался подвести итоги всему, что было сделано человечеством. Энциклопедия содержала 35 томов.
    Клод Андреан Гельвеций. (1715-1771)
    Поль Дитрих Анреан Гольбах. (1728-1789). <<Система природы>>. (Библия материализма)
    Жан Мелье (1664 - 1729)

Не нужно думать, что они все были абсолютно единодушны. Были убежденными материалистами и открыто себя провозгласили атеистами. Сомнений в отсутствии Бога не было.

    Главное понятие – мир. Мир – есть материя. Никакого мира, кроме материального, не существует. Мир (объективная материя) бесконечен во времени и в пространстве. Придерживались субстанциального взгляда на время и пространство. Мир находиться в вечном движении. Движение является свойством материи, и она не нуждается в первом толчке, в отличие от того, как считал Ньютон. Материя обладает активностью, нет, и не может быть абсолютного покоя, есть только бесконечное движение. Иногда это называют механистическим материализмом (редукцианизмом), то есть сведение всего к механическому движению. С их точки зрения всё это, в конечном счете, сводится к перемещению частиц. Здесь проглядывают идеи атомизма Демокрита. Существуют объективные причинно-следственные связи.

Все явления имеют причину. Отстаивали эти идеи, и защищали идеи абсолютного детермизима. С такой точки зрения случайность – просто результат нашего незнания. > Если существует предопрелённость, значит – нет и никакой свободы человека. Ему кажется, что он свободен, это есть просто обман самосознания. Были против идеи души Декарта. Просто человек механизм сложнее, чем животные. Самоё резкое отображение эта идея получила в книги Люметри – <<Человек-машина>>. Что качается души, так это просто субъективная сторона.

Занимались вопросами теории познания. Сознание человека отражает мир и оно вторично. Сознание есть продукт мозга, единственный источник знания о мире – только органы чувств, то есть они были чистыми сенсуалистами.
\subsection{Джозев Фриски (1733-1804)}

Был великим физиком и криминалистом. Открыл кислород. С его точки зрения нет двух видов опыта. > Все наши идеи, самые абстракции – переработанный чувственный опыт. Нет ничего в разуме, чего нет в чувстве. Чувства – единственный канал, связывающий мир и сознание. Считал, что к человеческой сущности подобраться так и не можем. То есть материализм был не очень последовательный.

Ставили проблемы истины. Истина – соответствие между нашим сознанием и миром. Проверка соответствия – это только опыт, практическая проверка.

Они были не просто атеистами, но были ещё и воинствующими атеистами. Они пытались объяснить людям, почему не нужна религия.

Наличие религии оны пытались объяснить: 1. невежество людей 2. страх перед грозными силами природы. 3. наличие группы людей, заинтересованных в наличие религии. <<Религия, это там где жулик встретил дурака>> - Вальтер.
\subsection{Общество и история}

Французские материалисты создали не последовательный материализм. Мы уже рассмотрели взгляды на познание, на борьбу с религией. Рассмотрим взгляды французских материалистов на общество. Как я уже говорил, что все материалисты до марксистов, все до единого, не были до конца последовательными. Они были материалистами снизу и идеалистами в понимании общественной жизни. Но возникает вопрос почему. Можно сказать, что они до конца идею материализма не воплотили. Они старались привести всё к материализму, но так и не смогли это сделать. Не смогли понять исторический процесс с материалистической точки зрения.

История – это огромный набор, пласт событий. Первый вопрос - из чего складываются события. События складываются из действительности. История есть результат человеческой деятельности. Значит - объяснить почему история разворачивается так или иначе – надо выяснить причину почему люди поступали так или иначе. Деятельность она направлена и осознана. Значит, - объяснение деятельности надо искать в головах людей. Причем это не просто сознание людей. Людей множество. Почему они действовали так, а не иначе. Когда говорят об исторических событиях, то нее все события есть исторические.
Общественное мнение

События, которые происходили в масштабах общества – исторические. Надо обращаться к таким событиям, которые изменили жизнь общества.

    Так французские материалисты вводили понятие общественного мнения. Такие представления, которые присущи не просто отдельной личности, а тех, которые разделятся значительной частью общества. Именно эти представления и приводили к историческим событиям. Общественное мнение определяет поведение людей. А откуда берется общественное мнение? Французские материалисты были сенсуалистами, решительными противниками врожденных идей. Все взгляды формируются под влиянием среды. Но взгляды на природу формируются под влиянием природы. Является отображением природных событий, тут они от теории отражений не отходили. Верные или не верные, но отражения реальной действительности.

А взгляды на общество? Они формируются тоже средой, но общественной средой. Вот так возникает общественная среда. > Существует общественные отношения, институты, общественные учреждения, они и образуют то, что называется <<общественной средой>>. Эта общественная среда и определяет общественное мнение. Что бы понять природу общественного мнения надо изучить общественную среду. Надо понять, что такое общественная среда. Вот когда они приступали к анализу общественной среды, они спотыкались. Природа дана, не зависимо от человека. Природа не нуждается в объяснениях. Общественная среда возникает под действием людей. Почему она такая? Она формируется от сознания людей. И что получается, общественная среда определяет общественное мнение, а сама среда образуется действиями людей. То есть мы оказываемся в порочном круге.

Если этот круг не разорвать, то объяснить ничего нельзя. Ведь общественная среда и общественное мнение меняются. В разных странах разная общественная среда и разное обще мнение. В одной стране существует несколько мнений. В таком случае общество развиваться не может. Что бы изменилось сознание нужно чтоб изменилось среда. Для объяснения развития нужно разорвать этот порочный круг.

Какие были пути? Кто-то предлагал географическую среду. Демографический детерминизм тоже не объясняет ни развития, ни существования. Вводили понятие человеческой страсти. Все не только сводится к разуму. Происходит взрыв страстей, и тогда человек делает вопреки разуму. То же самое существует в обществе. Люди действуют во власти страстей. Разум в этом случае производная. Человеческие чувства. А откуда они берутся? Тут возможно 2 объяснения - страсти вытекают из природы человека. Он так устроен, что не может жить без страстей. Но природа человека неизменная. И нельзя объяснить изменения чем-то неизменным. (Теорема Эммы Нётер прим ред). Тогда остаётся искать объяснения в общественной среде, которая меняется. А чем создаётся среда – страстями. И круг опять замкнулся. Не получается.

Некоторые пытались выдвинуть на первый план понятие человеческого интереса. Заинтересованность человека, потребность его в чем-то. Общественный интерес подогревает страсти, требует для разума, что бы найти средства и т.д. и т.д. Значит, казалось, нашли выход их положения. Но возникает вопрос, а откуда берётся интерес. В разных странах далеко не одинаковые интересы. Но объяснение из природы, или среды то снова порочный круг замыкается.

Последняя попытка разорвать круг. Они делили людей на 2 категории: серые, обыватели, которые только усваивают среду. Если бы такие все люди, то ничего не менялось бы. Но кроме обывателей появляются люди талантливые, у них возникают новые мысли, новые идеи, что этот строй не годится и они пропагандой своих взглядов изменяют общественного мнения.

Круг разорван, но ценой того, что история двигается волей людей. Это всегда носило название волюнтаризма. То есть они переходили на позиции идеализма.. И надо сказать, что было масса сложностей, у них получилось, так что великий человек творит историю по своему произволу. А где корни идей у великого человека? Нет.

    Они не могли найти объективного источника общественных идей. Источник – среда. Но объективного, не зависящего от сознания человека. Они не могли найти социальную материю. Материю природную они ввдели, а социальную материю они не видели. Но раз общественное сознание ни чем объективным не определяется, то они оно субъективно, они не могли отыскать объективный источник общественных идей.

\subsection{Противоречия французских материалистов}

Все их попытки построить целостную материалистическую теорию кончились крахом. С одной стороны они были абсолютными детерминистами. Всё в истории предопределено. А с другой стороны, великие люди, что хотят то и делают. С их точки зрения в мире нет ничего случайного, все связи необходимы и достаточны. Когда они объясняют всё абсолютно не обходимо - это тоже самое что всё абсолютно случайно. Всё может быть, любая песчинка может изменить ход истории. Они надеялись на то, что в силу случайности на троне окажется мудрый монарх, к которому придут философы и объяснят какой плохой существующий строй и монарх всё изменит к лучшему.

    Видим вопиющее противоречие. Фатализм – детерминизм. Необходимость – случайность. С одной стороны свободы нет, с другой люди что хотят то и делают. Вот такое вопиющие противоречие пронизывало всю философию французских материалистов. Они оказались в состоянии раздора сами с собой. Они не решили не только важнейших вопросов общественной жизни, то и ввели противоречия в понимании природы.

Когда разразилась ВФР, казалось что так всё и было. Был феодальный порядок. Феодальные порядки критиковали, люди мирились, но не понимали, что феодализм противоречит природе человека. Наступил XVIII век. Появились умные люди, они начали пропаганду, и вспыхнули революции. Где же движущая сила? В умах просветителей. Потом эти идеи отстаивали социалисты-утописты. Всё зависит от того, какие люди когда появились. Ну, надо сказать, что когда началась ВФР, то хотя она формально подтверждала волюнтаризм. Наполеон как бы доказывал волю и всемогущество великих людей. Меняю границы, создаю государства, назначаю королей, что хочу то и делаю. Но потом что получилось, что он оказался на остове. То есть какие-то объективные силы. После ВФР стало понятно, что старый строй Франции был обречен. Строй был обречен, он не мог не рухнуть. Эти перемены не могли не произойти. Напрашивался вывод, что действуют какие-то объективные силы.

Как раз в это наблюдается возрождение . Идея о том что есть некая сила, которая определила ход развития истории. И она не преодолима и не могла не быть. Она была развита людьми, которые ненавидели революцию. И тем самом был подготовлен новый взгляд на историю. Этот взгляд присущ творчеству немецких идеалистов в частности Гегеля.
