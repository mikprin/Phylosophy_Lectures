Философские взгляды классический немецких идеалистов
Представители

\begin{itemize}
    \item Иоганн Готлиб Фихте 1762-1814
    \item Фридрих Вильгельм Шеленг
    \item Георг Вильгельм Фридрих Гегель 1770-1831
\end{itemize}

Это и есть школа классического немецкого идеализма. Ещё есть Фейербах, который был материалистом. Понять Бекона, Локка понять можно. А вот немецкие классические философы писали очень сложным языком, и прочитать очень трудно. Но у них за внешней сложностью скрываются великие открытия. Надо было расшифровывать язык Гегеля. Это не так просто. Они четко ясно не осознавали, что они сделали. Они сделали действительно великий вклад в идеализм. Они преобразовали всё философию и открыли дорогу для дальнейшего её развития.

Во второй половине XIX века просвещение проникало и в Германию. В то время самой передовой страной была Англия. На этом фоне Германия воспринималась как отсталое государство. Была абсолютная монархия. Это был конгломерат множества 360 государств. Где-то к XIX веку осталось 30 государств. В результате такого отсутствия централизации наличия границ, таможни не было единого пласта немецкого общества. Германия гнила.

Французская революция привела к возникновению конституционной монархии. Возникло изменение власти. И это приветствовалось немецкой буржуазией. Но после начала Якобинского террора буржуазия поняла, что она бы этого не желала. С одной стороны они хотели преобразований, с другой стороны они не хотели революции. Поэтому они думали что, в конце концов, власть образумится и поумнеет. С одной стороны буржуазия была за преобразования, с другой стороны против революции.
\subsection{Имманул Кант. (Агностицизм)}

Родился в Прусском королевстве, в городе Кенигсберге. Родился в семье не богатого ремесленника, в семье было несколько детей, но он выделялся своими редкими способностями, и его отдали в гимназию, потом окончил в Фридрих-коллегию. После окончания коллеги в 1740 году поступает в университет. В 1745 его заканчивает.

Почти 10 лет работает домашним преподаванием в семьях богатых людей. Но в свободное время работает над своим образованием и в 1755 году защищает диссертацию, и получает звание приват-доцента. Стал помощником библиотекаря кенигсбергского университета. В 1770 становится профессором кафедры философии.

    Кроме философии читал курсы физики и математики, причем был в этих областях настоящим профессионалом. Был естествоиспытателем.

Линчая жизнь не сложилась, он не когда не женился, вел строгий, размеренный образ жизни.

Высоко почитал труды Жан Жака Руссо.

    Кант приветствовал ВФР с одной стороны, а с другой стороны опасался войн и кровопролития.

В 1793 году Кант написал работу <<Религия в пределах только разума>>, она поставила под сомнения доказательства бытия Божьего, когда эту работу показали прусскому королю Вильгельму IV, он воспретил Канту дальнейшее раскрытие этих тем в лекциях. И Кант вынужден был приостановить деятельность в этом направлении до смерти короля.

Развитие го философских исследований можно выделить 2 основных периода. Грань между ними 1770 год, тот самый год, когда он занял место профессора.

\begin{itemize}
    \item Докритический период.
    \item Критический период.
\end{itemize}

Во второй период мысли Канта получили наиболее яркое воплощение в трёх работах. И все эти три работы начинались со слова <<критика>>. Первая работа назвалась <<Критика чистого разума>> (1781). Самая крупная и самая глубокая работа Канта, спустя 7 лет в 1788 вышла вторая крупная работа <<Критика практического разума>> и в 1790 вышла работа <<Критика способностей суждения>>, где он подвел итог всем предыдущим работам.
\subsection{Докритический период}

Докритичекий период Кант занимал позиции близкие к материализму. И здесь он занимался проблемами естествознания. Кант создал первую гипотезу возникновения Солнечной системы, в дальнейшем она была подхвачена знаменитым французским физиком и математиком Лапласом. Прекрасно знал современную физику. Высоко ценил работы Ньютона, но ставил вопрос несколько по-другому.

    С их точки зрения Солнечная система это то, что неизменно. Кант создал не натурфилософскую гипотезу, а космогоническую гипотезу, то есть она объясняла все те факты, которые люди знали о Солнечной системе. Она базировалось на фактах, и объясняла факты.

Суть теории состоит в том, что в начале была скопление частиц, началось вращение, возникали сгустки, началось вращение, создавались сгустки, из сгустков планеты, спутники и Солнце. Кант пытается понять место Солнца и Солнечной системы в Галактике. Но этим он не ограничился, Сделал предположение, что таких систем типа нашей Галактики существует множество. Такие галактики вместе взятые образуют Вселенную. И попытался отождествить некоторые туманности с галактиками.

\subsection{Критический период}

До этого периода теория познания была вполне материалистическая. Но в 1770 происходит переворот, он перестаёт заниматься проблемой естествознания и все силы направляет на разработку теории познания. Некоторые историки считают, что Кант первым обратил внимания на теорию познания, однако это не совсем так.

Идея была такая: Мы познаём мир, и необходимо выяснить на что познание способно, а на что нет. Причем он считал, что это надо сделать раньше, чем приступить к познанию. Кант обращает внимание на 3 формы научного знания:
\begin{enumerate}
    \item Математика
    \item Естествознание (как обобщение фактов и создание теории). Теоретическое естествознание.
    \item Метафизика или философия.
\end{enumerate}
Научное знание является всеобщим и необходимым. Знание может быть только априорным. В математике кроме априорного знания ничего нет. Знания базируются на понятиях пространства и времени. Для него не сомненно, что это истинное и всеобщее знание. Второе это теоретическое естествознание. В отличие от математики, есть знания апостериорные. Возникает вопрос отношения априорного и апостериорного знания. Для Канта очевидно, что первые 2 формы дают истинное научное знание, вопрос только в том, а как они дают. А вот когда речь идёт о метафизике, возникает вопрос, а наука ли метафизика? А способна ли она дать знания.
Критика чистого разума

В своей работе <<Критика чистого разума>> он уделяет наибольшее внимание, и тем самым как идёт процесс человеческого знания.

    Существуют вещи, которые существуют, не зависимо от сознания он называет вещами в себе (ноуменами). Эти вещи воздействуют на органы человека.

Восприятие это комплекс ощущений, отражающих предмет в целом. С точки зрения Канта восприятие человека не имеет ни малейшего сходства с вещами в себе, то есть восприятие не несет никакой информации какие вещи сами по себе. Мы не знаем ничего о них, кроме того, что они есть.

    Возникает мир восприятия, не имеющий никакого сходства с миром реальным. То, что возникает в нашем сознании это Феномен. Есть 2 мира: ноуменов и феноменов (как потом назвали <<вещь для нас>>).

Есть 2 мира: есть мир <<вещей в себе>> и есть <<мир вещей для нас>>. Мир вещей в себе абсолютно не познаваем. Феноменальный мир – объект знания, а ноуменальный мир – объект веры. Мир ощущений и восприятий является полным хаосом, нагромождением беспорядочных ощущений и событий. Нужно навести в этом хаосе порядок. Этот мир преобразуется при помощи априорными формам, какими являются время и пространство. Время и пространство существуют только в феноменальном мире. Они существуют без опыта и вне опыта. Наложение связей в мире феноменов находиться при помощи категорий рассудка. При помощи этих связей познавающий превращает хаос в порядок и закономерный движущийся мир.

Кант выделяет следующие категории рассудка: 
\begin{itemize}
    \item Категории количества
    \begin{enumerate}
        \item Единство
        \item Множество
        \item Цельность
    \end{enumerate}
    \item Категории качества
    
    \begin{itemize}
        \item Реальность
        \item Отрицание
        \item Ограничение
    \end{itemize}

    \item Отношения

    \begin{itemize}
        \item Субстанция и принадлежность
        \item Причина и следствие
        \item Взаимодействия
    \end{itemize}

    \item Категории модальности

    \begin{itemize}
        \item Возможность и невозможность
        \item Существование и несуществование
        \item Предопределённость и случайность
    \end{itemize}

\end{itemize}

Все категории берутся из разума. Эти ощущения синтезируются, происходит то, что он называет категориальный синтез. Чувственный синтез + категориальный синтез дают знания. > Кант впервые показал, что наше знание о мире не является пассивным отображением реальности, а является результатом активной творческой деятельности человека. Категории выработаны тысячелетиями существования человечества, и для каждого человека они являются априорными. Вводит понятие транцедентального единство концепций. Знание входит в единое познание, которое едино.

Ошибкой Канта, по версии материализма, считается то, что он считал, что сознание создаёт мир, а на самом деле речь идет о воссоздании объективного мира в сознании. Сложностью было доказательство объективности феноменального мира, что Канту так и не удалось.

С точки зрения Канта мышление делится на 2 формы (ступени)

\begin{itemize}
    \item Рассудок.
    \item Разум.
\end{itemize}
Эти три раздела претендуют на то что они науки. Каждая из них решает свои проблемы, при помощи категорий рассудка мы с вами строим мир где существуют связи, законы.

При создании феноменального мира происходит категориальный синтез при помощи категорий рассудка, но до конца этот синтез не может быть доведён, так как он остаётся в пределах феноменального мира. Но кроме рассудка Кант выделяет ещё одну форму мышления - Разум. В отличие от рассудка Разум требует целостной картины мира, но это сделать нельзя, так как для этого нужно проникнуть в мир в себе, а это по Канту сделать не возможно. Таким образом, Разум ставит задачи неосуществимые, но ставит.

Это стремление разума довести синтез до конца выражается в появлении трёх идей разума: психологическая космологическая *теологическая или философская.

Психологическая и теологическая не могут дать окончательного знания. А вот космологическая идея у него широко разработана. Кант выделяет там три четыре вопроса: это вопросы о границах мира во времени и пространстве (конечен мир или бесконечен), второй вопрос: мир состоит из частей или мир целостен и един, третья проблема: есть ли свобода в мире и четвёртая проблема: существует ли необходимое верховное существо в качестве либо причины мира, либо необходимого элемента.

Кант показывает, что когда мы занимаемся решением этих проблем, то разум встречает непреодолимые противоречия. На каждую из этих проблем разум может дать два совершенно противоположных ответа и оба ответа можно доказать. Эти противоречия он называет антиномиями человеческого разума. Существуют ли эти противоречия реально в мире? Нет. Эти противоречия возникают, только когда разум пытается выйти за пределы мира для нас.
\fbox
\parbox{\textwidth}{%
        То есть это показывает, что есть пределы человеческого разума.
    }%
   }

Для первых трех пар антиномимй, он утверждает, что они ложны. То есть оба утверждния в антиномии(конечен мир или бесконечен; целостен или состоит из частей; есть свобода или нет) суть ложны. Человеческое тело как феномен подчинено принципу причинности и поэтому в мире феноменов оно не свободно, но человек как вещь в себе свободен в мире ноуменов. Философия занимается выяснением границ познания.
Критика практического разума

Вслед за работой <<Критика чистого разума>> Кант написал работу под названием <<Критика практического разума>>. В центре его внимания мораль.

    В обществе всегда есть моральные нормы, которые действуют как принудительные силы.

Кант рассуждает, что такое мораль, откуда она возникает, пытается выяснить всеобщие законы морали, для этого нужно отвлечься от содержания морали, так как у разных народов разная мораль. В качестве общей нормы, которая для всех обязательна, он предлагает <<категорический императив>>.

    Суть его такова- действуй так, чтобы норма твоего поведения была бы и нормой всеобщего законодательства(иногда переводят как- действуй так, как ты хотел бы что бы другие действовали по отношению к тебе). Только тот человек морален, который подчиняется категорическому императиву. Важнейшее понятие у Канта это понятие долга. Человек всегда должен следовать велениям долга. Любое отклонение аморально.

Но если человек соблюдает <<категорический императив>>, потому что ему это самому нравится, но это поведение не моральное, а легальное. Исполнение долга не обязательно ведёт к счастью. Но это не справедливо, поэтому нужно найти выход. Поэтому Кант вводит три понятия: свобода воли, бессмертие души и понятие Бога. Это всё недоказуемо, это всё понятия практического разума, доступные только вере. В мире в себе существует свобода воли и выбора, для того чтобы человек нес ответственность за свои поступки, человек за благодеяния получает бессмертие души, а Бог играет роль верховного судьи и воздаёт за благодеяния и наказывает за нарушение моральных норм.
