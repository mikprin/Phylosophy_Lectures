

\section{Введение}

\subsection{Развитие философской мысли}

Человечество появилось приблизительно 1.8 млн лет назад. Они не были похожими на нас, а были скорее существами, переходными между людьми и животными. Это были формирующиеся люди, которые жили в формирующемся обществе. Шел процесс так называемого антропосоциогенеза, становления человека и общества. С его завершением и возник вид Homo Sapiens - появилось человеческое общество. Это событие произошло 40 тысяч лет назад. С этого момента начинается история подлинного человеческого общества, она составляет крайне незначительный момент истории человечества как биологического вида. В истории человеческого общества мы должны выделить 2 основных этапа: эпоха первобытного общества и эпоха цивилизованного, или классового, общества.

Первые цивилизации возникли около 5 тыс. лет назад, т.е. 35 тысяч лет первобытности и 5 тысяч лет цивилизации. Большинство историков выделяет в этом периоде несколько эпох. Первая эпоха – эпоха Древнего Востока. Она началась приблизительно в 31 веке до н.э. и длилась до 8 века до н.э. В эту эпоху философии не существовало. Не было и науки, была лишь преднаука, однако был собран и систематизирован огромный материал, который в дальнейшем и лег в основу науки и философии античного мира. Да и в античном мире наука была еще далека от современной и находилась в стадии становления(так называемая "пранаука").

    С VIII века до н.э. начинается эпоха Античного мира, которая окончилась в V веке н.э.(Крушение античного мира). Ей на смену приходит эпоха Средних Веков, длившаяся до XVI века(по этому вопросу есть разногласия, но это мнение большинства ученых). С XVI века начинается эпоха Нового Времени. Выделяют раннее средневековье(V – X века) и позднее средневековье(XI - XV века). Возрождение XIV-XV века, находится в рамках Средних Веков. XVIII век – век Просвещения. В XX веке начинается уже Новейшее Время(2е десятилетие: 1914г или 1917г). В XVI веке произошла первая в истории человечества буржуазная революция, возникла первая капиталистическая страна - Голландия. XVII век – произошла английская буржуазная революция, по масштабу и влиянию большая, чем голландская. В конце XVIII века произошёл крупнейший перелом того времени – Великая Французская Революция(14 июля 1789г), событие, оказавшее влияние на всю Европу.

\subsection{Падение античного мира}

В античном мире произошел бурный прорыв, настоящая культурная революция. Началась она в Древней Греции. Бурный рассвет Античного Мира в последствии сменился его упадком. И когда говорят о его упадке, то обычно всегда связывают его с варварами. Варвары ворвались на территорию Западной Римской Империи, разрушили ее и создали множество своих, так называемых, варварских королевств. Культура погибла и только потом начала возрождаться снова. Но нужно учитывать, что варвары были всегда, и всегда они соседствовали с античным миром. Но античный мир не только не гиб, а расцветал. Он душил варваров. Кроме того большинство рабов были именно варварами (древний Рим – общество рабовладельческое). Античный мир прежде, чем погибнуть, начал угасать, начался его упадок. Древняя Греция развалилась в 4м веке до н.э. и центр тяжести переместился в Рим. Что в результате внесло новую жизнь в античный мир. Но потом и в Риме тоже начался упадок (где-то 2-3 век до н.э.). Упадок охватил все области, в том числе духовную и культурную. Наука приходит в упадок. Если раньше греческие ученые занимались теорией, то римские ученые занимались только прикладными делами. Сухопутное, военное дело, архитектура, инженерное дело.

И такую же деградацию испытывает и философия. Мы знаем, что философия возникает в 6 веке до н.э. благодаря великим представителям первой школы, Милетской школы, Фалесу, Анаксимену и Анаксимандру. После этого развитие философии идет по восходящей линии. Делаются все новые и новые открытия и в конце концов философия достигает своего наивысшего развития в трудах Аристотеля (VI век). После этого идет упадок философской мысли, развитие по нисходящей. Нельзя однако сказать, что развитие было только по нисходящей. Были и взлеты, определенные достижения, открытия, но в целом был регресс.

В античном мире философия возникла вне связи с религией. Она с самого начала носила светский характер. Дальше деградация выразилась в том, что светская философия начинает трансформироваться в религиозную. Это можно заметить у стоиков. Особенно это заметно на примере неоплатоников. Неоплатонизм – это уже религиозная философия. Дальше она постепенно превращается в богословие и философские проблемы отходят на задний план. Это видно у последних представителей неоплатонизма. На первый план выступает не разум, а вера. Открытий нет, эта философия не могла быть теорией познания. Философия перестает быть наукой и становится чистой религией. Так что когда вторглись варвары, античный мир находился в состоянии ужасающего упадка во всех областях. Варвары нанесли окончательный удар по ослабшему миру и погубили его. Но погиб он сам.

Варвары завоевали Западную Римскую Империю, но была еще и Восточная Римская империя (Византия, хотя сами они себя так не называли, это название дано историками). Византия сохранилась. Варвары туда вторгались, но сокрушить ее не смогли. Она пала под ударами турков и прочих спустя 1000 лет. А вот философии там не было, исчезла, угасла. Богословие было, еще немножко логики, использовавшейся в богословских спорах. Тем более, что император Юстиниан запретил философию и приказал все философам покинуть империю под угрозой казни. Шла борьба с остатками философской мысли. Осталось только богословие и христианская философия, но и она угасла.
Возникновение классового общества

Для коренного населения Западной Римской Империи завоевание ее варварами было огромным регрессом, а вот для варваров наоборот – прогрессом. Они использовали многие достижения римлян и у них возникли государства и классовые общества, т.е. они стали цивилизованными. Античная цивилизация была использована для создания варварской цивилизации.

    Признаком, отличающим обычно классовое общество от первобытного, считается: 1. Создание монументальных каменных сооружений; 1. Письменность.

Можно сказать, что варвары, с одной стороны, погубили античную цивилизацию, хотя она сама давно сгнила, а с другой стороны, они усвоили христианство, а вместе с ним и каменное зодчество (сооружались и сохранялись каменные храмы в христианстве), письменность. Правда надо сказать, что если в античном мире письменность одно время была повседневным делом, то у варваров письменность была достоянием верхов общества, а низы оставались неграмотными. И все-таки письменность сохранялась. Варвары усвоили письменность и в этом смысле варварские королевства были классовыми обществами. Варвары не только разрушали, но и пытались сохранить. Варварские короли создавали королевства, государства с государственным аппаратом, огромными подвластными территориями, управление которыми требовало существования в обществе образования, письменности. Так что варвары пытались кое-что сохранить из культуры античного мира. Это особенно хорошо можно наблюдать на территории Италии, которая был ядром Западной Римской Империи.

Надо сказать, что когда говорят о крахе Римской Империи, то обычно называют дату 476 год н.э. Но эта дата весьма условна, ведь вот уже 20 лет как римские императоры никакой властью не пользовались, а реальной властью обладали предводители варварских военных дружин. Римские императоры опирались на наемные войска из варваров, которые в конце концов превратили императоров в своих марионеток - смещали, назначали новых, правда сами не решались быть императорами. Говорят о инфильтрации варваров в римское общество, проникновении их во все сферы. Фактически Италию не завоевывали, но она была под властью варваров. И в конце концов одному из варварских предводителей, командующему армией Рима (Одоакр), надоело это, и он сместил последнего римского императора, Ромула Августа. Столицей Восточной Римской Империи стал Константинополь, а Одоакр объявил себя королем Италии. Это и считается концом Римской Империи. К тому времени уже вся её территория был разделена на варварские королевства.

Спустя некоторое время в Италию вторглись остготы, завоевали ее, создали варварское королевство, которое называлось Остготским королевством, и в Италии воцарился Теодорих Великий. Это произошло в 493 году. Окончательно оформилось варварское королевство – Остготское королевство. Теодорих Великий пытался навести порядок в Италии, довольно милостиво обращался с коренными жителями. Фактически в Италии существовало 2 общества: остготское и старое римское общество. Надо сказать что он, будучи заинтересованным в эффективном управлении, широко привлекал <<старые>> кадры.
Возникновение средневековой философии
Северин Боэций

Именно здесь, при дворе Теодориха Великого, мы встречаем первого средневекового философа. Этим первым средневековым философом был Северин Боэций (около 480-524). С него и начинается история средневековой философии. Это был коренной римлянин, не варвар конечно, человек необычайно образованный, прекрасно знавший не только латинский, но древнегреческий язык, что было в то время необычайной редкостью. Северин Боэций – его называют <<последний римлянин>>, имея в виду античных римлян.

Я его называю так – последний философ античности и первый философ средневековья. С него мы и начнем историю средневековой философской мысли. Северин Боэций родился около 480 года, а дата его смерти известна точно – 524 год. Как необычайно образованного человека Теодорих привлек его на государственную службу, он занимал высокие посты при Остготском королевстве Теодориха Великого, практически был своеобразным <<премьер-министром>>. Но участь его было плачевна: его заподозрили в заговоре против короля, какой-то переписке с Константинополем и т.д., он был арестован, судим, приговорен к смертной казни и казнен. Казнен он был не потому что был философ, а из-за различных политических интриг. Была ли истина в обвинениях? Возможно, что да. Но окончательно это историки так и не установили.

Сидя в тюрьме, он написал работу, которая навсегда вошла в историю философской мысли, называлась она <<Утешение философией>>, потому что ничего другого в камере ему не оставалось. Написана она было конечно на латинском языке, учитывая его положение, в работе пропагандируется стоическая моральность - что будет, то будет, изменить ничего нельзя, нужно отнестись ко всему достойно, что он и сделал. Однако это была не единственная его работа, были и другие. Он занимался переводом на латинским язык трудов Аристотеля, правда все это были логические труды, не философские. Следует отметить, что в последующее время европейцы не знали никаких других трудов Аристотеля, кроме логических. В частности, Боэций перевел на латинский язык работу Порфирия, который написал "Введение к "категориям" Аристотеля". К введению к этой работе Боэций написал комментарий, где вслед за Порфирием четко сформулировал одну из важнейших проблем философии, а именно проблема отношения отдельного и общего.

Эта проблема красной нитью проходит через всю историю философии, она же <<проблема о природе понятий>>. Все понятия выражаются общими, содержанием понятия является общее. Пример: содержанием понятия <<собака>> является не какая-нибудь конкретная собака, а вообще собака. Позже эта проблема получила название <<Проблема универсалий>> и заключается в том, существует ли общее или нет независимо он сознания человека?

    Ответ на этот вопрос в свое время дал Платон, который заявил, что общее существует на самом деле, существует не зависимо от сознания людей и образует свой мир, мир идей. В учениях Платона было два мира: мир идей и мир вещей. Причем каждый мир вещей является преходящим, ибо каждая вещь появляется и исчезает. А мир идей – это мир универсальный, вечный, потому что, скажем, каждая конкретная собака рождается и умирает, а <<собака вообще>> живет вечно. Есть мир идей, он порождает мир вещей. Это первая постановка и решение проблемы отдельного и общего, которая в средневековой философии получила название реализма - общие понятия существуют на самом деле независимо от сознания людей).

Именно Порфирий первый четко сформулировал эту проблему. Вслед за ним это повторил Боэций, попытался дать свое собственное решение, но решение иное, чем Платоновское, ближе к Аристотельскому (общее существует на самом деле, но существует в конкретных общих вещах, особого мира не существует). Вот эту точку зрения и отстаивал Боэций. И надо сказать, что в дальнейшем вся средневековая философия – это борьба вот этих двух точек зрения: реализма и номинализма. Пока что мы встретились с одной точкой зрения, которую называют умеренным реализмом (общее объективно существует, но существует конкретно в каких-то вещах). Это точка зрения Боэция.

Труд Боэция был широко известен, и в дальнейшем его широко использовали средневековые философы, обосновывая ту или иную точку зрения. Таков вклад Боэция, через него в основном и знали античную философию. Надо сказать, что кроме перевода, Боэций организовал переписывание философских трудов и был создателем целого ряда оригинальных философских работ.
Аврелий Кассиодор

Важнейшей задачей просвещенных людей в то время было сохранить знания античного мира. И эту задачу выполнял не только Боэций, но и его ближайший друг, иногда говорят ученик, - Аврелий Кассиодор. Его роль в сохранении античного наследия велика. Родился где-то в 480 или 490 года, умер в 580 году. Он тоже был придворным чиновником, приближенным к Теодориху, но после смерти Боэция решил завязать с чиновничей деятельностью, подал в отставку и ушел в монастырь (540 год). Он создал христианский монастырь и при нем грандиозную библиотеку. Он спасал где возможно книги по различным областям знаний. Так что монастырская библиотека Кассиодора – это был кладезь античной премудрости. Он организовывал переписку этих книг, пересылку их в другие монастыри. Он пытался систематизировать античное знание – античную науку и философию. Он написал руководство – такую своеобразную энциклопедию всех существующих знаний. Его призыв был <<не нужно заниматься новыми поисками, а нужно сохранить все старое>>. И как раз к Аврелию Кассиодору уходит знаменитое <<Положение о семи искусствах>> - совокупность всех достижений античного мира. И в дальнейшем вплоть до эпохи Возрождения во всех школах и университетах преподавали эти семь знаменитых искусств. Слово искусство подразумевало у него и науку, и знание и т.д. Мало кто знает, что когда в старину говорили об артистах, то подразумевали ученых людей. > Эти семь искусств он поделил на 2 группы:

    1. три искусства – тривиум, три пути, трехпутье. К ним относилось: грамматика (умение читать и писать), риторика (умение говорить), диалектика(логика, знание о том, что такое понятия, суждения, знание силлогизма); 1. квадривиум – четырёхпутье, считалось более высоким уровнем знания: арифметика (умение считать, совершать арифметические операции), геометрия (здесь кое-что сохранялось от Евклида), музыка (умение пользоваться инструментами, знать ноты и т.д.), астрономия (представление о том как устроен мир, основывались на учениях Птоломея (Земля – центр Вселенной, вокруг нее обращаются планеты, Солнце), но если Птолемей считал Землю шаром, то здесь считалось, что Земля – блин…)

Это все, что оставалось от античных знаний, все было сведено в этих 7 искусствах. Надо сказать, что библиотека в монастыре, основанном Кассиодором не была случайной, были и другие монастыри, которые тоже занимались собиранием, хранением достижений античной культуры. Но в основном монахи знали латинский язык, греческий был во многом забыт. Как ни странно, только в далеких монастырях Ирландии сохранялись знатоки греческого языка. Кстати, когда говорили о философии, то имели в виду в основном только неоплатонизм. Подлинники Платона, Аристотеля были не доступны. Так как большинство были неграмотными, грамотными были только монахи, да и то не все, священники, в то время возникла монополия церкви на духовную жизнь. Вся культура была под контролем церкви и носила религиозный характер.

В некоторых монастырях были школы, которые учили монахов и иногда людей, не принадлежащих к ним, но не во всех, и школ таких было не так уж и много. Были и другие попытки как-то все это соединить. Например, был такой епископ города Севилья в Испании Исидор Севильский. Он создал грандиозную книгу под названием <<Этимологий>>, где пытался также записать все знания, оставшиеся от античности.

Это целый начальный период 5 век, 6й век, начало 7го. Что дальше происходит? Варварские королевства воевали между собой. В результате из них выдвинулось самое крупное – королевство Франков, которое существовало на территории нынешней Франции. Франки – это германское племя, которое завоевало Галлию, а потом слились с галлами, усвоили французский язык и стали французами. Франкское королевство стало самым могущественным. Оно отбило натиск арабов в 8 веке, которые завоевали всю Испанию и пытались завоевать остальную Западную Европу, спасли таких образом Западную Европу от исламизации. В конце концов франкские короли покорили все до единого варварские королевства и под их властью оказалась вся территория Западной Европы, исключая только Англию (значительная часть Испании( Северная Испания), значительная часть Италии, вся нынешняя Западная Германия, вся Франция…). Возникла грандиозная держава, которая называлась империей Каролингов, т.к. своего могущества она достигла при Карле Великом, который в 800 году провозгласил себя Римским императором, и формально было установлено Западное Римское королевство.

Когда возникло такое огромное государство, которым нужно было управлять, возникла нужда в образовании, культуре, грамоте и началось возрождение духовной жизни. Наступило так называемое <<каролингское возрождение>>. Самым активным деятелем этого возрождения был некий Алкуин , который жил и работал при дворе Карла Великого. Он был своеобразным министром просвещения. Он создал первую светскую школу при дворе Карла, которая называлась Палантинская школа(придворная), где изучали все семь искусств, готовили специалистов, чиновников, администраторов и т.д. Сам Алкуин был англичанином, англосаксом из графства Йорк, и был приглашен Карлом Великим, где он и занялся возрождением образования. Впервые в эпоху Каролингов начали возникать школы не только в монастырях, но и при крупных городах, там где были епископы, при кафедральных соборах (епископальные школы), куда принимали не только монахов, но и обычных людей. Алкуин не был философом, она его мало интересовала - роль его в сохранении и распространении знаний велика.
Иоанн Скотт Эриугена

Каролингское возрождение привело к тому, что появился первый настоящий средневековый философ, который преподавал в палатинской школе при дворе наследников Карла (его внука Карла Лысого). Сам был либо ирландцем, либо шотландцем. Это был знаменитый Иоанн Скотт Эриугена. Родился около 810г. Умер где-то после 877 года. Примерно с 840 года работал преподавателем в Палатинской школе. И он создал первую средневековую философскую систему. Иоанн Скотт Эриугена поставил проблему соотношения веры и разума. Он конечно был истинный христианин, верующий, абсолютно убежден, что Священное писание – это авторитет.> Но все-таки вера верой, но ее недостаточно, нужен и разум. Нужно, чтобы все положения писания принимались не только на веру, но и были аргументированы разумом. <<Дело в том, - говорил он, - что в священных писаниях все истины христиан содержатся в образной форме, не в виде абстрактных рассуждений, а в виде неких образов>>. Нужно было раскрыть, что за ними скрывается. Истину нужно сделать достоянием разума, и разум должен истолковывать писание. Писание бесспорно верно, но его нужно истолковывать. И вот здесь нужен разум. Все должно быть так истолковано, чтобы было согласовано с требованиями разума. Вера нуждается в обосновании. И вообще-то первична вера, а потом разум, но в познавательном смысле разум выше веры.

Из-за такого своего взгляда он был человеком одиноким, во многом был забыт. В 13 веке вспомнили о нем, и когда начали читать, Папа Римский решил его осудить, а книги его сжечь. Надо сказать, что он также занимался проблемой общего мира и создал целую законченную систему картину мира. За основу он брал Платона, но Платона в неоплатонистской трактовке. Нарисовал такую картину:

    1й мир – это Бог, Бог выше всего, это то, что творит; 2й мир – это Божественный разум, которое содержит в себе идеи всех вещей(своеобразная переработка Платона, у которого был мир идей, который существует сам по себе, не завися от Бога), это 2я природа, с одной стороны сотворенная Богом, но в то же время творящая(реалист); 3й мир – это вещи, порожденные миром идей То есть у него фактически триада – общее существует как Божественный мир, общее существует в конкретных отдельных вещах и общее существует в человеческой голове. Надо сказать, что его взгляд потом так или иначе разделяли многие средневековые философы. Как видно, он был реалистом. Это такой умеренный момент, который шел от Аристотеля. Кстати, он знал Аристотеля, знал Платона, но не все работы. Был одним из тех, кто знал греческий язык, что к тому времени было вообще большой редкостью. Это и позволило ему создать такую более или менее законченную философскую систему.

\subsection{Распад христианской церкви. Зарождение схоластики}

Церковь распадается на 2 ветви: католическая (вселенская) и православная (лат. – ортодоксальная, та, которая держится за все и не отступает). Главой католической церкви был Папа Римский. Ему подчинялись все епископы Западной Европы, которые были владельцами монастырей, так что в этом смысле у него была прочная основа, и когда королевская власть практически исчезла, он стал единственным в Западной Европе, кто имел власть. Короли оказались у него в подчинении, кроме того у него в руках была огромная сила – предание анафеме. Если кто-то не повиновался, он отлучал от церкви всю страну и запрещал служение. Так что можно сказать, что началась борьба между Папой Римским и местными владыками и Папа Римский давил на то, что он – повелитель Западной Европы и короли должны быть у него в услужении. Им это не очень нравилось, но они вынуждены были с этим считаться.

В эпоху полного распада Папа Римский стал играть огромную роль. Он был арбитром и от него зависело многое. Надо сказать, что наряду с этим шел другой процесс – процесс возникновения городов. При варварском завоевании практически все города исчезли. Они сохранились как столицы государств, как торговые пути, но не более того. Но городов в нашем понимании в Западной Европе не было. Эти города стали возникать тогда, когда завершился синтез(???) С X – XI века они стали возникать повсеместно, причем такие, каких никогда в греческой истории не было. Торгово – промышленные города, которые обладали определенной автономией, которые постепенно избавились от власти феодалов, завели свою армию и прочее, стали силой. Возникли местные рынки, развивалась внешняя торговля, и стали складываться рынки в пределах целых регионов: Германии, Франции, Испании. Именно это экономическое явление легло в основу преодоления феодальной раздробленности, возникновения единых централизованных государств. Короли начали опираясь на города подавлять сопротивление феодалов. В конце концов Папе пришлось примириться с тем, что ему удастся сохранить только духовную власть, но не светскую.

Когда стали возникать города как центры ремесла, как центры торговли, то им понадобились знания и образование. Началось настоящее возрождение образования. В городах стали возникать школы. Хотя они и были под контролем духовенства, но они были уже не при монастырях. В них обучались городские жители. В городах начинает получать распространение грамотность. Нужны были знания как вести торговлю, правовые знания. Возникают специализированные школы (знаменитая медицинская школа в городе Саверна; и не только там, школа права в итальянском городе Болонья…). В специалистах нуждались и города, и короли, которые создавали теперь единое централизованное государство. Кстати, здесь возникает слой людей, которые занимаются преподаванием(так называемый слой средневековых интеллектуалов). Сначала их целью было распространение знания, а затем и получение иного знания (намечается в X-XI веках, а получает развитие в дальнейшем). Появление такого количества интеллектуалов привело к оживлению философской мысли, появились масса философов. Развернулась борьба между реалистами и номиналистами (сторонники понятий универсалии, реализм).

    Номинализм – точка зрения, согласно которой общего в мире нет, мир есть скопление конкретных отдельных вещей, и кроме них никаких других вещей нет. Номиналисты разделились на два направления: крайнее направление(согласно которому общего нигде нет, ни в мире, ни в уме, а есть конкретные отдельные вещи и есть слова, которые являются колебаниями воздуха и являются просто знаками - радикальный номинализм) и умеренный номинализм, или концептуализм - общее есть только в уме, в виде понятий, есть слово и есть понятия.Понятие есть обозначение общего, общего в мире нет, а оно создается людьми на основании того, что между вещами есть сходство, которое улавливается и создаются понятия.

\subsection{Ансельм Кентерберийский}

В X-XI веке философов было немало, одним из самых крупных был Ансельм Кентерберийский. Он родился в 1033 году (умер в 1109 году), родом из города Аоста из Италии. До самой смерти был епископом Англии. Он снова поставил ту же проблему, которую ставил Эриугена – проблему веры и разума. Но он решал ее по другому, он решал, что в отношениях веры и разума ведущей является конечно вера, а разум является служанкой, но разум нужен, веры недостаточно. Мало, что человек поверит, его еще нужно убедить в том, что такие положения являются истинными, от этого его вера только укрепится, поэтому разумом нельзя пренебрегать. Надо сказать, что в этом смысле он был материалистом, также, как и Скотт Эриугена. Вообще схоластика носила такой рационалистический характер, они обращались к разуму, и всегда признавали, что важна не только вера, но и разум, и пренебрегать им нельзя. Более того, разум необходим. > Ансельм Кентерберийский настаивал на том, что вера выше разума и разум нужен для того, чтобы обосновать положения веры. Все нужно доказывать, вплоть до того, что нужно доказывать бытие Божее. Он первым выдвинул так называемое онтологическое доказательство бытия. Он считал, что он неопровержимо доказал, что Бог есть (Мы мыслим о Боге как о существе, которое сверхсовершенно, обладает всеми совершенствами, а совершенное существо должно существовать, ибо совершенство включает в себя существование) В общем он считал, что нужно верить, чтобы понимать. Нужно разум поставить на службу религии. Был реалистом.

У него были противники. В то время в средневековье начал зарождаться номинализм, причем в крайней форме. И вот таким первым крайним номиналистом был Иоанн Росцелин(ок.1050-ок.1122). Был признан еретиком, его преследовали, все книги уничтожили. С его точки зрения в мире кроме конкретных вещей ничего нет. Но как же с Богом? Он говорил, что Бог есть, его не быть не может. Но ему нет места.
Пьер Абеляр(1079-1142)

Самым крупным философом того времени был Пьер Абеляр(1079-1142). Автор нефилософской работы, которая навсегда вошла мировую литературу <<История моих бедствий>>. Пожалуй второе произведение в литературе, в которой человек пытается передать свои собственные интимные переживания (первым был Августин Блаженный). К тому времени начало оформляться новое явление в развитии духовной жизни Европы и философии – появление философских школ. Некоторые из кафедральных школ специализировались только на философии. Они отличались направленностью, т.е. люди из одних школ придерживались одинаковых взглядов.

Возникают три школы на территории Франции: Шартрская, Сен-Викторианская и Парижская, три разных направления философии. Абеляр обучался в Парижской школе, с самого начала отличался необычайной самостоятельностью, он никогда ничего не принимал на веру и донимал преподавателей всякими каверзными вопросами. Обладал большой самоуверенностью и смотрел на многих студентов, когда был молодым, как бы свысока. Когда получил образование и начал преподавательскую деятельность, у него была уже слава смутьяна, слава дерзкого, самоуверенного человека, для которого ничего святого нет. В результате создал свою собственную школу. Набрал немало людей, которые учились у него. Но он был великолепным оратором, хотя и нажил себе много врагов. Он был приглашен в качестве учителя (давал уроки) к некой Элоизе, дочери почтенных граждан. Он ее обучал философии, и они влюбились друг в друга. В результате они вступили брак, но брак тайный, ибо родственники Элоизы не хотели этого. Об этом узнал дядя Элоизы, который считал, что Абеляр соблазнил Элоизу. В результате он решил ему отомстить. По его наущению группа людей превратила Абеляра в евнуха. После этого Элоиза ушла в монастырь, он стал на всю жизнь одиноким. Абеляр описал эту драму в своей книге.

Абеляр был одним из самых выдающихся мыслителей этого периода, называвшегося ранней схоластикой (X-XI век). > С его точки зрения философия – это наука об истине, ее цель – найти истину и философия – это есть метод, при помощи которого ищут истину и отличают истину от заблуждения. Человек ищет истину, значит ему нужна диалектика, которая является методом, при помощи которого ищут истину и отличают истину от заблуждения. При этом человек руководствуется разумом. Конечно он считал, что вера нужна, но ее недостаточно (он занимался богословскими проблемами). Провозглашал, что в отношении веры и разума разум выше веры. Если Ансельм Кентреберийский говорил, что нужно верить, чтобы понимать, то Абеляр – понимать, чтобы веровать, и в этом отношении разум выше веры. Все должно быть подчинено разуму, все должно быть аргументировано и доказано. Он написал работу, которая называлась <<Да и нет>>. В ней он собрал высказывания, содержащиеся в священном писании и церковных догмах, и показал, что в них имеются противоречия. Составил большой список из противоречий. А если в писании говорят и то и другое, то где же истина? По закону исключенного третьего, одно из утверждений неверно. Значит либо в писании либо в трудах отцов церкви есть ошибки. Он вывода этого не делал, но к нему подходил. > Он считал, что нужно начинать не с верования, а с сомнения. Все ставить под сомнение и потом проверять разумом. Он возродил идею, причем в более яркой форме, Ионна Скотта Эриугены. Таким образом он потряс основы, сам того может быть и не желая. Он был одним из первых концептуалистов, т.е считал, что общего в мире нет, но существуют понятия и они носят общий характер, а это создание общих понятий происходит на основе сходства реальных предметов. Они не отражают объективный мир, но у них есть основа в объективном мире.