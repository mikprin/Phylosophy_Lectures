

\section{Поздняя схоластика. Западноевропейская религиозная философия.}
Лекции по философии науки проф. Семенова Юрия Ивановича
Лекция 3, семестр 9, 17. 9.2005





\subsection{Возникновение университетов}

В течение XIIв. в Европе происходят существенные изменения, которые сказались на XIIIв. Так как возникли и расцвели города с их торговлей и промышленностью, то возникла нужда в образовании и образованных людях, поэтому повсеместно стали создаваться школы. Раньше они были монастырскими, далее добавились кафедральные и епископальные школы, а теперь стали возникать городские. Некоторые из них специализировались (там изучалось право, медицина), некоторые были широкопрофильными. Некоторые школы были известны во всей Западной Европе, и туда стекались люди, которые хотели получить хорошее образование.

Возникала нужда в интеллектуалах, преподавателях, людях, занятых интеллектуальным трудом. Интеллектуалы понимали, что чтобы хорошо преподавать, нужно самим заниматься исследованиями. Но в то время опыт был не в чести. Число интеллектуалов стало непрерывно увеличиваться. Появилось огромное число школяров - обучающихся в этих школах. Это была буйная ватага, они часто перебирались из школы в школу, из одной страны в другую сторону, тем более что преподавание везде велось на латыни. Латынь была живым языком в среде интеллектуалов. Эти школяры назывались вагантами. Они создавали произведения на тематики, отличные от духовных, возникла знаменитая поэзия вагантов. Ваганты менее всего занимались восхвалением существующей системы власти. Поэзия была сугубо иронической, сатирической, поэтому их преследовали. XIIв. - начало расцвета поэзии вагантов. Они бродили из школы в школу, жили чем можно, стипендии не было, не брезговали кражами и воровством, в общем - шумная разношерстная толпа.

Возникла нужда в организации тех, кто занимается и готовится заниматься интеллектуальным трудом. Когда умножилось число интеллектуалов, когда они стали не одиночками, а массой, стало необходимо организовать корпорации интеллектуалов - своеобразный цех.

    Такого рода корпорации стали возникать и добиваться самоуправления, автономии. Эти корпорации получили название университетов.

Начиная с XIIIв. они стали возникать по всей Европе. Они добивались независимости от церковных, городских и королевской властей. Полиция не могла вмешиваться в дела университетских городков. Существовала университетская автономия.

Обычно университет состоял из 4х факультетов.

    Факультет искусств
    Факультет права
    Факультет медицины
    Факультет богословия

Младший - факультет искусств (на нем преподавались семь свободных искусств). Студенты на нем - артисты (от латинского слова art). Обычно поступали в 14 лет и 6 лет учились. После окончания получали диплом доктора и право преподавать. Выпускник мог заняться преподавательской деятельностью, а мог перейти на один из трех старших факультетов - факультет права, медицины и богословия. На медицинском факультете срок обучения состовлял 6 лет. Выпускник защищал диплом и получал доктора медицины. На богословском - 7 лет, столько же на праве. Обычно декан факультета свободных искусств был ректором университета.

Университеты могли переезжать в другой город, но были выгодны городам, т.к. туда стекались массы людей. Угрожая уходом, заставляли городские власти идти на компромиссы. Нередко университеты учреждались декретом папы. Таким образом, они попадали в зависимость от папской власти. Но она была далеко, так что это было предпочтительнее власти местной.

В XIIIв. возникли Парижский, Оксфордский, Кэмбриджский университеты и множество других по городам Европы. Нужны были все новые и новые знания. Основной научной работой было чтение и выписки из книг. В поисках требуемых новых знаний интеллектуалы обратили взгляды на Восток. В начале контакты шли через Сицилию, а потом главной связью стала Испания. Испанские короли отвоевывали новые и новые земли, на которых была арабская культура.

Город Толедо был очагом арабской культуры. Его освободили от арабов испанцы в 1085г. Там были крупные учебные заведения и библиотеки. Все было на арабском языке, нужны были переводы. Возникла коллегия переводов в Толедо. Знания в области геометрии, арифметики, алгебры, астрономии, медицины и т.д. Переводились труды как греческих авторов, так и арабских. Эвклида, Архимеда, Гиппократа, Птолемея, арабских ученых - Аль-Хорезми. Переводили Авиценну. Наконец, философов: Платона, Аристотеля. (долгий путь - с греческого на сирийский, с сирийского на арабский, с арабского на латынь). Это не способствовало ясности, многое терялось и искажалось. Раньше европейцам из греческих авторов были известны только неоплатоники.

Работы Аристотеля были переведены на латынь, в т.ч. "Метафизика" - главный его труд. Перед европейцами встала грандиозная фигура античной философской мысли. Они стали овладевать системой Аристотеля. Встал вопрос, как использовать его систему. Аристотель не знал Христа и был нехристианским философом. Начались споры на высшем уровне - папства, руководства университетов. Начиная с середины XIIIв. Аристотель начал победное шествие по интеллектуальным уголкам Европы. Аристотелизм столкнулся с системой Августина Блаженного (августинианство), который пытался скрестить неоплатонизм с христианством. Фигура Аристотеля была слишком значительной, и с ней считались даже ярые августинианцы.
\subsection{Философы Схоластики}
\subsubsection{Августинианцы}

Бонавентура (Джовани Фиданца) (1223 - 1274) был итальянцем по происхождению, но в основном жил во Франции, Bon Aventura - "благая весть". Первая фигура в схоластике XIIIв. Был генералом ордена францисканцев. Бонавентура преподавал, был богословом и философом.

Главные ордена того времени - орден францисканцев (Франциск Ассизский) и доминиканцев (монах Доминик). Ордена соперничали, францисканцы принимали линию Августина, а доминиканцы - Аристотеля. Domini Canos - "господь"+"собаки". Именно доминиканцы преследовали и обрекали на смерть еретиков - "псы господни".

Пытался создать философскую систему на основе августинианской в противовес системе Аристотеля. В его системе было много мистики. Он доказывал, что мир не вечен, был сотворен богом из ничего, в мире все - единство формы и материи, только Бог есть чистая форма. Форма - свет. Разные существа отличаются степенью совершенства - света.

Он считал, что чувственное познание не всегда нужно. Так же, как не всегда нужен интеллект. Чувственное познание тогда, когда мы имеем дело с внешним миром, после чувств нужна работа интеллекта. Главное познание - не мира, а Бога, а его мы познаем разумом. Существует три пути познания Бога:

    Через познание вещей (вещи - творение Бога).
    Через познание души (более современный способ).

    Мистика, слияние с божеством (высшая ступень). Наибольший момент счастья.

    Главное - вера. Знание подчинено вере.

\subsubsection{Средневековый аристотелизм}

Против Бонавентуры выступили доминиканцы. Крупнейшие - Альберт Великий, и самый крупный Фома Аквинский.
Альберт Великий

Альберт Великий (Альберт Фон Больштедт) (1193..1206-1280) Преподавал в Парижском университете, затем в Кельне. Именно через его работы философия и богословие средневековой Европы восприняло идеи и методы аристотелизма. "Доктор универсалис" - грандиозная эрудиция. Знал греческие, римские работы, арабскую и еврейскую философию.

    Занимался наукой - зоологией, ботаникой. Большую часть знаний он почерпнул у других авторов, но другую - из собственного опыта! Это выражало его грандиозную ученость. Считал, что Земля шарообразна. Стал различать философию и богословие, обосновав это тем, что философия непосредственно связана с разумом, а в основе теологии лежит вера.

Встал вопрос об отношениях разума и веры. Он считал, что есть истины-откровения и истины разума, а поскольку источник - Бог - един, то они не могут противоречить. Разум не вступит в противоречие с верой. Разум дает нам только знание о внешнем мире, а теология идет дальше и раскрывает недоступное разуму.

    Путь веры - путь выше разума. Разум должен быть подчинен вере.

Фома Аквинский

Законченной системы Альберт Великий не создал. Один из его учеников, Фома Аквинский (итальянец, 1225..26 - 1274) (Thomas) создал систему томизма. Его философия в XIXв. была обновлена и признана официальной философией католической церкви - неотомизм. У него есть последователи по сей день. Создал целостную всеобъемлющую систему, где пытался объединить все знания средневековья, подобие энциклопедии. Синтезировал положения христианского богословия и философии Аристотеля. Пытался доказать верность положений христианской церкви, а Аристотеля брал как орудие. Пришлось многое подчистить, т.к. Аристотель не был христианином.

Главные работы Фомы: "Сумма теологии", "Сумма против язычников". Как и Альберт, продолжает различать теологию и философию, два вида знания - разум и веру. Но! Философия должна быть служанкой богословия. Примат веры. Существуют, говорил он, истины, не поддающиеся объяснению разумом - сверхразумные. Такие истины - главнейшие. (Пример: о личности бога - един ли он в трех ликах) Бытие Бога можно доказать, опираясь на опыт: "в мире все движется=>что-то приводит это в движение=>это Бог!" "есть причинно-следственная связь, значит есть первопричина. Это может быть только Бог". Мир сотворил Бог - из ничего. Идеи Аристотеля: форма существует субстанциальная и окцидентальная. Наша душа - форма субстанциальная, частичка Бога. Отстаивает идею бессмертия души. Блаженство человека в созерцании божества. Сторонник реализма, считает, что идеи, универсалии существуют трояко (из Авиценны):

    до вещей, в божественном разуме,
    в вещах,
    в нашем уме в форме понятий.

Пытается разработать проблемы теории познаний. Исходит из того, что существует объективный мир, созданный Богом. Все начинается с контакта органов чувств человека с вещами, они пронимают в наше сознание в виде вида, образа, нематериально. Единственный источник знаний о мире - чувственное познание. Но чувственного познания недостаточно для познания сущности, глубинных связей вещей. Фома ярый монархист, идеальная форма правления для него - монархия. Монарх, с его точки зрения - не просто правитель, а творец монархии по божьему велению. Монарх, однако, управляет только телами подданных, а их бессмертная душа под властью церкви. Власть духовная выше власти государей. Государство карает преступников, но еретики страшнее фальшивомонетчиков, ибо портят души. Они подлежат смерти.

И Фома Аквинский, и Бонавентура - святые католической церкви.

Главное положение православия: святой дух исходит из Бога-отца, католицизм: и от сына.
\subsubsection{Материалистический аристотелизм}

Были и аристотельяне-враги Фомы. В этом направлении была попытка понять Аристотеля материалистически. Латинские аверроисты. Крупнейший представитель - Сигер Брабантский. (ок1235 - ок1282). Преподавал в Парижском университете, был популярен, но высшее церковное руководсто привлекло его к суду инквизиции, вызван к папскому двору и был убит своим секретарем.

    В его учении Бог есть, и он первопричина мира, но мир был всегда. Мир совечен Богу. В дела мира Бог не вмешивается. Существуют причинно-следственные связи и нет никаких чудес. Есть общечеловеческий разум. Он воплощается в душах людей, но души людей смертны.

Положение о двойственности истин - истина откровений может не совпадать с истиной разума. Это попытка освободить богословие из-под контроля церкви. Напрашивался вывод, что служение Богу не требуется, за что автор и поплатился. Взгляды передовые.
\subsubsection{Роджер Бэкон}

В XIIIв. появилась проблема соотношения веры, разума и опыта. Один из тех, кто попытался поставить и решить эту проблему - знаменитый Роджер Бэкон (ок1214-ок1294). Англичанин. Учился в Оксфорде, Парижском университете, начал преподавать в Оксфорде (в Париже сильно прижимали). Был членом ордена францисканцев. Его взгляды очень не понравились главе ордена Бонавентуре. Он в 1257г. запретил Бэкона и отправил его в монастырь. В дальнейшем заключил в монастырскую тюрьму за непокорство (в 1278г.). В ней он пробыл до 1292г.

    Имел четкую социальную позицию: обличал феодальную вольницу, беззаконие королевской власти, духовенство, церковь, обвинял в разврате, критиковал поведение высших представителей духовенства.

Он считал христианство одной из шести сект. Считал, что философия должна быть тесно связана с наукой и базироваться на их данных. Она должна помогать наукам искать истину. Важнейшее - размышления об истоках знания. Говорил, что существует три источника знания:

    Авторитеты и их труды
    Мнения простых людей
    Опыт

Первый - ненадежен. Второй тоже может подвести. Настоящий, подлинный источник - опыт. Развитие наук, говорил он, приведет к тому, что человек подчинит объективную реальность. Пророчествовал техпрогресс: летательные машины, повозки от неживой силы. Он верил в магию, занимался наукой, его интересы и взгляды переплетались.

XIIIв. - наивысший подъем схоластики.

