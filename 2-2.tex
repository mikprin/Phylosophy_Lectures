
Лекции по философии науки проф. Семенова Юрия Ивановича
Лекция 2, семестр 10, 18. 2.2006
Введение и Теория познания

Отформатировано: Проверено:
Продолжение предыдущей лекции

В прошлый раз мы начали с вами курс философии, а конкретно теории познания, разобрали материализм, субъективный и объективный идеализм, убедились в том, что материализм и идеализм - это противоположные направления, которые по-разному решают основной вопрос философии. И в то же время между материализмом и идеализмом есть общее: и материалисты и идеалисты признают, что есть вещи и есть мысли, что есть бытиё и сознание, есть два вида реальности - идеальная реальность, материальная реальность. И в тоже время все они считают, что существует только один единый мир; две реальности, но только одна из них образует мир, другая же является производной от первой. И в этом смысле материализм и идеализм являются формами монизма.

Для лучшего понимания понятия монизма и иных понятий, которые мы будем проходить, вспомним понятие субстанции, которое у некоторых философов лежало в основе построения. Оно наиболее ярко выступало в философии Спинозы. Субстанция – это вполне самостоятельно бытие, бытие, которое не обязано своему существованию никакому другому бытию. Как говорил Спиноза: "Субстанция - принина самого себя". Это бытие совершенно независимо, кстати, другого-то и нет. Акциденция – это различное проявление субстанции. Субстанция – вечна, а акциденция – временная. В материализме субстанция – материя, а сознание – производная от материи. А с точки зрения идеализма, есть только одна духовная субстанция. Вещи есть, но они все производные от духовной субстанции.

Кроме материализма и идеализма существуют еще несколько направлений, но они не являются основными.

    Материя и сознания – два самостоятельных бытия, они существуют рядом и ни про одну из них нельзя сказать, что она первична. Такое течение называется дуализм. Представитель – Рене Декарт(главный и практически единственный представитель). Но надо сказать, что дуалистическую точку зрения Декарт до конца последовательно выдержать не мог. В его философии есть материальная субстанция - мир, где нет никаких чудес, нет никаких вмешательств духовной субстанции, и есть субстанция духовная, которая выступает в роли временной духовной субстанции - душа человека. Также есть постоянная духовная субстанция - Бог, душа - это производная от Бога. Бог породил мир. То есть есть духовная субстанция, которая порождает и материальное и духовное, а значит это монизм.
    Ещё одно направление – отказ от основного вопроса философии, основателем считается Давид Юм. Раскрыть, что первично, а что вторично абсолютно не возможно. Эту ветвь начали называть агностицизмом(а - не, гносис - знание, <<не знание>>). Этот термин впервые ввел Генрих Гексли. Мы знаем мир, как он является в нашем сознании. То есть мы не можем решить вопрос, есть ли объективная реальность. Мы знаем только явления, но не знаем сущность. Позднее эту точку зрения назвали феноменализм(от лат. фенОмен - явление). Сейчас это господствующее направление в западной философии современности – позитивизм.

Еще один вопрос о соотношении философии и религии. Из материализма вытекает, что нет никаких сверхъестевенных сил, в том числе и Бога. Другое дело, что не всегда все материалисты были до конца последовательны, были среди них и пантеисты(пример: Спиноза)(мир и Бог - одно и то же) – это была попытка избавиться от религии, сохраняя старую терминологию. Вообще говоря, суть религии она в теизме – то есть вере в то, что есть некое верховное существо, которое сотворило мир, которое наблюдает за ним, и постоянно подправляет действия людей. Построения объективных идеалистов во многом напоминает построение религиозное, нередко приходит вывод, что идеализм и религия – суть одно и тоже. Это точка зрения абсолютно не верна. > Религия имеет существенную практическую составляющую – культ. Целая система действий, чтобы эта сила помогала, а не пакостила. Вера проявляется в делах, если нет дел, то и веры нет - разве можно назвать верующим человека, который ходит в церковь только чтобы венчаться, окрестить детей? Многие идеалисты были атеистами, например ученики Гегеля – младогегельянцы. С точки зернения юмистов опровергнуть или доказать существование Бога не возможно, а поэтому заниматься этим и не имеет смысла. Все работы Юма были признаны запрещенными, а сам он считался еретиком. Религия и идеализм, совершенно разные вещи. Религия требует веры. Но ни одна система идеализма не аппелировала к вере, а всё доказывали разумом.
Теория познания
Тема 1. Проблемы природы знания и познания

Мы будем заниматься познанием о познании, умозрением об умозрении.
Эмпиризм и сенсуализм

Юм был эмпириком и сенсуалистом, что привело его в агностицизм. Значит ли это, что эмпиризм и сенсуализм всегда ведёт к агностицизму? Ничего подобного, может привести как к субъективному идеализму, так и к материализму.

Эмпиризм – от греческого эмпирея – опыт, это точка зрения, согласно которого единственным источником знаний о мире является опыт. Эмпирики делят знания на две категории; одна категория - это знания, полученные в опыте (a posteriori - апостариорное) и априорные (a priori) знания(существующее без опыта и до опыта). С их точки зрения, всякое знание является апостериорным, априорного знания нет и быть не может. Но что же такое опыт? В целом, под опытом понимается какая-то деятельность. Нет деятельности - нет знаний. Пока деятельность не началась – нет знаний.

> Кстати, деятельность не обязательно осуществляется руками, также она может осуществляться головой

Это направление определилось в начале Нового Времени и первым представителем эмпиризма был Бекон(16-17 век). Дальше эту линию подхватил Томас Гоббс. Надо сказать, что они были эмпириками, но не занимались исследованием познания. Т.е. детально не обосновывали положение о том, что все знание апостериорно. Впервые этим занялся Джон Локк(17 век). Он детально исследовал процесс эмпирического познания. Он детально доказывал, что знания следуют только из опыта, что пока нет опыта, наше сознание представляет собой чистый лист бумаги. А в процессе опыта оно заполняется знанием. Локк подверг критике концепцию рационализма, доказывал, что никаких врожденных идей нет и быть не может, не может быть никакого априорного знания.

Многие философы сводили весь опыт к чувственному опыту, т.е. к взаимодействию между нашими органами чувств и внешним миром (Внешний мир действует на органы чувств человек, возникают ощущения и восприятия - это и есть чувственный опыт). А значит органы чувств – единственный канал, по которому поступают знания. Это и называется сенсуализмом. Нет ничего в разуме, чего не было бы в чувствах. Они не отрицали мышление, не сомневались, что есть разум, но разум не располагает никаким другим материалом, кроме как материал чувственного познания. Как видите, сенсуалисты – эмпирики, но те, которые сводят опыт к чувственному опыту. Не все эмпирики были сенсуалистами, например Джон Локк считал, что есть два вида опыта, внешний опыт(чувственный опыт) и внутренний опыт - полученный в результате умственной деятельности. Однако это была непоследовательность - ведь если есть источник знаний иной, даже называемый опытом, то откуда берется это знание?(За это его критиковали другие сенсуалисты: Пристли, Гартли, Коллинз, Толланд и пр.) Сенсуалисты говорят, что все наши, даже самые абстрактные построения уходят корнями в чувственныое познание, внутреннего опыта нет. Любой внутренний опыт - лишь результат внешнего.
Разновидности сенсуализма

Не трудно понять, что сенсуализм не является единым течением, где все сторонники придерживались одной точки зрения. Вернее, они были едины только в одном: никакого другого источника знаний, кроме чувтсвенного знания, не сущесвтует. а дальше они разделялись на несколько направлений. Беркли, Юм, Локк, Бекон, Гоббс, французские материалисты. Существует по меньшей мере 3 вида сенсуализма – субъективный идеалистический (Беркли), феноменалистический(агностический)(Юм и его последователи, включая позитивистов) и материалистический. Они делятся по отношению к источнику ощущения. Материализм – действие объективного мира на органы чувств (есть объективный мир, который порождает ощущения, порождает восприятия, которые являются образами внешнего мира), берклианец – нет никакого источника, ощущения - единственное, что существует, феноменалисты(Юмисты) – а я не знаю, этот вопрос абсолютно не разрешим. Вопрос об источнике ощущений - это формулировка(разновидность) основного вопроса философии.
Проблема природы восприятия

Вопрос этот разбирается не только в философии. Этим занимаются конкретные науки, это и психология, физиология органов чувств. В чем же отличие? По физиологии существует 5 видов ощущений. Ощущения – это отражение объективных свойств предметов внешнего мира в нашем сознании. Все ощущения образуют целостный образ предмета – восприятие предмета. Любое восприятие – комплекс ощущений. Говорить о восприятиях и ощущениях с точки зрения философии – одно и то же. Есть ещё и третья форма чувственного познания. Например, я вспомнил яблоко, образ всплыл у меня в сознании. И вкус, и цвет и т.п. Это называется представление. Возникает своеобразный мозговой отпечаток предмета – энграмма. Представление и восприятие – не одно и тоже.

Вопрос о том, как происходит процесс восприятия – не философская проблема. Философская проблема – об источнике восприятия с одной стороны, и с другой - о природе восприятия. Здесь вопрос о природе восприятия – это вопрос об отношении между предметом и восприятием этого предмета. А это опять переформулировка основного вопроса философии. Предмет восприятия и восприятие предмета - это одно и тоже или нет? Предмет существует вне сознания, он материален, а восприятие - существует в сознании и оно субъективно, зависит от строения, состояния мозга. В этом смысле восприятие субъективно, а предмет же объективен. Но с другой стороны, что является содержанием восприятия? Объективный предмет, который подействовал на органы чувств. Если восприятие есть образ внешнего мира, то содержанием восприятия является предмет, и в этом смысле предмет восприятия и восприятие - одно и то же. Например, фотография и человек. Это не одно и тоже, однако, восприятие похоже. Вполне понятно, что восприятие может быть одно и то же и не одно и то же. Предмет объективен, а восприятие и субъективно и объективно. Это выражается в понятии субъективный образ объективного.
Проблемы природы понятий и субъективный идеализм

Восприятие и предмет восприятие – это одно и не одно. А раз так, мысля по принципу формальной логики, то можно взять одну сторону и постараться не заметить второй стороны. Это называется абсолютизация. Если мы заявим, что предмет восприятия и восприятие – одно и тоже, то получится, что мир существует только в нашем сознании, и другого мира кроме восприятия нет и быть не может. Это точка зрения Беркли, субъективного идеализма. Это заблуждение, но в этом заблуждении есть частица истины. Когда Беркли говорит, что восприятие и предмет - одно и то же, он берет только одну сторону. Здесь истина возведена в абсолют и становится заблуждением. В этом примере можно видеть гносеологические корни идеализма.

    Ни одно явление не возникает без причин. Всегда есть какие-то причины. Когда говорят об идеализме, то выделяют два вида причин. Одна из – социальная. То есть один определённый класс, который заинтересован в искажении взгляда на мир. Есть такие вещи, которые нужны для изменения взгляда на мир. Например, взгляды Коперника, который назвал землю одной из планет, противоречило с представлением церкви о мировоззрении. Интересы церкви требовали сохранения старой гипотезы, для сохранения своего влияния на народ, потому как усомнение в каких-то догмах церкви могло поколебать позиции власти. Гносеологические корни идеализма состоят в том, что процесс познания - чрезвычайно сложный процесс и можно всегда заглядеться и выдернуть только одну сторону, то обязательно получится субъективный идеализм.

Кант: ноумены и феномены. Вещи в себе и вещи для нас

Если мы раздуем сходство, превратим его в тождество, то окажемся на позициях субъективного идеализма. Но если пойдём по другому пути, то получится, что между восприятием и предметами не будет никакого сходства, они только и только различны. Появиться два мира – мир восприятия и мир объективных предметов. Между ними нет ничего общего.

    Это точка зрения Канта. Есть объективный мир. Эти объективные вещи дейтсвуют на рецепторы организма и возникает восприятие. Всё это пока в рамках материализма. Восприятие не имеет ничего общего с этими предметами, и не несёт никакой информации. Значит, о внешнем мире мы ничего не знаем и ничего не узнаем. Вот эти вещи Кант назвал ноумены – вещи в себе. А вещи, которые мы ощущаем и знаем о них. Эти вещи в нашем сознании Кант называл феноменами – вещь для нас. Ввёл понятия <<вещи для нас>> и <<вещи в себе>> ввёл Гегель. Не на пустом месте возникло раздувание, абсолютизация различия между вещами и ощущениями. Термины <<вещь в себе>> и <<вещь для нас>> очень трудны для понимания, но очень удобны. Сейчас они понимаются шире, чем понимал Кант. Более широкие определения дали Гегель, Энгельс и Ленин. Что же они понимали по вещью в себе? Вещь в себе – это вещь вне сознания, вещь для нас – вещь в сознании. А Кант имел ввиду под этими терминами вещи, которые существуют только в нашем сознании и только в объективном мире. У Канта и Беркли полностью противоположные точки зрения. Однако, один из миров Канта и мир Беркли – полностью совпадают.

Взгляды Юма. Агностицизм и феноменализм

Юм был убедждённым эмпириком и ни на шаг не хотел отступать от сенсуализма, хотя это до конца не получалось. С точки зрения Юма мы живём вне нашего сознания. Единственный источник знаний – наши ощущения. Но откуда берутся ощущения? По Юму существует три гипотезы: есть объективный мир, который воздействует на органы чувств, никакого мира нет (Беркли), и третья точка зрения - есть Бог, который рождает мыслящим духам ощущения. И какое же верное решение? Мир такой, какой он в наших восприятиях. Но выйти за пределы мира восприятий мы не можем. Мы можем знать, что в наших восприятиях, но не можем посмотреть, что с другой стороны. Доказать и проверить мы не можем, так как выйти за пределы этого круга невозможно. Нельзя ни доказать, ни опровергнуть ни идеализм, ни материализм.
Великое открытие философии: мир существует в нашем сознании

Общее у Юма, Беркли и Канта – мир существует в нашем сознании. И с точки зрения материалистов – мир тоже существует в нашем сознании, и это бесспорно. И в этом отношении все философы едины. А существует мир вне сознания – это уже разные ветви философии. Это отличает философа от обычного человека. У обычного человека точка зрения наивного реалиста – мир един и находится вне сознания, что отрицает познание как таковое. И философия сформировалась тогда, когда стало ясно, что мир существует в нашем сознании. Первыми это поняли представители Эллейской школы (Ксенофан 6-5 век, Пармемид и Зенон). Вспомним их точку зрения:

    У Парменида основным понятием было Бытиё, основным положением было – Бытиё есть, а Небытия – нет. Бытиё – вечно и абсолютно цельно. Оно не состоит ни из каких частей, оно вечно, а значит неизменно. И поэтому есть одно единство и нет никакого многообразия. Ничего не возникает и не исчезает, движения нет. Но как эту концепцию совместить с наблюдаемым миром? Мы видим, что все вещи исчезают и возникают,что есть многообразие. Вывод прост: всё это иллюзия, многообразный мир не имеет бытия. Он существует только в нашем сознании. А каким путём мы получаем знание об объективном мире? Умозрением, так как чувства не дают ничего. Реальное бытие - это только умозримый мир. Они оторвали сущность от явлений. Они открыли чувственное и умственное познания, пришли к выводу, что мир существует в нашем сознании. А теперь рассмотрим точу зрения Демокрита. Он был один из создателей атомистического материализма. Демокрит исследовал процесс чувственного познания. Мир – это атомы, атомы образуют вещи, они воздействуют на органы чувств человека и получается восприятие. Есть вещи, котороые обладают цветом, вкусом и запахом. А атомы не имеют ни в вкуса, ни цвета, ни запаха. Получается, что есть мир в нашем сознании, а есть мир атомов. Но нет резкой абсолютной грани. Атомы различны по форме, и мы испытываем разные ощущения. Мир в сознании отличается от мира вне сознания. Есть какая-то связь между мирами и мир для нас не абсолютно субъективен, поскольку характер ощущений определяется характером атомов. Потом идея заглохла, но через какое-то время возродилась. Эту точку зрения отставали Рене Декарт и Томас Гоббс, встречалась она в работах Галилео Галилея. Ну и наконец эта точка зрения получила детальное развитие в трудах Джона Локка, который создал концепцию первичных и вторичных качеств. Предметы внешнего мира, действуют на человека, и возникают в сознании. Предметы, которые появляются в сознании, обладают разными качествами – одни качества такие же как качества реальных предметов: вес, объём и тп. Они присущи реальным вещам и отражаются в нашем сознании. Это первичные качества. А, кроме того, есть вторичные качества у предметов – они существуют только в нашем сознании – запах, вкус и тп. Хотя и есть какие-либо объективные свойства у предметов, которые порождают именно такие ощущения - зеленого, красного. Но в мире нет ни зеленого, ни красного. Мир в сознании отличается от мира вне сознания. Вещи существуют двояко и есть различие между мирами. А дальше развитие пошло в сторону Беркли. Все качества существуют только и только в сознании, а значит они только и только вторичные. Значит, мир существует только в сознании. Философия привела к тому, что мир существет в нашем сознании. Пришел вопрос, что такое знание, что значит знать о какой-то вещи. Знать о вещи - значит эта вещь является содержанием нашего сознания. Это важнейшая вещь.

&nbsp;

Сайт создан с помощью CSS фреймворка Amazium
